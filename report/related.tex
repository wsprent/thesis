\chapter{Research in Related Problems}
\label{chap:related}


\section{The Steiner Tree Problem}
\todo[inline]{This section should be a section which highlights the research history of the STP
 which has been ``replicated'' in the PCSTP}
\section{The Prize-Collecting Travelling Salesman Problem and Variants}
\todo{Think about whether there should be a super-section about all the variants}
First introduced by \citet*{balas1989prize}, the Prize-Collecting Travelling Salesman Problem
(PCTSP) is a problem closely related to the
PCSTP.

The problem is formulated by \citeauthor{balas1989prize} as follows. Let
$G = (V, A, c, p)$ be a directed
graph with arc costs $c: A \to \RR_+$ and vertex prizes (or penalties) $p: V \to \RR_+$
then find a cycle $C = (V_C, A_C)$ with $V_C \subseteq V$ and $A_C \subseteq A$ which
minimises the function
$$\sum_{ij \in V_C} c_{ij} + \sum_{v \in V \setminus V_C} p_v$$
such that
$$\sum_{v_i \in V_C} p_i \geq B$$
for some $B \in \RR_+$.

If we recollect the definition of the PCSTP and its variants (Section \ref{sec:pcstp}) then
we immediately see some similarities. The PCTSP involves optimising the
\textit{Goemans-Williamson Minimization Problem} for the circuit $C$, however with the extra
constraint of collecting at least $B$ prize on the tour. This extra constraint draws comparisons
to the \textit{Quota Problem}.

As the PCTSP is a natural member of the family of Vehicle Routing Problems, it is also sometimes
stated with a \textit{depot} vertex \citep{feillet2005traveling} -- that is a vertex which
must be part of the solution cycle, $C$. This corresponds to the \textit{Rooted}
Prize-Collecting Steiner Tree Problem.

\paragraph{The Orienteering Problem}
The Orienteering Problem (OP) is very much the circuit version of the \textit{Budget Problem}.
Given a graph $G  = (V, E, c, p)$, the OP inolves finding a circuit, $C = (V_C, E_C)$,
which maximises
$$\sum_{v \in V_C}  p_v \quad \text{s.t.} \quad \sum_{ij \in E_C} c_{ij} \leq B$$
for some $B \in \RR_+$.
\todo{Add something short on the history of the OP}
\paragraph{The Profitable Tour Problem}
A maybe more familiar problem in the realm of prize collecting tour problems, is the
so-called Profitable Tour Problem (PTP). This variant -- which earlier shared name with
the PCTSP -- involves dropping the quota constraint from the PCTSP, making it a true
 ``cousin'' of the PCSTP.

 While similar in formulation to the PCSTP, according to a fairly recent survey by
 \citet{archetti2014chapter}, the PTP has received very little attention -- 
 especially when compared to the PCSTP.
 In fact, to the best of our knowledge, no exact solution to the PTP
 has been proposed outside the lower bounding procedure proposed by \citet{dell1995prize},
  which is a big step in defining an exact algorithm.
 
 However, there has been some research into approximation algorithms for the PTP.
 In fact, an interesting symptom of exactly how related these two families of
 prize collecting problems are
is how the GW Algorithm (see Section \ref{sec:solving:approx:gw})
can be used used to generate a 2-approximation algorithm for the PTP
on graphs satisfying the triangle inequality in the same way as
polynomial time MST algorithms can be used to generate 2-approximation algorithms
for the TSP \citep{goemans1995general}.



 \section{The Median  Problems}
 *general idea*
 
 *k-median*
 
 *median shortest path*
\subsection{The Median Ciruit/Tour Problem}
*short history*

*problem defintion*

*MIP formulation*
\subsection{The Median Subtree Problem}
*short history*

*problem definition*

*MIP Formulation*

*Maybe motivate a Median Steiner Problem*

%%% Local Variables:
%%% TeX-master: "report"
%%% reftex-default-bibliography: ("lit.bib")
%%% End:

