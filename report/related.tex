\chapter{Research in Related Problems}
\label{chap:related}
The \gls{pcstp}, while unique in formulation,
is part of a family of problems which somehow deal with the trade-off of
\textit{paying} for connecting vertices to gain their \textit{prize}.

In this chapter, we will look at a collection of similar problems, see how far research is
on them, and discuss similarities with the \gls{pcstp} both with respect to the motivation and
intutive understanding of the problems, but also to how the problems are approached and
how similar the solutions are.

By doing this, we aim to gain deeper insight into the structure of this collection of problems
 and separate generalisable methods from problem-specific methods.

\todo[inline]{Do examples for this section. Show the problems visually.}
\todo[inline]{Be clear about the intuitive differences between the \gls{pcstp} and these problems}


\section{The Prize-Collecting Travelling Salesman Problem and Variants}\label{sec:rel:tsp}
\todo{Think about whether there should be a super-section about all the variants}
First introduced by \citet*{balas1989prize}, the Prize-Collecting Travelling Salesman Problem
(PCTSP) is a problem closely related to the \gls{pcstp}.

The problem is formulated by \citeauthor{balas1989prize} as follows. Let
$G = (V, A, c, p)$ be a directed
graph with arc costs $c: A \to \RR_+$ and vertex prizes (or penalties) $p: V \to \RR_+$
then find a circuit $C = (V_C, A_C)$ with $V_C \subseteq V$ and $A_C \subseteq A$ which
minimises the function
$$\sum_{ij \in V_C} c_{ij} + \sum_{v \in V \setminus V_C} p_v$$
such that
$$\sum_{v_i \in V_C} p_i \geq B$$
for some $B \in \RR_+$.

If we recollect the definition of the \gls{pcstp} and its variants (Section~\ref{sec:pcstp}) then
we immediately see some similarities. The PCTSP involves optimising the
\textit{Goemans-Williamson Minimization Problem} for the circuit $C$, however with the extra
constraint of collecting at least $B$ prize on the tour. This extra constraint draws comparisons
to the \textit{Quota Problem}.

As the PCTSP is a natural member of the family of Vehicle Routing Problems, it is also sometimes
stated with a \textit{depot} vertex \citep{feillet2005traveling} -- that is a vertex which
must be part of the solution circuit, $C$. This corresponds to the \textit{Rooted}
\gls{pcstp}.

Intuitively, these kinds of problems seek to solve concrete problems with respect to transport routes.
Where the Travelling Salesman Problem seeks to find the optimal ``route'' to visit all vertices in
a graph, the PCTSP (and its variants) look to answer which routes/vertices are worth visiting and
which are not. This is reminiscent of how the \gls{pcstp} is to the Steiner Tree Problem.

By being routing problems, the PCTSP variants are more often than not defined on directed graphs in
contrast to the \gls{pcstp} which is exclusively defined on a undirected graph (while the Steiner Aborescence
 Problem is very much the directed variant).

\paragraph{The Orienteering Problem}
The Orienteering Problem (OP) is very much the circuit version of the \textit{Budget Problem}.
Given a graph $G  = (V, E, c, p)$, the OP inolves finding a circuit, $C = (V_C, E_C)$,
which maximises
$$\sum_{v \in V_C}  p_v \quad \text{s.t.} \quad \sum_{ij \in E_C} c_{ij} \leq B$$
for some $B \in \RR_+$.
\todo{Add something short on the history of the OP}

\paragraph{The Profitable Tour Problem}
A maybe more familiar problem in the realm of prize collecting tour problems, is the
so-called Profitable Tour Problem (PTP). This variant -- which earlier shared name with
the PCTSP -- involves dropping the quota constraint from the PCTSP, making it a true
 ``cousin'' of the \gls{pcstp}.

 Specifically, the PTP involves finding a circuit,
 $C = (V_C, E_C)$ in a graph which minimises
 $$\sum_{ij \in V_C} c_{ij} + \sum_{v \in V \setminus V_C} p_v$$
 with no additional constraints.
 
 While similar in formulation to the \gls{pcstp}, according to a fairly recent survey by
 \citet{archetti2014chapter}, the PTP has received very little attention -- 
 especially when compared to the \gls{pcstp}.
 In fact, to the best of our knowledge, no exact solution to the PTP
 has been proposed.
 However, \citet{dell1995prize} does define a lower bounding procedure for the problem
 which is a big step in defining an exact algorithm.
 
 There has, however, been some research into approximation algorithms for the PTP.
 In fact, an interesting symptom of exactly how related these two families of
 prize collecting problems are
 is how the GW Algorithm (see Section~\ref{sec:solving:approx:gw})
 is used by \citeauthor{goemans1995general} to generate a 2-approximation algorithm for the PTP
 on graphs satisfying the triangle inequality -- much in the same way that
 polynomial time MST algorithms can be used to generate 2-approximation algorithms
 for the Travelling Salesman Problem \citep{goemans1995general}.

 \todo[inline]{A branch-and-cut algorithm for the capacitated profitable tour problem}
 \paragraph{ILP Formulations}

 As part of developing an exact solution to the PCTSP, \citet{fischetti1988additive} stated the problem
 on a directed graph as
 an integer linear program. This formulation is shown in Formulation~(\ref{form:pctsp:cut}) for
 a graph $G = (V, A, c, p)$.
 We note the similarity between this and the cut-based \gls{ilp} (CUT-IP) formulation used by
 \citet{ljubic2005solving} for their variant of a Prize-Collecting Aborescence Problem
 (Formulation~\ref{form:exact:cut}).
 
 \begin{formulation}
   \begin{subequations}
     \begin{alignat}{3} %TODO: Find reference for this formulation
       &\underset{\bd x, \bd y}{\text{minimize}}
       & & \sum_{ij \in A} c_{ij} x_{ij} + \sum_{i \in V} (1 - p_iy_i)  & \\
       & \text{subject to}\quad
       & & \sum_{i \in V} p_i y_i \geq Q && \\ 
       &&& y_i = x(\delta^-(i)) && \forall i \in V \label{form:pctsp:prize}\\
       &&& x(\delta^-(i)) = x(\delta^+(i)) && \forall i \in V \label{form:pctsp:deg} \\
       &&& x(\delta^+(S)) \geq y_k && \forall S \subset V. r \in S, k \in V \setminus S \label{form:pctsp:conn}\\
       &&& \bd x \in \BB^{|A|} && \\
       &&& \bd y \in \BB^{|V|}.
     \end{alignat}\label{form:pctsp:cut}
   \end{subequations}
   \caption{ILP formulation of the PCTSP by \citet{fischetti1988additive}.}
 \end{formulation}

 Constraints (\ref{form:pctsp:prize}) and (\ref{form:pctsp:conn}) correspond directly to constraints
 (\ref{form:exact:pcsap:prize}) and (\ref{form:exact:pcsap:conn}) respectively in (CUT-IP).
 The latter set of constraints, the so-called connectivity constraints, ensure that every feasible solution
 is a single connected subgraph containing the root by enforcing the existence of at least a single edge crossing
 every cut which separates the root from some selected vertex. Combined with the
 additional constraint (\ref{form:pctsp:deg}), these ensure that any feasible solution is a single circuit.

 The similarities of the two formulations reflect the similarities of the two problems, with the main difference being
 the shape of the solution (circuit / arborescence).
 As a result there are also similarities in how these prize-collecting problems can be solved
 -- e.g. the method of solving a max-flow problem on a support graph
 to separate connectivity constraints or \glspl{gsec} can be directly applied to PCTSP variants, and with the GW Algorithm being
 directly applicable to the PTP, there is not much in the way from directly lifting a branch and cut routine from
 one problem to another.

 The main uniqueness of working with Steiner trees versus circuits in this case seems to be the possibilities of applying
 preprocessing routines to the input graphs.

 \todo[inline]{Either change title of paragraph or find more examples}
 \section{The Median Facility Problems}
 \label{sec:related:median}
 A type of problems which are closesly related to the ``prize-collecting'' problems such as the \gls{pcstp}, are the so-called
 ``median'' problems. These problems involve finding optimal location of a ``facility'' with regards to the sum
 of distances from all vertices to the facility.

 For example, first introduced in \citeyear{hakimi1964optimum} by \citet{hakimi1964optimum} is a problem known as
 the \textit{$p$-Median Problem}. Given a graph $G = (V, E, c)$, this involves selecting a subset $S \subseteq V$ with
 cardinality $p$ such that $S$ minimises the sum of minimum distances from vertices in $V$ to vertices in $S$.
 
 \citet{current1987median} adds structure to the selection of vertices by introducing the \textit{Median Shortest Path Problem (MSPP)}.
 Here, given a graph $G = (V,E,c,d,p)$ -- where $c$ and $d$ are edge weight functions --
 and source and sink vertices $s$ and $t$, one must find a facility
  -- now a simple path $P$ --  in $G$
 from $s$ to $t$ in s.t. the function
 $$\sum_{i,j \in E(P)} c_{ij} + \sum_{i \in V} p_i D_{i, P}$$
 where $p_i$ is the weight of vertex $v_i$ and
 $D_{i, P}$ is the shortest path from vertex $v_i$ to any vertex in $P$ with regards to
 weights $d$.
 We commonly denote the first sum as the cost of the facility/subgraph and the second sum as the assignment cost.

 Intuitively, the MSPP involves weighing a trade off between the cost of the facility
 in $c$ and the assignment costs. 
 This is analogous to the trade off between the cost of a subgraph and the cost of missed prize in the
 family of prize-collecting problems. In the median problems, the fixed prize of a vertex is
 replaced by the variable ``prize'' represented by the shortest path to the facility sometimes weighted
 by a vertex prize.
 \subsection{The Median Ciruit/Tour Problem}
 The problem of finding a circuit shaped facility in a graph has also been studied.
 \citet{labbe1999themedian} introduced the ``Median Cycle Problem'' which involves
 minimising the cost of a cycle and assignment cost in a mixed graph (undirected for
 the facility and directed for assignment) given a root vertex. Alongside this problem
 -- which they denote (MCP1) -- they also introduced a budget version (MCP2) where the
 assignment cost minimisation is replaced with a constraint, and branch-and-cut algorithm
 were stated for them both.

 Additional research has been performed into both MCP varients
 in the form of new heuristics based on variable neighbourhood tabu search
 (similar to the methods used for the \gls{pcstp} by \citet{canuto2001local} --
 See Section \ref{sec:canuto-search}) by \citet{perez2003variable}, and both
 greedy and evolutionary heuristics were proposed by \citet{renaud2004efficient}.
 
 \citet{current1994median} introduced a similar problem named the ``Median Tour Problem''
 which is defined on a directed graph which has two edge cost functions and vertex prizes
 and involves optimising a bicriterion objective function
 (facility cost and prize-weighted assignment cost)
 when defining a cycle facility which spans a minimum number of vertices.

 The Profitable Tour Problem turns out to be a special case of certain variants of Median
 Tour Problems. Consider a variant defined as follows:
 \todo{Consider whether we want \gls{ilp} formulations here}

 Let $G = (V, E, c, d)$ be a undirected graph with two edge cost functions
 $c : E \to \RR^+$ and $d : E \to \RR^+$, then find a circuit $C = (V_C, E_C) \subseteq G$
 which minimises
 the function
 $$c(C) = \sum_{ij \in E_C} c_{ij} + \sum_{i \in V} \min_{j \in C} d_{ij} \mathnormal{.}$$
 Given this definition and an instance of the PTP on a graph $G = (V, E, c, p)$
 we can define an assignment cost function as
 $$d_{ij} =
 \begin{cases}
   0 & i = j \\
   p_i & i \neq j
 \end{cases}\mathnormal{.}
 $$
 Thus the assignment cost of a vertex is 0 if it is part of the facility and its prize
 if it is not. Then the cost of a tour becomes the following,
 $$c(C) = \sum_{ij \in E_C} c_{ij} + \sum_{i \in V} \min_{j \in C} d_{ij} =
 \sum_{ij \in E_C} c_{ij} + \sum_{i \not\in C} \min_{j \in C} d_{ij} + \sum_{i \in V_C} \min_{j \in V_C} d_{ij}
 = \sum_{ij \in E_C} c_{ij} + \sum_{i \not\in V_C}  p_i$$
 which corresponds to the cost of a tour in the PTP.
 \subsection{Median Trees}
 We then turn our focus to the problem of finding median \textit{trees} in graphs.
  To the best of our knowledge, this problem has not recieved much attention.

  \citet{minieka1985optimal} and \citet{george2003bi} consider the problem of finding
  median subtrees of a tree graph.
  A linear time solution to this was developed by
  \citet{kim1991locating}, who also consider the problem in cactus graphs.

  Finding a median tree in a general graph is, to the best of our knowledge, not well researched.
 \citet{aneja1992location} establishes the
  problem as being NP-hard -- by reduction from the \gls{stp}, and shows that the problem can
  decomposed into finding median trees on distinct ``blocks'' of a graph.

  \citet{kim1991locating} considers a variant of the problem where the graph is embedded in the plane
  and assignment costs correspond to euclidean distances.
  
  However, we have found no \gls{ilp} formulations of the problem nor any attempts to solve it to optimality
  in the general case. Additionally, there seems to be no consensus of a ``most interesting'' variant
  of the problem of finding median trees in graphs.

  

%%% Local Variables:
%%% TeX-master: "report"
%%% reftex-default-bibliography: ("lit.bib")
%%% End:

