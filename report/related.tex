\chapter{Research in Related Problems}
\label{chap:related}


\section{The Steiner Tree Problem}
\todo[inline]{This section should be a section which highlights the research history of the STP
 which has been ``replicated'' in the PCSTP}
\section{The Prize-Collecting Travelling Salesman Problem and Variants}
First introduced by \citet*{balas1989prize}, the Prize-Collecting Travelling Salesman Problem
(PCTSP) is a problem closely related to the
PCSTP.

The problem is formulated by \citeauthor{balas1989prize} as follows. Let
$G = (V, A, c, p)$ be a directed
graph with arc costs $c: A \to \RR_+$ and vertex prizes (or penalties) $p: V \to \RR_+$
then find a cycle $C = (V_C, A_C)$ with $V_C \subseteq V$ and $A_C \subseteq A$ which
minimises the function
$$\sum_{ij \in V_C} c_{ij} + \sum_{v \in V \setminus V_C} p_v$$
such that
$$\sum_{v_i \in V_C} p_i \geq B$$
for some $B \in \RR_+$.

If we recollect the definition of the PCSTP and its variants (Section \ref{sec:pcstp}) then
we immediately see some similarities. The PCTSP involves optimising the
\textit{Goemans-Williamson Minimization Problem} for the circuit $C$, however with the extra
constraint of collecting at least $B$ prize on the tour. This extra constraint draws comparisons
to the \textit{Quota Problem}.

As the PCTSP is a natural member of the family of Vehicle Routing Problems, it is also sometimes
stated with a \textit{depot} vertex \citep{feillet2005traveling} -- that is a vertex which
must be part of the solution cycle, $C$. This corresponds to the \textit{Rooted}
 Prize-Collecting Steiner Tree Problem.
 \section{The Median  Problems}
 *general idea*
 
 *k-median*
 
 *median shortest path*
\subsection{The Median Ciruit/Tour Problem}
*short history*

*problem defintion*

*MIP formulation*
\subsection{The Median Subtree Problem}
*short history*

*problem definition*

*MIP Formulation*

*Maybe motivate a Median Steiner Problem*

%%% Local Variables:
%%% TeX-master: "report"
%%% reftex-default-bibliography: ("lit.bib")
%%% End:

