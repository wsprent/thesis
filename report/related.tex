\chapter{Research in Related Problems}
\label{chap:related}
*some motivation*
\todo[inline]{Do examples for this section. Show the problems.}

\section{The Steiner Tree Problem}
\todo[inline]{This section should be a section which highlights the research history of the STP
 which has been ``replicated'' in the PCSTP}
\section{The Prize-Collecting Travelling Salesman Problem and Variants}
\todo{Think about whether there should be a super-section about all the variants}
First introduced by \citet*{balas1989prize}, the Prize-Collecting Travelling Salesman Problem
(PCTSP) is a problem closely related to the
PCSTP.

The problem is formulated by \citeauthor{balas1989prize} as follows. Let
$G = (V, A, c, p)$ be a directed
graph with arc costs $c: A \to \RR_+$ and vertex prizes (or penalties) $p: V \to \RR_+$
then find a cycle $C = (V_C, A_C)$ with $V_C \subseteq V$ and $A_C \subseteq A$ which
minimises the function
$$\sum_{ij \in V_C} c_{ij} + \sum_{v \in V \setminus V_C} p_v$$
such that
$$\sum_{v_i \in V_C} p_i \geq B$$
for some $B \in \RR_+$.

If we recollect the definition of the PCSTP and its variants (Section \ref{sec:pcstp}) then
we immediately see some similarities. The PCTSP involves optimising the
\textit{Goemans-Williamson Minimization Problem} for the circuit $C$, however with the extra
constraint of collecting at least $B$ prize on the tour. This extra constraint draws comparisons
to the \textit{Quota Problem}.

As the PCTSP is a natural member of the family of Vehicle Routing Problems, it is also sometimes
stated with a \textit{depot} vertex \citep{feillet2005traveling} -- that is a vertex which
must be part of the solution cycle, $C$. This corresponds to the \textit{Rooted}
Prize-Collecting Steiner Tree Problem.

\paragraph{The Orienteering Problem}
The Orienteering Problem (OP) is very much the circuit version of the \textit{Budget Problem}.
Given a graph $G  = (V, E, c, p)$, the OP inolves finding a circuit, $C = (V_C, E_C)$,
which maximises
$$\sum_{v \in V_C}  p_v \quad \text{s.t.} \quad \sum_{ij \in E_C} c_{ij} \leq B$$
for some $B \in \RR_+$.
\todo{Add something short on the history of the OP}
\paragraph{The Profitable Tour Problem}
A maybe more familiar problem in the realm of prize collecting tour problems, is the
so-called Profitable Tour Problem (PTP). This variant -- which earlier shared name with
the PCTSP -- involves dropping the quota constraint from the PCTSP, making it a true
 ``cousin'' of the PCSTP.

 While similar in formulation to the PCSTP, according to a fairly recent survey by
 \citet{archetti2014chapter}, the PTP has received very little attention -- 
 especially when compared to the PCSTP.
 In fact, to the best of our knowledge, no exact solution to the PTP
 has been proposed outside the lower bounding procedure proposed by \citet{dell1995prize},
  which is a big step in defining an exact algorithm.
 
 However, there has been some research into approximation algorithms for the PTP.
 In fact, an interesting symptom of exactly how related these two families of
 prize collecting problems are
is how the GW Algorithm (see Section \ref{sec:solving:approx:gw})
can be used used to generate a 2-approximation algorithm for the PTP
on graphs satisfying the triangle inequality in the same way as
polynomial time MST algorithms can be used to generate 2-approximation algorithms
for the TSP \citep{goemans1995general}.



 \section{The Median Facility Problems}
 A type of problems which are closesly related to the ``prize-collecting'' problems such as the PCSTP, are the so-called
 ``median'' problems. These problems involve finding optimal location of a ``facility'' with regards to the sum
 of distances from all vertices to the facility.

 For example, first introduced in \citeyear{hakimi1964optimum} by \citet{hakimi1964optimum} is a problem known as
 the \textit{$p$-Median Problem}. Given a graph $G = (V, E, c)$, this involves selecting a subset $S \subseteq V$ with
 cardinality $p$ such that $S$ minimises the sum of minimum distances from vertices in $V$ to vertices in $S$.
 
 \citet{current1987median} adds structure to the selection of vertices by introducing the \textit{Median Shortest Path Problem (MSPP)}.
 Here, given a graph $G = (V,E,c,d,p)$ -- where $c$ and $d$ are edge weight functions --
 and source and sink vertices $s$ and $t$, one must find a facility
  -- now a simple path $P$ --  in $G$
 from $s$ to $t$ in s.t. the function
 $$\sum_{i,j \in E(P)} c_{ij} + \sum_{i \in V} p_i D_{i, P}$$
 where $p_i$ is the weight of vertex $v_i$ and
 $D_{i, P}$ is the shortest path from vertex $v_i$ to any vertex in $P$ with regards to
 weights $d$.
 We commonly denote the first sum as the cost of the facility/subgraph and the second sum as the assignment cost.

 Intuitively, the MSPP involves weighing a trade off between the cost of the facility
 in $c$ and the assignment costs. 
 This is analogous to the trade off between the cost of a subgraph and the cost of missed prize in the
 family of prize-collecting problems. In the median problems, the fixed prize of a vertex is
 replaced by the variable ``prize'' represented by the shortest path to the facility sometimes weighted
 by a vertex prize.
 \subsection{The Median Ciruit/Tour Problem}
 The problem of finding a circuit shaped facility in a graph has also been studied.
 \citet{labbe1999themedian} introduced the ``Median Cycle Problem'' which involves
 minimising the cost of a cycle and assignment cost in a mixed graph (undirected for
 the facility and directed for assignment) given a root vertex. Alongside this problem
 -- which they denote (MCP1) -- they also introduced a budget version (MCP2) where the
 assignment cost minimisation is replaced with a constraint, and branch-and-cut algorithm
 were stated for them both.

 Additional research has been performed into both MCP varients
 in the form of new heuristics based on variable neighbourhood tabu search
 (similar to the methods used for the PCSTP by \citet{canuto2001local} --
 See Section \ref{sec:canuto-search}) by \citet{perez2003variable}, and both
 greedy and evolutionary heuristics were proposed by \citet{renaud2004efficient}.
 
 \citet{current1994median} introduced a similar problem named the ``Median Tour Problem''
 which is defined on a directed graph which has two edge cost functions and vertex prizes
 and involves optimising a bicriterion objective function
 (facility cost and prize-weighted assignment cost)
 when defining a cycle facility which spans a minimum number of vertices.

 The Profitable Tour Problem turns out to be a special case of certain variants of Median
 Tour Problems. Consider a variant defined as follows:

 Let $G = (V, E, c, d)$ be a undirected graph with two edge cost functions
 $c : E \to \RR$ and $d : E \to \RR$, then find a circuit $C = (V_C, E_C) \subseteq G$
 which minimises
 the function
 $$c(C) = \sum_{ij \in E_C} c_{ij} + \sum_{i \in V} \min_{j \in C} d_{ij} \mathnormal{.}$$
 Given this definition and an instance of the PTP on a graph $G = (V, E, c, p)$
 we can define an assignment cost function as
 $$d_{ij} =
 \begin{cases}
   0 & i = j \\
   p_i & i \neq j
 \end{cases}\mathnormal{.}
 $$
 Thus the assignment cost of a vertex is 0 if it is part of the facility and its prize
 if it is not.
 \subsection{The Median Subtree Problem}
*short history*

*problem definition*

*MIP Formulation*

*Maybe motivate a Median Steiner Problem*

%%% Local Variables:
%%% TeX-master: "report"
%%% reftex-default-bibliography: ("lit.bib")
%%% End:

