\chapter{Solving the Prize-Collecting Steiner Tree Problem}
\label{chap:solving}

In this chapter, we will present a survey of methods used to solve the Prize-Collecting Steiner Tree Problem. Due to the nature of solving
NP-hard combinatorial optimisation problems, this involves the the mixing and matching of methods within larger frameworks.
 After a rundown of the history of solving the PCSTP in section \ref{sec:solving:history}, we will present the following:
\begin{itemize}
\item Section \ref{sec:solving:pre} presents preprocessing algorithms used to reduce instance sizes.
\item Section \ref{sec:solving:approx} presents approximations for solving the PCSTP.
\item Section \ref{sec:solving:heuristics} presents heuristics for solving the PCSTP.
\item Section \ref{sec:solving:lower} presents methods used for generating lower bounds.
\item Section \ref{sec:solving:exact} presents frameworks and combinations of the above used find exact solutions
   to the PCSTP.
\end{itemize}

\section{History}
\label{sec:solving:history}

Solutions to the PCSTP were first explored by \citet{Bienstock1993} who
proposed a 3-approximation algorithm based on Christophides algorithm for the TSP.
This was later improved to 2-approximation algorithm by
\citet{goemans1997primal} using a primal-dual method,
which was refined and implemented
for faster runtimes and typically better
bounds by \citet{Johnson:2000:PCS:338219.338637}.

\citet{canuto2001local} presented a primal heuristic based on multi-start
local search with pertubations and path relinking.

An ILP formulation based on the General Subtour Elimination Constraint (GSEC) formulation of the Travelling Salesman Problem was
stated by \citet{lucena2004strong}, who presented an algorithm for solving the LP relaxation of the GSEC formulation using
a separation procedure as a method for generating a strong lower bound for the PCSTP. Similarly, \citet{Ljubic:2004:memetic}
presented a memetic algorithm using tournament selection and mutation. They additionally presented an ILP formulation of the Steiner
 Aborescence Problem, based on cuts, the LP-relaxation of which was solved on a subgraph extracted from the memetic algorithm to
 improve the solutions as a postprocessing step.

 \citet{ljubic2005solving} proposed the first full framework for solving the PCSTP to optimality.
 This was done by solving
 equivalent SAP instances through branch-and-cut. Solving the LP-relaxation of a cut-formulation of the SAP with a
 a separation procedure was used to generate lower bounds.
  Furthermore, they presented a set of real-world problem instances for the PCSTP based on network layout in Berlin.

  \citet{lucena2004strong}, \citet{Ljubic:2004:memetic}, and \citet{ljubic2005solving} all applied preprocessing routines
  -- known as reduction tests -- to their input graphs to reduce the size of instances.
  These were originally gathered for
  the Steiner Tree Problem by \citet{duin1989edge,duin1989reduction},
  but no reduction tests had been stated and proven for the PCSTP
  until \citeauthor{uchoa2006reduction} did so in \citeyear{uchoa2006reduction}.
  \citeauthor{uchoa2006reduction} revisited
  reduction tests originally defined for the STP,
  but also introduced a new
  novel way of determining distances between nodes,
  allowing for stronger reduction tests.

  The 11th DIMACS challenge on Steiner tree problems\footnote{\url{http://dimacs11.zib.de/}}
  brought not only a new set of instances for the PCSTP, but
  also a new wave of contributions.

  Solving the PCSTP by transforming instances into equivalent problems in directed graphs
  was revisited by \citet{leitner2016dual} and \citet{gamrath2017scip}.
  \citet{leitner2016dual} transformed instances into the Asymmetric Prize-Collecting Steiner Tree Problem (APCSTP)
  and employed a
  branch-and-bound framework generating lower bounds with a dual ascent method which was originally presented by \citet{wong1984dual} for the SAP, and
  generating incumbents with
  a shortest-path primal heuristics.
  \citet{gamrath2017scip} presented SCIP Jack -- an exact solver for STP variants based on the solver (of the STP, by way of SAP) presented by
  \citet{koch1998solving} and the MIP solver, SCIP \footnote{\url{http://scip.zib.de/}}.
  Preprocessing routines were natively applied to PCSTP instances before transforming them into SAP instances.
  The SAP was formulated as an
  ILP in a so-called \textit{Flow-Balanced Directed Cut Formulation}. The general branch-and-bound
  procedures of the SCIP optimization suite were
  then employed together with a contraint separation procedure defined in \citet{koch1998solving}.

  Developments were also made in field of heuristics for the PCSTP. \citet{fu2014knowledge} presented a tabu search and a neighbourhood definition
  using a new \textit{vertex swap} operation.
  \citet{akhmedov2016divide} presented a clustering heuristics for dividing PCSTP into subproblems,
  and additionally a
  reimplementation of the solver by \citet{ljubic2005solving} which was used to solve problems defined by the subgraphs. Additionally, they presented
  applications for the PCSTP within biology.
  \citet{biazzo2012performance} explored the performance of the cavity method on the PCSTP.
  
 \section{Preprocessing}
 \label{sec:solving:pre}
Applying preprocessing routines to heavily reduce input graphs is a common technique which has been proven succesful in many cases both for instances of the \gls{stp}
\citep{koch1998solving}
and instances of the \gls{pcstp}
\citep{lucena2004strong,canuto2001local,ljubic2005solving, gamrath2017scip}. %TODO: MORE
These routines make use of invariants in the problem to make transformations of the
input graph to reduce its size without removing optimal solutions.

A set of common preprocessing routines are applied in different manners in literature.
The way in which this is done is mainly differentiated by:
\begin{enumerate}[label=\alph*)]
\item which routines to apply,
\item in which order, and
\item when to recursively apply routines and how many times.
\end{enumerate}
Preprocessing routines can be very effective. For example, the preprocesing routine
presented by \citet{koch1998solving} for the \gls{stp} removes
up to 98\% of edges in some instances. In other words, making good use of preprocessing
routines, have been shown to increase the size of solvable instances by upto 50 times in some
cases.

In this section, we will first give a full overview of
existing preprocessing methods for the \gls{pcstp},
 including proofs of validity,
 and then, in the end of the section,
 summarize their usage in recent literature.

 In the following, we will denote any edge or vertex (or part of a graph) as
 \textit{redundant} if there exists an optimal solution
  which does not contain it, and we say that a 
  transformation to a problem instance is \textit{valid} when produces an
  equivalent problem, that is a problem which has the same optimal
  value and in which solutions map back to the original problem.

  Thus, making a series of valid transformation to the input graph may actually
  hide optimal solutions, but will leave \textit{at least} one optimal solution
  which will map directly to an optimal solution in the untouched input graph.
 \subsection{Local Reduction Tests}\label{sec:pre:local}
 The first type of reductions we will look at, we denote as \textit{local} reduction tests.
 These are tests which at most require knowledge
 of the neighbourhood of a couple of vertexes in the graph
 and static global information
 such as maximum prize. As such, the tests
 presented in this section are generally of low computational cost,
 allowing for testing a full graph in linear time.

 Note that the first tests presented (NTD1, NTD2, TD1, TD2) are known collectively
 in recent literature as \textit{Degree Tests}, for example in \citet{rehfeldt2016reduction}.
\paragraph{Non-Terminals of Degree 1 (NTD1)}\label{sec:red:test:deg1}
Let $G = (V, E, c, p)$ be a PCST instance and let $v \in V$ be any non-terminal of degree 1, then
clearly
--- since edges have positive weights ---
$v$ can not be part of an optimal solution.
Thus, $v$ is redundant, and
 all vertices in the set
 \[\{v \mid v \in V \setminus N \wedge |\delta(v)| = 1\}\]
 are redundant. Hence it is valid to remove them from $G$ along with their adjacent edges.

\begin{figure}[h]\centering
    \begin{tikzpicture}[auto, node distance=1.3 cm]
      % Pre
      \begin{scope}[shift={(-0.5,0)}]
        \node[terminal] (a) at (0, 0) {a};
        \node[steiner] (b) [left=of a] {v};
        \node (sg) [above right = 1.3 and 0.1 of a] {};
        \node (sg2) [below right= 1.3 and 0.1 of a] {};

        %Edges
        \draw (a) edge node {$k$} (b);
        \draw[dashed] (a) to (sg);
        \draw[dashed] (a) to (sg2);
      \end{scope}

      \draw [->,decorate, thick,
      decoration={snake,amplitude=.4mm,segment length=2mm,post length=1mm}]
      (1.0,0) -- (3,0);
      % Post      
      \begin{scope}[shift={(4cm, 0)}]

        \node[terminal] (a) at (0, 0) {a};
        \node (sg) [above right = 1.3 and 0.1 of a] {};
        \node (sg2) [below right= 1.3 and 0.1 of a] {};

        %Edges
        \draw[dashed] (a) to (sg);
        \draw[dashed] (a) to (sg2);
      \end{scope}

  \end{tikzpicture}
  \caption{Removing a non-terminal of degree 1.}\label{fig:red:test:deg1}
\end{figure}

 While this reduction test is originally stated for the \gls{stp} \citep{hwang1992steiner}, it is also applicable to the \gls{pcstp}. Figure~(\ref{fig:red:test:deg1})
  shows an example of removing a degree 1 non-terminal.

  \paragraph{Non-Terminals of Degree 2 (NTD2)}
    \label{fig:red:test2}
    Similarly, let $G = (V,E,c,p)$ be an instance of the \gls{pcstp}, let $v \in G$ be a non-terminal of degree $|\delta(v)| = 2$, and let
    $u$ and $w$ be the two vertices adjacent to $v$. Then we can obtain a reduced, equivalent graph,
    \[G' = (V', E', c', p)\]
    where
    \[V' = V - v \mathnormal{,}\]
    \[E' = (E \setminus \{(u,v),(w,v)\}) \cup \{(u,w)\}\mathnormal{,}\]
    and $c' = c$ except for
    $$c'_{uw} =
    \begin{cases}
      \min(c_{uw}, c_{uv} + c_{vw}) & (u,w) \in E\mathnormal{,} \\
      c_{uv} + c_{vw} & \text{otherwise.}
    \end{cases}$$

    In other words, if $c_{uv} + c_{vw} <  c_{uw}$ then $(u,w)$ is redundant, can be removed,
    and the path from $u$ along $v$ to $w$
    can be contracted into a single edge.
    Otherwise $v$ and its edges are redundant and can be removed. Figure~(\ref{fig:pre:ntd2})
    shows an example this reduction test.

\begin{figure}[h!]\centering
    \begin{tikzpicture}[auto, node distance=1.3 cm]
      % Pre
      \begin{scope}
        \node[terminal] (a) at (0, -0.65) {u};
        \node[terminal] (b) [above=of a] {w};
        \node[steiner] (c) [above left= 0.65 and 1.3 of a] {v};
        \node (sg) [above right = 1.3 and 0.1 of b] {};
        \node (sg2) [below right= 1.3 and 0.1 of a] {};

        %Edges
        \draw (a) edge node {$k$} (b);
        \draw (a) edge node {$c_1$} (c);
        \draw (b) edge node[swap] {$c_2$} (c);
        \draw[dashed] (b) to (sg);
        \draw[dashed] (a) to (sg2);
      \end{scope}

      \draw [->,decorate, thick,
      decoration={snake,amplitude=.4mm,segment length=2mm,post length=1mm}]
      (1,0) -- (3.25,0);
      % Post      
      \begin{scope}[shift={(4.5cm, 0)}]
        \node[terminal] (a) at (0, -0.65) {u};
        \node[terminal] (b) [above=of a] {w};
        \node (sg) [above right = 1.3 and 0.1 of b] {};
        \node (sg2) [below right= 1.3 and 0.1 of a] {};

        %Edges
        \draw (a) edge node[swap] {$\min(k, c_1 + c_2)$} (b);
        \draw[dashed] (b) to (sg);
        \draw[dashed] (a) to (sg2);
      \end{scope}

  \end{tikzpicture}
  \caption{Removing a non-terminal of degree 2.}  \label{fig:pre:ntd2}
\end{figure}

This test is another example of a reduction test for the \gls{stp} which can by directly applied to
 the \gls{pcstp}.
 \paragraph{Terminals of Degree 1 (TD1)}\label{sec:pre:td1}
 When we have a terminal, $u$, of degree one, we can make the following two observations:
 \begin{enumerate}
 \item If the cost of the edge adjacent to $u$ is higher than $p_u$, then it is
   never worth it to connect $u$ to a solution, hence the adjacent edge is redundant, and
 \item if the above is the case and there exists
   a vertex with at least as high prize as $u$, the single vertex solution containing $u$
   can \textit{at best} share objective value with other optimal solutions. Thus $u$ is redundant.
 \end{enumerate}
 These observations were stated first by \citet{uchoa2006reduction} ---
 however without making the second observation,
 making the redundancy of $u$ potentially invalid
 --- and later corrected and restated by \citet{rehfeldt2016reduction} as a combined test.

 We state here only the first part here as the TD1 test to allow for the disconnection of vertices with maximal prize.
 \begin{theorem}[Terminals of Degree 1 Test]
   Let $u$ be a terminal in $G$ of degree of 1 with adjacent edge $(u, v)$. If
   $p_u < c_{uv}$ then $(u,v)$ is redundant.
 \end{theorem}

 Applying this definition of TD1 in conjection with the UDV test (defined in Section~\ref{sec:pre:udv})
 allows for the potential removal of terminals of degree one (see Figure~\ref{fig:red:test:deg1}), giving the
  test defined by \citet{uchoa2006reduction} and \citet{rehfeldt2016reduction}.

\begin{figure}[h!]\centering
    \begin{tikzpicture}[auto, node distance=1.3 cm]
      % Pre
      \begin{scope}[shift={(-4,0)}]
        \node[terminal, label={$p_a$}] (a) at (0, 0) {v};
        \node[terminal, label={$p_v$}] (b) [left=of a] {u};
        \node (sg) [above right = 1.3 and 0.1 of a] {};
        \node (sg2) [below right= 1.3 and 0.1 of a] {};

        %Edges
        \draw (a) edge node {$k$} (b);
        \draw[dashed] (a) to (sg);
        \draw[dashed] (a) to (sg2);
      \end{scope}

      \draw [->,decorate, thick,
      decoration={snake,amplitude=.4mm,segment length=2mm,post length=1mm}]
      (-3,1) -- node [above=1mm,midway,text width=3cm, sloped, align=center] {$k\geq p_v$}
      (-1,2.3);
      \draw [->,decorate, thick,
      decoration={snake,amplitude=.4mm,segment length=2mm,post length=1mm}]
      (-3,-1) -- node [above=1mm,midway,text width=3cm, sloped, align=center] {$k < p_v$}
      (-1,-2.3);

      % Post      

      \begin{scope}[shift={(1.5cm, 0)}]

        \begin{scope}[shift={(0, 3cm)}]
          
          \node[terminal, label=right:{$p_a$}] (a) at (0, 0) {v};
          \node[terminal, label={$p_v$}] (b) [left=of a] {u};
        
          \node (sg) [above right = 1.3 and 0.1 of a] {};
          \node (sg2) [below right= 1.3 and 0.1 of a] {};

          % Edges
          \draw[dashed] (a) to (sg);
          \draw[dashed] (a) to (sg2);
        \end{scope}


        % Post2      
        \begin{scope}[shift={(0, -3cm)}]
          \node[terminal, label=right:{$p_a + (p_v - k)$}] (a) at (0, 0) {v};
          \node[terminal, label={$p_v$}] (b) [left=of a] {u};
          
          \node (sg) [above right = 1.3 and 0.1 of a] {};
          \node (sg2) [below right= 1.3 and 0.1 of a] {};
          
          % Edges
          \draw[dashed] (a) to (sg);
          \draw[dashed] (a) to (sg2);
        \end{scope}
      \end{scope}

      \draw [->,decorate, thick,
      decoration={snake,amplitude=.4mm,segment length=2mm,post length=1mm}]
      (4,3) -- node [above=1mm,midway,text width=3cm, sloped, align=center] {$\exists p_u \geq p_v$}
      (5,3);

      \draw [->,decorate, thick,
      decoration={snake,amplitude=.4mm,segment length=2mm,post length=1mm}]
      (4,-3) -- node [above=1mm,midway,text width=3cm, sloped, align=center] {$\exists p_u \geq p_v$}
      (5,-3);

      \begin{scope}[shift={(6.5cm, 0)}]

        \begin{scope}[shift={(0, 3cm)}]
          
          \node[terminal, label=right:{$p_a$}] (a) at (0, 0) {v};
        
          \node (sg) [above right = 1.3 and 0.1 of a] {};
          \node (sg2) [below right= 1.3 and 0.1 of a] {};

          % Edges
          \draw[dashed] (a) to (sg);
          \draw[dashed] (a) to (sg2);
        \end{scope}


        % Post3      
        \begin{scope}[shift={(0, -3cm)}]
          \node[terminal, label=right:{$p_a + (p_v - k)$}] (a) at (0, 0) {v};
          
          \node (sg) [above right = 1.3 and 0.1 of a] {};
          \node (sg2) [below right= 1.3 and 0.1 of a] {};
          
          % Edges
          \draw[dashed] (a) to (sg);
          \draw[dashed] (a) to (sg2);
        \end{scope}
    \end{scope}

  \end{tikzpicture}
  
  \caption{Removing a terminal of degree 1 (TD1 + UDV).}
  \label{fig:red:test:deg1}
\end{figure}

\paragraph{Terminals of Degree 2 (TD2)} 
Similarly to TD1, we can disconnect terminals of degree 2 where both edges have costs higher than the prize
 of the terminal. Formally we state this as,
 \begin{theorem}[Terminals of Degree 2 Test]
   Let $u$ be a vertex of degree 2, such that vertices $v$ and $w$ are adjacent to $u$. Suppose that
   \[p_u \leq \min(c_{uv}, c_{uw})\]
   then any optimal solution must \textit{either} contain both edges or neither.
   Thus it is valid to replace the edges $(u,v)$ and $(u,w)$ with a single edge $(w,v)$ with cost
   \[c_{wv} = c_{uv} + c_{uw} - p_u\mathnormal{.}\]
 \end{theorem}

\begin{figure}[h!]\centering
    \begin{tikzpicture}[auto, node distance=1.3 cm]
      % Pre
      \begin{scope}
        \node[terminal] (a) at (0, -0.65) {v};
        \node[terminal] (b) [above=of a] {w};
        \node[terminal] (c) [above left= 0.65 and 1.3 of a] {u};
        \node (sg) [above right = 1.3 and 0.1 of b] {};
        \node (sg2) [below right= 1.3 and 0.1 of a] {};

        %Edges
        \draw (a) edge node {$k$} (b);
        \draw (a) edge node {$c_1$} (c);
        \draw (b) edge node[swap] {$c_2$} (c);
        \draw[dashed] (b) to (sg);
        \draw[dashed] (a) to (sg2);
      \end{scope}

      \draw [->,decorate, thick,
      decoration={snake,amplitude=.4mm,segment length=2mm,post length=1mm}]
      (1,0) -- node {$p_u \leq \min(c_1, c_2)$}
      (3.25,0);
      % Post      
      \begin{scope}[shift={(6.25cm, 0)}]
        \node[terminal] (a) at (0, -0.65) {v};
        \node[terminal] (b) [above=of a] {w};
        \node[terminal] (c) [above left= 0.65 and 1.3 of a] {u};
        \node (sg) [above right = 1.3 and 0.1 of b] {};
        \node (sg2) [below right= 1.3 and 0.1 of a] {};

        %Edges
        \draw (a) edge node[swap] {$\min(k, c_1 + c_2 - p_u)$} (b);
        \draw[dashed] (b) to (sg);
        \draw[dashed] (a) to (sg2);
      \end{scope}

      \draw [->,decorate, thick,
      decoration={snake,amplitude=.4mm,segment length=2mm,post length=1mm}]
      (9,0) -- node {$\exists p_v \geq p_u$}
      (10,0);

      \begin{scope}[shift={(11cm, 0)}]
        \node[terminal] (a) at (0, -0.65) {v};
        \node[terminal] (b) [above=of a] {w};
        \node (sg) [above right = 1.3 and 0.1 of b] {};
        \node (sg2) [below right= 1.3 and 0.1 of a] {};

        %Edges
        \draw (a) edge node[swap] {$\min(k, c_1 + c_2 - p_u)$} (b);
        \draw[dashed] (b) to (sg);
        \draw[dashed] (a) to (sg2);
      \end{scope}

  \end{tikzpicture}
  \caption{Removing a terminal of degree 2 (TD2 + UDV).}\label{fig:red:td2}
\end{figure}

\paragraph{Unconnected Dominated Vertex (UDV)}\label{sec:pre:udv}
Proposed by \citet{rehfeldt2016reduction},
the \textit{unconnected dominated vertex test} reduces the
graph by removing any subgraph which contains at most one terminal which has less than maximum prize.
Stated formally as in the following theorem, which we will not prove.
\begin{theorem}[Unconnected Dominated Vertex Test]
  Consider a connected subgraph $S = (V_S, E_S, c, p)$ of $G$. Let $N_S = \{v \in V_s \mid p_v > 0\}$
  be the set of terminals in $S$. Then $S$ is redundant and removing it is valid if either
  \begin{enumerate}
  \item $N = \emptyset$, or
  \item $N = \{u\}$ and $p_u \leq \max_{v \in V \setminus \{u\}} p_v$.
  \end{enumerate}
\end{theorem}

This test can easily be applied as a special case for subgraphs consisting of a single,
 unconnected vertex.
 Figure~(\ref{fig:red:test:deg1}) shows the application
 of TD1 followed directly by UDV.\@
\paragraph{Minimum Adjacency (MA)}
Again defined by \citet{duin1989reduction} for the \gls{stp}
but also applicable for the \gls{pcstp}, is
the \textit{Minimum Adjacency test}
which also known as the \textit{$V \setminus K$ test}.
It is a reduction test which contracts any adjacent
terminals which are connected by an edge with lower cost than either of their prizes.
 This is shown in Figure~(\ref{fig:red:test:ma}).
 
\begin{figure}[h!]\centering
    \begin{tikzpicture}[auto, node distance=1.3 cm]
      % Pre
      \begin{scope}[shift={(-0.5,0)}]
        \node[terminal, label={$p_u$}] (u) at (0, 0) {u};
        \node[terminal, label={$p_v$}] (v) [left=of u] {v};
        \node (sgu) [above right = 1.3 and 0.1 of u] {};
        \node (sgu2) [below right= 1.3 and 0.1 of u] {};

        \node (sgv) [above left = 1.3 and 0.1 of v] {};
        \node (sgv2) [below left= 1.3 and 0.1 of v] {};

        %Edges
        \draw (u) edge node {$k$} (v);
        \draw[dashed] (u) to (sgu);
        \draw[dashed] (u) to (sgu2);
        \draw[dashed] (v) to (sgv);
        \draw[dashed] (v) to (sgv2);
      \end{scope}

      \draw [->,decorate, thick,
      decoration={snake,amplitude=.4mm,segment length=2mm,post length=1mm}]
      (1.0,0) -- node [above=1mm,midway,text width=3cm, sloped, align=center] {$k \leq \min(p_v, p_u)$}
      (3,0);
      % Post      
      \begin{scope}[shift={(6cm, 0cm)}]
        \node[terminal, label={$p_u + p_v - k$}] (u) at (0, 0) {uv};
        \node (sgu) [above right = 1. and 2.0 of u] {};
        \node (sgu2) [below right= 1. and 2.0 of u] {};
        \node (sgv) [above left = 1. and 2.0 of u] {};
        \node (sgv2) [below left= 1. and 2.0 of u] {};

        %Edges
        \draw[dashed] (u) to (sgu);
        \draw[dashed] (u) to (sgu2);
        \draw[dashed] (u) to (sgv);
        \draw[dashed] (u) to (sgv2);
      \end{scope}

  \end{tikzpicture}
  \caption{Minimum adjacency test.}
  \label{fig:red:test:ma}
\end{figure}

\begin{theorem}[Minimum Adjacency]
  Let $u$ and $v$ be adjacent terminals in $G$. If we have
  \[c_{uv} \leq \min(p_u, p_v)\]
  and
  \[c_{uv} = \min_{(u, w) \in \delta(u)}c_{uw}\]
  then it is valid to contract $u$ and $v$.
\end{theorem}
\begin{proof}
  It can be shown that any solution $T$ for the \gls{pcstp} problem defined by a graph $G$
  which contains $u$ but not $(u,v)$ can
  be transformed into a solution $T'$ which contains $(u,v)$ where $c(T') \leq c(T)$.

  \paragraph{Case 1: ($v \in V_T$)}
  Let $(u, w) \in E_T$ be the first edge in the simple path from $u$ to $v$ in $T$
  ($T$ is a tree, so there is only
  one such path). By assumption, we
  have that $c_{uv} \leq c_{uw}$, and the tree $T'$ constructed by removing $(u,w)$ from $T$ and
  adding $(u,v)$ has cost
  \[c(T') = c(T) - c_{uw} + c_{uv} \leq T\mathnormal{.}\]
  \paragraph{Case 2: ($v  \not\in V_T$)}
  Let $T'$ be the tree obtained by adding $(u,v)$ to $T$. As per the assumption that $\min(p_u, p_v) \geq c_{uv}$,
  we have that $T'$ has cost,
  \[c(T') = c(T) + c_{uv} - p_{uv} \leq c(T)\mathnormal{.}\]
\end{proof}


\subsection{Steiner Distance Reduction Tests}\label{sec:sd-red-test}
% \[d(u,v) = \min \{ c(P) \mid P \in \mathcal{P}_{uv}\}\]
More complex tests can be described in terms
of the concepts Steiner distance and Bottleneck distance. These were originally stated for
the \gls{stp} by, amongst others, \citet{duin1989edge,duin1989reduction} and later
adapted for the \gls{pcstp} by \citet{uchoa2006reduction}.

Before describing the reduction tests, we must make some definitions.
If $P = (\ldots, u, \ldots, w, \ldots)$ is a simple path in $G$, then $P_{uw}$ is
the subpath of $P$ which starts at vertex $u$ and ends in vertex $w$,
that is $P_{u,w} = (u, \ldots,w)$. Then we define the \textit{Steiner distance} of
$P_{uw}$ as the cost of edges in $P_{uw}$ subtracted by the prize collected along
 the way,
 $$sd(P_{uw}) = \sum_{(i,j) \in E(P_{u,w})} c_{i,j} -
 \sum_{v \in V(P) \setminus \{u,w\}} p_{v}\mathnormal{.}$$
 We then denote the Steiner distance of a simple path, $P$, as the maximal Steiner
 distance found among subpaths of $P$,
 \[sd(P) = \max_{u,w \in P} sd(P_{uw})\mathnormal{.}\]
 Let $\mathcal{P}_{uw}$ be the set of all simple paths
 connecting vertices $u$ and $w$ in
 $G$, then we denote the \textit{bottleneck distance} between $u$ and $w$ as the minimal Steiner
  distance among paths in $\mathcal{P}_{uw}$,
  \[B(u,w) = \min_{P \in  \mathcal{P}_{uw}} sd(P)\mathnormal{.}\]
  The bottleneck distance is a measure of the worst case \textit{opportunity cost}
 of bridging some cut $(S, T)$
 where $u \in S$ and $v \in T$.

 Finally, we denote ${B(u,w)}^{-e}$
 as the minimum Steiner distance of a path between $u$ and $w$ which does
   not include $e$, that is
  \[{B(u,w)}^{-e} = \min_{P \in  \mathcal{P}_{uw}, e \not \in P} sd(P)\mathnormal{.}\]

 \citet{uchoa2006reduction} shows that calculating the bottleneck distance
 is an NP hard problem by reduction from the Hamiltonian Path problem, suggesting that
 exact bottleneck distances are infeasible to calculate. However, \citet{uchoa2006reduction}
 also claims that existing heuristics are fast and give strong upper bounds.

It is worth noting that due to
\begin{enumerate*}[label={\alph*)}]
\item the relatively new proposal of the Steiner/bottleneck distances by \citet{uchoa2006reduction} for the \gls{pcstp}, and
\item their computational complexities,
\end{enumerate*}
the reduction tests stated in this section have been applied in literature
using ordinary distance (i.e.\ ignoring prizes) and shortests paths
instead of the Steiner and bottleneck distances. This corresponds to the direct application of
reduction tests
proposed for the \gls{stp} by \citet{duin1989edge,duin1989reduction}. Since this results in
working with
upper bounds on the bottleneck distance, the original
tests are still valid for the \gls{pcstp},
but are weaker than the natively tests proposed by \citet{uchoa2006reduction}.

When we refer to
\textit{shortest path relaxations} of the following tests, we refer to application of the test
 as described above.

 The tests shown below, while computationally more costly than the local tests, can potentially
 ``break'' a graph, which previously could not be reduced further with basic tests, in such a
 way that the basic tests can be applied.
 \paragraph{Special Distance (SD)}\label{sec:red:test:sd}
 Using the \gls{pcstp} version of the Steiner and bottleneck distances,
 \citet{uchoa2006reduction} proves
  the validity of the more general \textit{special distance test}.
 \begin{theorem}[Special Distance Test]
 Consider any edge $(u,v) \in E$. If we have
 \[{B(u,v)}^{-(u,v)} \leq c_{uv}\]
 then $(u,v)$ is redundant.
\end{theorem}
 \begin{proof}
   Let $T  \subseteq G$ be an optimal solution to the \gls{pcstp} in graph $G$
   where we have
   \[B(u,v)^{-(u,v)} \leq c_{uv}\]
   for some edge $(u,v) \in E_T$, and let $(T_1, T_2)$ be the cut bridged
   by $(u,v)$ in $T$.

   Consider then the simple path $P \in \mathcal{P}_{uv}$ in $G$ from $u$ to $v$ which doesn't
    contain $(u,v)$ and has
   \[sd(P) = {B(u,v)}^{-(u,v)}\mathnormal{.}\]

\begin{figure}[h!]\centering
    \begin{tikzpicture}[auto, node distance=1.3 cm]
      % Pre
      \begin{scope}[]
        \path[fill=black!5,use Hobby shortcut,closed=true]
        (-2.5, -2) .. (0.3,-1) .. (.5,1) .. (.5,2)  .. (-2.5,3.5);
        \path[fill=black!5,use Hobby shortcut,closed=true]
        (2.5, -2) .. (1.3,-1) .. (1.5,1) .. (1.5,2)  .. (2.5,3.5);

        \node[terminal] (u) at(0, 0) {u};
        \node[subgraph, above left= 0.05cm and 1.4cm of u] {$T_1$};
        \node[steiner, densely dashed] (w1) [above=of u] {w};
        
        \node[terminal] (v) [right= of u] {v};
        \node[subgraph,above right= 0.05 and 1.4cm  of v] {$T_2$};
        \node[steiner, densely dashed] (w2) [above=of v] {z};
        \node (sga) [left= of u] {};
        \node (sga2) [below left= 1.3 and 0.1 of u] {};

        \node (sgb) [right=  of v] {};
        \node (sgb2) [below right= 1.3 and 0.1 of v] {};

        
        
        %Edges
        \draw[selected] (u) edge node {$k$} (v);
        \draw[dashed, selected] (u) to (sga);
        \draw[dashed, selected] (u) to (sga2);

        \draw[dashed, selected] (v) to (sgb);
        \draw[dashed, selected] (v) to (sgb2);

        \draw (u) edge[snake it, red, bend left] node[text=black] {$P_1$} (w1);
        \draw (w1) edge[snake it] node {$P_2$} (w2);
        \draw (w2) edge[snake it, red, bend left] node[text=black] {$P_3$} (v);
      \end{scope}

      % \draw [->,decorate, thick,
      % decoration={snake,amplitude=.4mm,segment length=2mm,post length=1mm}]
      % (4,0) -- node [above=1mm,midway,text width=3cm, sloped, align=center] {}
      % (6,0);
      % Post      

  \end{tikzpicture}
  \caption{The optimal solution, $T = T_1 \cup T_2$ connected by $(u,v)$.
    Since $c(u,v) \geq B(u,v)^{-(u,v)}$,
  the simple path $P_2$ has cost no larger than $(u,v)$ and can replace it in $T$.}\label{fig:red:test:sd:thm}
\end{figure}
   
   Since $T$ is a tree, and $P$ by definition doesn't contain the edge $(u,v)$
   then we must have that $P$ consists of a subpath contained in $T_1$, followed by
   a subpath not contained in $T$, followed by a subpath contained in $T_2$.
   In other words, we have,
   \[P = (u, P_1, w, P_2, z, P_3, v)\]
   as shown in Figure~\ref{fig:red:test:sd:thm}.

   Then by definition we have,
   \[sd(P_2) \leq sd(P) = B(u,v)^{-(u,v)} \leq c_{uv}\]
   and we can construct another solution $T'$ by replacing the edge
   $(u,v)$ with the
   vertices and edges of $P_2$ which then must have the cost
   \[c(T') = c(T) - c_{uv} + sd(P_2) \leq c(T)\mathnormal{.}\]
   Hence $T'$ also optimal in $G$ and $(u,v)$ is by definition redundant.
\end{proof}

\paragraph{Non-Terminals of Degree 3 (NTD3)}\label{sec:red:test:deg3}
Another bottleneck distance based test,
also proved valid for the \gls{pcstp} by \citet{uchoa2006reduction},
is the \textit{non-terminals of degree 3 test}.
\begin{theorem}[Non-Terminals of Degree 3 Test]\label{thm:ntd3}
  Let $u$ be a nonterminal of degree 3 in $G = (V, E, c, p)$,
  and let $v$, $w$, and $z$ be its adjacent
  vertices (see Figure~\ref{fig:red:test:ntd3:thm}). If we have
  $$\min\left(B(v,w) + B(v,z), B(w,v) + B(w,z),  B(z, v)+ B(z, w)\right) \leq
  c_{uv} + c_{uw} + c_{uz}$$
  then there exists an optimal solution to $G$ where $u$ has degree of
  \textit{at most} 2, that is $|\delta(u)| \leq 2$. Thus $u$ and its three edges, can be replaced by
  the edges $\{(v, w), (w,z), (z,v)\}$ with costs
  \[c_{vw} = c_{vu} + c_{uw},\quad c_{wz} = c_{wu} + c_{uz},\quad c_{zv} = c_{zu} + c_{uv}\mathnormal{.}\]
\end{theorem}
\begin{figure}[h!]\centering
    \begin{tikzpicture}[auto, node distance=1.3 cm]
      % Pre
      \begin{scope}[]
        \path[fill=black!5,use Hobby shortcut,closed=true]
        (-0.5, 1) .. (-4, 3) .. (-0.5,3);
        \path[fill=black!5,use Hobby shortcut,closed=true]
        (.5, 1) .. (4,3) .. (.5,3);
        \path[fill=black!5,use Hobby shortcut,closed=true]
        (0, -.5) .. (-2,-3) .. (2,-3);

        \node[steiner] (u) at(0, 0) {u};
        \node[terminal] (w) [above left=1.3 and 1.3 of u] {w};
        \node[terminal] (v) [below= of u] {v};
        \node[terminal] (z) [above right= 1.3 and 1.3 of u] {z};

        \node[subgraph, above left= 0.05cm and 1 cm of w] {$T_w$};
        \node[subgraph,above right= 0.05cm and 1 cm of z] {$T_z$};
        \node[subgraph, below right= 0.05cm and 1 cm of v] {$T_v$};
        
        \node (sgw) [above left= 0.7 of w] {};
        \node (sgv) [below= 0.7 of v] {};
        \node (sgz) [above right = 0.7 of z] {};
        
        %Edges
        \draw[selected] (u) edge node {$c_{uv}$} (v);
        \draw[selected] (u) edge node {$c_{uw}$} (w);
        \draw[selected] (u) edge node[swap] {$c_{uz}$} (z);
        \draw[dashed, selected] (v) to (sgv);
        \draw[dashed, selected] (w) to (sgw);
        \draw[dashed, selected] (z) to (sgz);

        \draw (v) edge[snake it, bend left] node[text=black] {$P_1$} (w);
        \draw (v) edge[snake it, bend right] node[swap,text=black] {$P_2$} (z);
      \end{scope}

      % \draw [->,decorate, thick,
      % decoration={snake,amplitude=.4mm,segment length=2mm,post length=1mm}]
      % (4,0) -- node [above=1mm,midway,text width=3cm, sloped, align=center] {}
      % (6,0);
      % Post      

  \end{tikzpicture}
  \caption{Non-Terminal of $|\delta(u)| = 3$ which connects the subtrees $T_v$, $T_w$, and $T_z$. The simple paths $P_1$ and $P_2$ provide an
     alternate way of reconnecting $T$ with at least as good cost.}\label{fig:red:test:ntd3:thm}
\end{figure}
\begin{proof}   
  W.l.o.g.\ consider the case where $B(v,w) + B(v,z) \leq c_{uv} + c_{uw} + c_{uz}$,
  and let $T$ be an optimal solution which contains $(u,v)$, $(u,w)$, and $(u,z)$.

  Let $P_1 = (v, ..., w)$ be the simple path with Steiner distance
  \[sd(P_1) = B(v,w)\]
  and similarly let $P_2 = (v, ..., z)$ be the simple path with Steiner distance
  \[sd(P_2) = B(v,z)\mathnormal{.}\]
  This gives us the situation in Figure~(\ref{fig:red:test:ntd3:thm}). Note that $u$ may be a part of either paths.

  Let $T_v$, $T_w$, and $T_z$ be the subtrees of $T$ obtained by
  removing the edges adjacent to $u$ from $T$, and construct a new solution with
   total cost no-larger than $T$ as,
   \[T' = T_v \cup T_w \cup T_z \cup P_1 \cup P_2\mathnormal{.}\]
   We must have $|\delta_{T'}(u)| \leq 2$. If we had $|\delta_{T'}| = 3$ then
   we would have
   \[P_1 = \left[v, (v,u), u, (u,w), w \right]\]
   and
   \[P_2 = \left[v, (v,u), u, (u,z), z \right]\]
   giving us
   \[B(v,w) + B(v, z) = sd(P_1) + sd(P_2) = 2 c_{vu} + c_{uw} + c_{uz} > c_{vu} + c_{uw} + c_{uz}\]
   which is a contradiction to our original assumption that
   \[B(v,w) + B(v,z) \leq c_{uv} + c_{uw} + c_{uz}\mathnormal{.}\]
   Thus $T'$ is an optimal solution to $G$ with $|\delta_{T'}(u)| \leq 2$.
\end{proof}

\subsection{Voronoi Reductions}

Reduction test based on a Voronoi decomposition of the input graph was introduced for the
\gls{pcstp} by \citet{gamrath2017scip}. These reduction tests
involve using Voronoi decomposition of the graph to generate conditional lower bounds based on
vertex inclusion/exclusion. Given a strong upper bound, these can then be used
to reduce the graph.

\paragraph{Voronoi Decomposition of a PCSTP Graph}

Given an instance of the \gls{pcstp} with the undirected graph $G = (V, E, c, p)$ and terminal set
$N \subseteq V$, we define a Voronoi region for a given terminal, $i \in N$, as the set of nonterminals
which are closer to $i$ than any other terminal in terms of simple shortests paths, that is,
\[R_i = \left\{v \in V \setminus N \mid i = \argmin_{u \in N} d_{vu}\right\}\]
where $R_i$ is the Voronoi region covered by the terminal $i$ and $d_{vu}$ is the shortest path between vertices
$v$ and $u$.

We then define the \textit{radius} of the $i$th Voronoi region as the shortest path
from its centre (the $i$th terminal)
to any vertex not in the region,
\[\radius(i) = \min_{v \not\in R_i} d_{iv}\mathnormal{.}\]
From this we define a \textit{prize-collecting radius} of a region as,
\[\pcradius(i) = \min(\radius(i), p_i)\mathnormal{.}\]
We can intuitively see the \textit{pc radius} of a region as a kind of lower bound
on the cost contribution
 of that region to any
 \textit{connected}, multi-vertex solution. Either the prize for the respective terminal must be paid as a penalty
 \textit{or} the terminal must be connected to the rest of the tree by a path which must have some cost \textit{higher}
 than the \textit{radius} of the region.

 Finally, let
 \[z_1, z_2, \ldots, z_{|N|}\]
 be a monotonically increasing ordering of terminals such that
\[\pcradius(z_1) \leq \pcradius(z_2) \leq ... \leq \pcradius(z_{|N|})\mathnormal{.}\]

\paragraph{Lower Bound Reductions}

The first reduction we can present based on Voronoi regions is based on a lower bound given that some nonterminal is
part of a solution. This was first introduced for the \gls{stp} by \citet{polzin2001improved} and later adapted by
\citet{rehfeldt2016reduction} for the \gls{pcstp}.

\begin{theorem}\label{thm:vor:1}
  Let $T = (V_T, E_T, c, p)$ be a solution to the \gls{pcstp} in the graph $G = (V, E, c, p)$ with terminal set $N$ such that
  $v_k \in V_T$ for some $v_k \in V \setminus N$. Finally, let $v_a \in N$ and $v_b \in N$ be the two terminals \textit{closest}
  to $v_k$, then we have
  \[d(v_a, v_k) + d(v_b, v_k) + \sum_{i = 1}^{|N| - 2} \pcradius(z_i) \leq c(T)\mathnormal{.}\]
\end{theorem}
\begin{proof}
  Consider a solution $T$ as described above which contains some nonterminal $u \in V_T$.
  For all $i \in (N \cap V_T)$ let $P_i$ be the
  path from $v_i$ to $u$ in $T$ (this must be well defined as $T$ is a tree). Furthermore,
  let $P_i'$ be the subpath of $P_i$ which starts in $v_i$ and ends in the first vertex along
  $P_i$ not in $R_i$, and
  let $\Delta(P_i)$ be the number of times a Voronoi border is crossed by the path $P_i$ where
  we have $\Delta(P_i) = \infty$ if $v_i \not\in V_T$.

  \begin{figure}[h!]
     \centering
     \begin{tikzpicture}[auto, node distance=1.3cm]
       \node[steiner] (vi) {$v_i$};
       \node[steiner] (vib) [below= of vi] {$\phantom{v_i}$};

       \node[terminal] (vj) [right= of vi] {$v_m$};
       \node[terminal] (vl) [left= of vib] {$v_l$};
       \node [above left= 0.2 of vib] {$P_l$};

       
       \node[steiner]  (vlal) [above left= 1.2 and 1 of vl] {$\phantom{v_i}$};
       \node [above right= 0.05 and 0.1 of vlal] {$P_{i_3}$};
       \node[terminal]  (vlala) [above= of vlal] {$\phantom{v_i}$};

    \node[steiner] (vll) [below left=  of vl] {$\phantom{v_i}$};
    \node[terminal] (vlll) [below left=  of vll] {$\phantom{v_i}$};

    \node[terminal] (vibr) [below right= 0.5 and 1.3 of vib] {$\phantom{v_i}$};

    \begin{scope}
      % VORONOI :(
            \draw[dashed] let \p1=(vll),\p2=(vlll) in 
            ({(\x1+\x2)/2-1cm},{(\y1+\y2)/2+1cm}) coordinate(A)
            --
            ({(\x1+\x2)/2+1cm},{(\y1+\y2)/2-1cm}) coordinate(AA);
            \draw[dashed] let \p1=(vl),\p2=(vlal) in 
            ({(\x1+\x2)/2+1cm},{(\y1+\y2)/2+1cm}) coordinate(B)
            --
            (A);

            \draw[dashed] let \p1=(vib),\p2=(vibr) in 
            ({(\x1+\x2)/2+1cm},{(\y1+\y2)/2+1cm}) coordinate(C)
            --
            (AA);

            \draw[dashed] (B) -- (C);
            \draw[dashed] (C) -- ([xshift=2cm]C);
            \draw[dashed] (B) -- ([yshift=2cm]B);
            \draw[dashed] (AA) -- ([yshift=-2cm]AA);
            \draw[dashed] (A) -- ([xshift=-2cm]A);
   \end{scope}
   \begin{scope}
     % Edges
     \draw[selected, blue] (vlala) edge (vlal);
     \draw[selected, blue] (vlal) edge (vl);

     \draw[selected, green] (vlll) edge node[black] {$P_{i_2}$} (vll);
     \draw[selected, yellow] (vibr) edge node[black] {$P_{i_1}$} (vib);

     \draw[selected, black] (vll) edge (vl);
     
     \draw[selected] (vj) edge node[above] {$P_m$} (vi);

     \draw[selected, orange] (vl) edge (vib);
     \draw[selected, orange] (vib) edge (vi);
   \end{scope}
  \end{tikzpicture}
  
  \caption{Example solution of the \gls{pcstp} split into paths.}\label{fig:pre:vor:pro}
\end{figure}

  Let $v_m$ and $v_l$ be two terminals such that $\Delta(P_m) + \Delta(P_l)$ is minimal among
  all pairs of terminals where $P_m$ and $P_l$ are disjoint paths. Since $T$ is a tree and
  $v_i$ must have degree of at least $2$ (See Section~\ref{sec:red:test:deg1}) there must
  be at least one such pair.

  Consider the set of paths
  $$\mathcal{P} = \{P_i' \mid i \in (V_T \cap N) \setminus \{v_m, v_l\} \}
  \mathnormal{.}$$
  These paths must be disjoint as $T$ contains no cycles. The paths in $\mathcal{P}$
  alongside $P_m$ and $P_l$ are then all pairwise disjoint paths which cover a subset of the
  edges
  in $E_T$.

An example of this partition of $T$ is shown in Figure~(\ref{fig:pre:vor:pro}). The two ``closest''
terminals $v_l$ and $v_m$ have disjoint paths $P_l$ and $P_m$ to the interesting nonterminal $v_i$.
The rest of the solution is partitioned by $P_i'$ paths, these paths run from a terminal towards $v_i$
until ---inclusive--- the first edge which crosses a Voronoi boundary. No two paths in such a decomposition
can share an edge. This follows directly from the fact that each Voronoi region contains
\textit{only} a single terminal, and that there can be only a single path from vertex to $v_i$ (no cycles).
Hence a border edge can only be crossed in one ``direction'' this way.

With this, clearly, the total edge costs of these paths cannot be greater than the edge cost of $T$, as
we have only selected a subset of the edges of $T$, and never select an edge twice. In other words, we have
  \begin{equation}
    c(E(P_m)) + c(E(P_l)) + \sum_{P_i \in \mathcal{P}} c(E(P)) \leq c(E_T)\mathnormal{.}\label{eq:vor:one}
  \end{equation}

  
  Additionally, by definition
  we must have that $P_i' \leq \pcradius(i)$, $d(v_m, v_i) \leq c(E(P_m))$, and
  $d(v_l, v_i) \leq c(E(P_l))$. This gives us

  \begin{equation}
d(v_l, v_i) + d(v_m, v_i) + \sum_{P_i \in \mathcal{P}} \pcradius(i) \leq c(E(P_m)) + c(E(P_l)) + \sum_{P_i \in \mathcal{P}} c(E(P))\label{eq:vor:two}
\end{equation}

  Finally, for the set of terminals not covered by $T$, we must ---again by the definition of $\pcradius$--- have
  \begin{equation}
  \sum_{i \in N \setminus V_T} \pcradius(i) \leq \sum_{i \in N \setminus V_T} p_i \mathnormal{.}\label{eq:vor:three}
  \end{equation}

  By combining the equations (\ref{eq:vor:one}), (\ref{eq:vor:two}), and (\ref{eq:vor:three}), and
   by the definition of vertices $v_a$ and $v_b$ being the terminals closest to $v_i$ we have,
   \[d(v_a, v_k) + d(v_b, v_k) + \sum_{i = 1}^{|N| - 2} \pcradius(z_i) \leq c(T)\]
   and we are done.
 \end{proof}
 If a nonterminal is found such that the bound in Theorem \ref{thm:vor:1} is larger than the objective value of
 any found feasible solution, then that nonterminal is redundant and it is valid to remove it and its edges.

 We can produce a similar lower bound if the ``interesting'' vertex is instead a terminal.
\begin{theorem}\label{thm:vor:2}
  Let $T = (V_T, E_T, c, p)$ be a solution to the \gls{pcstp} in the graph $G = (V, E, c, p)$ with terminal set $N$ such that
  $v_k \in V_T$ for some $v_k \in N$ and $V_T \neq \{v_k\}$. Finally, let $v_b \in N$ be the terminal \textit{closest}
  to $v_k$, and let $z'_1, \ldots, z_{|N|-1}$ be an ordering of the set $N \setminus \{v_k\}$ such that
  \[\pcradius(z_1) \leq \cdots \leq \pcradius(i) \leq \cdots \leq \pcradius(z'_{|N|-1})\]
  then we have
  \[d(v_b, v_k) + \sum_{i = 1}^{|N| - 2} \pcradius(z'_i) \leq c(T)\mathnormal{.}\]
\end{theorem}
The proof of Theorem~\ref{thm:vor:2} is symmetric to the proof of Theorem~\ref{thm:vor:1}. If a terminal is found such that
the respective lower bound is higher than the objective value of a given incumbent, then the terminal is redundant and it is
 valid to remove it from $G$ along with its adjacent edges.

 \subsection{Summary of Usage}\label{sec:pre:summary-usage}
There's a general consensus in literature that preprocessing is an important part of producing solutions
for the \gls{pcstp}. Preprocessing described in this section is applied natively to the \gls{pcstp}
by all of
\citet{lucena2004strong, Ljubic:2004:memetic, ljubic2005solving,akhmedov2016divide,gamrath2017scip}.
Additionally, preprocessing is applied to the \gls{sap} by \citet{leitner2016dual}, and proposed
as further improvements by \citet{fu2014knowledge}.

\paragraph{Combinations and Effectiveness}

\citet{lucena2004strong} applies all four of the degree tests (NTD1, NTD2, TD1+UDV, TD2+UDV)
alongside shortest path relaxations of SD and NTDk (for ``small'' numbers of $k$)
tests. They do not
present in which order tests are applied or how often they are repeated, but they
do present the graph sizes of their benchmark cases before and after applying preprocessing.

\citet{Ljubic:2004:memetic} and \citet{ljubic2005solving} share a preprocessing routine
 which applies in order:
\begin{enumerate}
\item degree one tests: NTD1+TD1+UDV,
\item degree two tests: NTD2+TD2.
\item shortest path relaxation of the special distance test (here called the least cost test),
\item shortest path relaxation of NTDk for $k = 3,\ldots,8$, and
\item the minimum adjacency test.
\end{enumerate}
until a fixpoint is reached. They present both numbers on graph reduction and preprocessing
time. Across all problem instances, they report on average
a 45\% reduction in graph size.

\citet{akhmedov2016divide} applies the four degree tests until a fixpoint is reached. They
report reduction in graph sizes for their test cases as well as pre-processing time and
total computation time with/without preprocessing.

\citet{rehfeldt2016reduction} present an depth study of preprocessing routines for the
\gls{pcstp}. They report an on average reduction of vertex and edge counts by $91.5\%$ and
$88.8\%$ respectively for their ``strongest'' reduction package across DIMACs instances.
\begin{example}
Finally, we'll give an example of the power of
these preprocessing methods. Figure~\ref{fig:pre:ex} shows how just
the local tests defined in Section~\ref{sec:pre:local} are enough
to solve the ---admittedly simplistic--- instance of the \gls{pcstp}
from Figure~\ref{fig:pcstp:01}.


\begin{figure}[h!]\centering
  \begin{tikzpicture}[auto,
    node/.style={minimum width=10cm, inner sep={(10cm, 5cm)}}]
    \node[minimum width=5cm,
      minimum height=2.5cm] (1) {
      \begin{tikzpicture}[node distance=1.5 cm,
            every node/.style={minimum width=0, minimum height=0}]]
    % Nodes
    \node[terminal, label={12}] (a) at (0,0) {a};
    \node[steiner] (b) [right=of a] {b};
    \node[steiner] (c) [right=of b] {c};
    \node[terminal, label={10}] (d) [above =of c] {d};
    \node[steiner] (e) [left=of d] {e};
    \node[steiner] (f) [left=of e] {f};
    \node[terminal, label={3}] (g) [above right=0.75 and 1.3 of c] {g};
    % Edges

    \draw (a) edge node[below]{4} (b);
    \draw (b) edge node[near start]{5} (d);
    \draw (b) edge node[below]{8} (c);
    \draw (c) edge node{3} (d);
    \draw (c) edge node{2} (g);
    \draw (c) edge node[near start]{5} (e);
    \draw (d) edge node {6} (e);
    \draw (d) edge node{10} (g);
    \draw (e) edge node{1} (f);
    \end{tikzpicture}
  };
  
  \node[minimum width=5cm,
      minimum height=2.5cm] (2) at (9, 0) {
    \begin{tikzpicture}[node distance=1.5 cm,
          every node/.style={minimum width=0, minimum height=0}]]
    % Nodes
    \node[terminal, label={12}] (a) at (0,0){a};
    \node[steiner] (b) [right=of a] {b};
    \node[steiner] (c) [right=of b] {c};
    \node[terminal, label={10}] (d) [above =of c] {d};
    \node[steiner] (e) [left=of d] {e};
    \node[terminal, label={3}] (g) [above right=0.75 and 1.3 of c] {g};
    % Edges

    \draw (a) edge node[below]{4} (b);
    \draw (b) edge node[near start]{5} (d);
    \draw (b) edge node[below]{8} (c);
    \draw (c) edge node{3} (d);
    \draw (c) edge node{2} (g);
    \draw (c) edge node[near start]{5} (e);
    \draw (d) edge node{6} (e);
    \draw (d) edge node{10} (g);
  \end{tikzpicture}
};

\node[minimum width=5cm,
      minimum height=2.5cm](3) at (0, -4) {
  \begin{tikzpicture}[node distance=1.5 cm,
        every node/.style={minimum width=0, minimum height=0}]]
    % Nodes
    \node[terminal, label={12}] (a) at (0,0){a};

    \node[terminal] (b) [right=of a, label={8}] {b};
    \node[steiner] (c) [right=of b] {c};
    \node[terminal, label={10}] (d) [above =of c] {d};
    \node[steiner] (e) [left=of d] {e};
    \node[terminal, label={3}] (g) [above right=0.75 and 1.3 of c] {g};
    % Edges

    \draw (b) edge node[near start]{5} (d);
    \draw (b) edge node[below]{8} (c);
    \draw (c) edge node{3} (d);
    \draw (c) edge node{2} (g);
    \draw (c) edge node[near start]{5} (e);
    \draw (d) edge node{6} (e);
    \draw (d) edge node{10} (g);
  \end{tikzpicture}};
\node[minimum width=5cm,
      minimum height=2.5cm] (4) at (9, -4) {
  \begin{tikzpicture}[node distance=1.5 cm,
        every node/.style={minimum width=0, minimum height=0}]]
    % Nodes
    \node[terminal, label={12}] (a) at (0,0){a};

    \node[steiner] (c) [right=of a] {c};
    \node[terminal, label={13}] (d) [above =of c] {d};
    \node[steiner] (e) [left=of d] {e};
    \node[terminal, label={3}] (g) [above right=0.75 and 1.3 of c] {g};
    % Edges

    \draw (c) edge node{3} (d);
    \draw (c) edge node{2} (g);
    \draw (c) edge node[near start]{5} (e);
    \draw (d) edge node{6} (e);
    \draw (d) edge node{10} (g);
  \end{tikzpicture}};
\node[minimum width=5cm,
      minimum height=2.5cm] (5) at (0, -8) {
  \begin{tikzpicture}[node distance=1.5 cm,
        every node/.style={minimum width=0, minimum height=0}]]
    % Nodes
    \node[terminal, draw=none, fill=none] (a) at (0,0){\phantom{a}};
    \node[steiner] (c) [right =of a] {c};
    \node[terminal, label={13}] (d) [above =of c] {d};
    \node[steiner] (e) [left=of d] {e};
    \node[terminal, label={3}] (g) [above right=0.75 and 1.3 of c] {g};
    % Edges

    \draw (c) edge node{3} (d);
    \draw (c) edge node{2} (g);
    \draw (c) edge node[near start]{5} (e);
    \draw (d) edge node{6} (e);
    \draw (d) edge node{10} (g);
  \end{tikzpicture}};
\node[minimum width=5cm,
      minimum height=2.5cm] (6) at (9, -8) {
        \begin{tikzpicture}[node distance=1.5 cm,
          every node/.style={minimum width=0, minimum height=0}]]
    % Nodes

    \node[steiner] (c) at (0,0) {c};
    \node[terminal, label={13}] (d) [above =of c] {d};
    \node[terminal, label={3}] (g) [above right=0.75 and 1.3 of c] {g};
    % Edges

    \draw (c) edge node{3} (d);
    \draw (c) edge node{2} (g);
    \draw (d) edge node{10} (g);
  \end{tikzpicture}
};
\node[minimum width=5cm,
minimum height=2.5cm] (7) at (0, -12) {
  \begin{tikzpicture}[node distance=1.5 cm,
    every node/.style={minimum width=0, minimum height=0}]
    % Nodes
    \node[terminal, label={13}] (d) {d};
    \node[terminal, label={3}] (g) [below right=0.75 and 1.3 of d] {g};
    % Edges

    \draw (d) edge node{5} (g);
  \end{tikzpicture}};
\node[minimum width=5cm,
      minimum height=2.5cm] (8) at (9, -12) {
  \begin{tikzpicture}[node distance=1.5 cm,
        every node/.style={minimum width=0, minimum height=0}]
    % Nodes
    \node[terminal, label={13}] (d) {d};
  \end{tikzpicture}};

% \draw [->,decorate, thick,
% decoration={snake,amplitude=.4mm,segment length=2mm,post length=1mm}]
% (1)
% edge node [above=1mm,midway,text width=3cm, sloped, align=center] {hej}
% (2);

\draw (1) edge[snake it] node[snake node] {1: $NTD_1$} (2);
\draw (2) edge[snake it] node[snake node] {2: $TD_1$} (3);
\draw (3) edge[snake it] node[snake node] {3: $MA$} (4);
\draw (4) edge[snake it] node[snake node] {4: $UV$} (5);
\draw (5) edge[snake it] node[snake node] {5: $NTD_2$} (6);
\draw (6) edge[snake it] node[snake node] {6: $NTD_2$} (7);
\draw (7) edge[snake it] node[snake node] {7: $TD_2 + UV$} (8);
\end{tikzpicture}
\caption{Graph reduction on the \gls{pcstp} instance in Figure~\ref{fig:pcstp:01}. In the final graph,
 the node $d$ represents nodes $\{a,b,d\}$.}
\label{fig:pre:ex}
\end{figure}

This example shows how successive application of preprocessing routines can be at
reducing the size of an input graph. These routines have a cascading effect as one
graph reduction makes another one legal. In Figure~(\ref{fig:pre:ex}) this comes to
pass when the third reduction -- the \textit{Minimum Adjacency} test -- contracts
the edge between nodes $b$ and $d$, causing the degree of nodes $c$ and $e$ to
 drop to 2, which allows for the fifth and sixth reductions.
\end{example}

It is an open question which selection of graph reduction tests are
 most efficient, and likewise with the ordering of graph reductions.
%%% Local Variables:
%%% TeX-master: "report"
%%% reftex-default-bibliography: ("lit.bib")
%%% End:


\clearpage
\clearpage
\section{Approximation Algorithms}\label{sec:solving:approx}

\subsection{The Goemans-Williamson Algorithm}\label{sec:solving:approx:gw}
\citet*{goemans1995general} presented a primal-dual 2-approximation algorithm for the rooted variant of the
PCSTP, based on an algorithm for solving the general \textit{constrained forest problem}.

\todo[inline]{Quick summary of usage. It is popular.}

\paragraph{Definitions}
\citeauthor{goemans1995general} stated the ILP (GW-ILP) in Formulation \ref{form:approx:gw} for the rooted PCSTP.

 \begin{formulation}[h!]
   \begin{subequations}
     \begin{alignat}{3} 
       &\underset{x, z}{\text{minimize}}
       & & \sum_{e \in E} c_e x_e + \sum_{X \subset V; r \not\in X} z_X \left( \sum_{v \in X} p_v \right) & \\
       & \text{subject to}\quad
       & & x(\delta(S)) + \sum_{X \supseteq S}z_X \geq 1 \qquad&& \forall S \subset V; r \not\in S \label{form:approx:gw:cut}\\
       &&& \sum_{S \subset V; r \not\in S} z_X \leq 1 && \label{form:approx:gw:cz}\\
       &&& x_e \in \BB  && \forall e \in E \\
       &&& z_X \in \BB  && \forall S \subset V; r \not\in S
     \end{alignat}\label{form:approx:gw}
   \end{subequations}
   \caption{(GW-IP) formulation of the PCSTP.}
 \end{formulation}

 GW-IP has two decision vectors. The decision vector $x$ denotes which edges are contained in the solution,
 and the decision vector $z$, we have that $z_X$ is
 $1$ when iff. all $v \in X$ are not part of a feasible solution. Clearly, any optimal solution will have $z_X = 1$ for just a single
 $X \subset V$. However, it is still enforced by constraint (\ref{form:approx:gw:cz}).

 Constraint (\ref{form:approx:gw:cut})
 ensures that the solution is connected subgraph by requiring that for each subset $S$, the cut defined by $S$ must either be
 bridged by an edge in a feasible solution or $S$ must be counted as not included.

 \begin{formulation}[h!]
   \begin{subequations}
     \begin{alignat}{3} 
       &\underset{x, z}{\text{maximize}}
       & & \sum_{S \subset V; r \not\in S} y_S & \\
       & \text{subject to}\quad
       & & \sum_{S: e \in \delta(S)} y_S \leq c_e \qquad&& \forall e \in E \label{form:gw:pe}\\
       &&& \sum_{S \subseteq X} y_S \leq \sum_{v \in X} p_v  && \forall X \subset V; r \not\in X \label{form:gw:pv} \\
       &&& y_S \geq 0  && \forall S \subset V; r \not\in S
     \end{alignat}\label{form:approx:gw:dual}
   \end{subequations}
   \caption{(GW-D): Dual of the LP relaxation of (GW-ILP) from Formulation \ref{form:approx:gw}.}
 \end{formulation}
 
 The dual to the LP relaxation of GW-IP is the linear program in \ref{form:approx:gw:dual}. We make note of the
 \textit{packing} constraints (\ref{form:gw:pe}) and (\ref{form:gw:pv}), and note that by \textit{complementary slackness}
 of the primal LP, we have for optimal primal solutions to the LP, $(x^*, z^*)$,
 $$x^*_e > 0 \Rightarrow \sum_{S: e \in \delta(S)} y_S = c_e$$
 and
 $$z^*_X > 0 \Rightarrow \sum_{S \subseteq X} y_S = \sum_{v \in X} p_v\mathnormal{.}$$
 

 
 \paragraph{The Algorithm} The GW Algorithm consists of a growing phase and a pruning phase. The growing phase
 builds a feasible solution $T$ by maintaining a set of active and inactive \textit{components} which
 initially are singleton sets which are active if they do not contain the root vertex. The dual objective value is then
  increased by simultaneously increasing
  the dual variable values for all active components until one of the following happen:
 \begin{enumerate}[label=\alph*)]
 \item A packing constraint (\ref{form:gw:pe}) becomes binding for some edge $e = (i,j) \in E$. Then $e$ is added
   to $T$, and the components connected by $e$ are merged into a new component with combined cost.
   The new component is considered active iff the root vertex is not contained in it.
 \item A packing constraint (\ref{form:gw:pv}) becomes binding for some subset $X  \subset V$ (i.e. component). Then $X$ is deactivated
    and all vertices it contains are marked with the label $X$.
 \end{enumerate}
 This continues until there are no active components left; at which point $T$ is a feasible solution to the IP (GW-IP).

 
 \begin{algorithm}[h!]
   \begin{algorithmic}[1]
     \Input{Graph $G = (V, E, c, p)$ and root vertex $r \in V$.}
     \Output{Tree $T' \subseteq E$, and the set of vertices not spanned by $T'$, $X$.}
     \Procedure{GW Algorithm}{}
     \State $T \gets \emptyset$ \label{gw:init-start}
     \State $\mathcal{C} \gets \{\{v\} : v \in V\}$
     \For {$v \in V$}
       \State $d(v) \gets 0$
       \State $w(\{v\}) \gets 0$
       \State Unmark $v$.
       \If {$v = r$}
         \State $\lambda(\{r\}) \gets 0$
       \Else
         \State $\lambda(\{v\}) \gets 1$
       \EndIf
     \EndFor \label{gw:init-end}
     \While{$\{ C \in \mathcal{C} \mid \lambda(C) = 1\} \neq \emptyset$}
     \State $\epsilon_1 \gets \min_{i,j} \frac{c_{ij} - d(i) - d(j)}{\lambda (C_i) + \lambda(C_j)},
     \qquad i \in C_i, j \in C_j, C_i \neq C_j$ \label{gw:select-start}
     \State $\epsilon_2 \gets \min_{w} \sum_{v \in C_w} p_v - w(C_w), \qquad C_w \in \mathcal{C}$
     \State $\epsilon \gets \min(\epsilon_1, \epsilon_2)$ 
     \State $w(C) \gets w(C) + \lambda(C) \epsilon$ for all $C \in \mathcal{C}$ \label{gw:add-component}
     \State $d(v) \gets d(v) + \lambda(C) \epsilon$ for all $v \in C \in \mathcal{C}$ \label{gw:add-vertex}
     \If{$\epsilon = \epsilon_1$}
     \State Add $(i,j)$ to $T$: $T \gets T \cup \{(i,j)\}$
     \State Merge $C_i$ and $C_j$ to $C_{ij}$,
     $\mathcal{C} \gets \mathcal{C} \cup \{C_{ij}\} \setminus \{C_i, C_j\}$
     \State If $r \in C_{ij}$ then $\lambda(C_{ij}) \gets 0$ else $\lambda(C_{ij}) \gets 1$
     \State $w(C_{ij}) \gets w(C_i) + w(C_j)$
     \Else
     \State Deactivate $C_w$, $\lambda(C_w) \gets 0$
     \State Mark all $v \in C_w$ with the label $C_w$
     \EndIf
     \EndWhile
     \State $T' \gets \Call{Prune}{T, Labels}$ 
   \EndProcedure
 \end{algorithmic}
 \caption{The GW Algorithm}\label{approx:gw:alg}
 \end{algorithm}

 The growing phase of the GW Algorithm is shown in Algorithm \ref{approx:gw:alg}. Lines \ref{gw:init-start}-\ref{gw:init-end}
 initialise the set of components as described above -- only the component containing the root vertex is set as inactive.
 The lines \ref{gw:select-start}-\ref{gw:add-vertex} then determine the amount, $\epsilon$, which is
 the value that dual decision variables
 can be increased by for all active components
 until a packing constraint becomes binding. Since this behaviour is not directly clear for the definition of the algorithm itself,
  we attempt to give some intuition in the following.

 At the beginning of each iteration we have,
 $$d(v) = \sum_{S: v \in S} y_S, \forall v \in V \mathnormal{.}$$
 This follows directly from line \ref{gw:add-vertex}. Suppose that the corresponding
 packing constraint for the edge $e = (i,j)$ is
  the tightest ``active'' constraint. Notice that we can reformulate the left side of the constraint (\ref{form:gw:pe}) as follows,
 $$\sum_{S: (i,j) \in \delta(S)} y_S = \sum_{S : i \in S} y_S + \sum_{S : j \in S} y_S - \sum_{S : i,j \in S} y_S\mathnormal{.}$$
 Since we are only considering edges which connect components, we clearly must have $\sum_{S : i,j \in S} y_S = 0$
 and thus,
 $$\sum_{S: (i,j) \in \delta(S)} y_S = d(i) + d(j)\mathnormal{.}$$
 Suppose, w.l.o.g., that both $i$ and $j$ belong to active components.
 When the iteration is over, we will have increased $d(i)$ and $d(j)$ by $\epsilon = \frac{c_{ij} - d(i) - d(j)}{2}$.
 Then clearly the packing constraint for $e = (i,j)$ is then binding as demostrated below,
 $$\sum_{S: (i,j) \in \delta(S)} y_S = d(i) + \left( \frac{c_{ij} - d(i) - d(j)}{2} \right) + d(j) +
 \left( \frac{c_{ij} - d(i) - d(j)}{2} \right)   = c_{ij}\mathnormal{.}$$
 The case where a packing constraint (\ref{form:gw:pv}) is the tightest is similar, and follows directly from the loop
 invariant
 $$w(C) = \sum_{S \subset C} y_S, \forall C \in \mathcal{C}\mathnormal{.}$$

 The pruning phase of the algorithm is described by \citet{goemans1995general}
 as a procedure which removes as many
 edges from $T$ while upholding that:
 \begin{enumerate}
 \item any unlabeled vertex is connected to the root vertex, $r$, and
 \item if a vertex with label $C$ is connected to $r$ then any vertex with
   label $C' \supset C$ must also be connected to $r$.
 \end{enumerate}
 This pruning procedure was improved upon by the ``strong pruning'' procedure introduced by \citet{Johnson:2000:PCS:338219.338637},
  which we will detail in Section \ref{sec:approx:strongpruning}.
 \paragraph{Analysis}
 The GW Algorithm is proven to produce a result which is less than twice as bad as an optimal solution to (GW-IP), $Z^*_{\text{GW-IP}}$, and thus
 is a 2-approximation algorithm. This follows from the following inequality,
 $$\sum_{e \in T'}c_e + \sum_{v \in X} p_v \overset{\text{(I)}}{=}   \sum_{e \in T'} \sum_{S: e \in \delta(S)} y_S  + \sum_{j} \sum_{S \subseteq C_j} y_S
 \overset{\text{(II)}}{\leq} (2 - \frac{1}{n-1}) \sum_{S \subset V} y_S \overset{\text{(III)}}{\leq} (2 - \frac{1}{n-1}) Z^*_{\text{GW-IP}}\mathnormal{.}$$
 The reader is encouraged to read \citet{goemans1997primal} for the full proof, but it roughly goes as follows:
 \begin{itemize}
 \item (I) holds from primal complimentary slackness. An edge is only added to $T$ if its corresponding packing constraint is binding.
   Additionally, for a vertex to not be spanned by $T$, it must have been deactivated at some point during the
    algorithm (and thus the corresponding packing constraint must have
    been binding).
  \item (II) holds from an induction proof on main loop of the procedure, and
  \item (III) holds from the feasibility of the $y$ vector which is implicitly procduced by the procedure and then weak duality.
 \end{itemize}
 Furthermore, the GW Algorithm is proven to run in $O(n^2 \log n)$.
 \subsection{Modifications to the GW Algorithm}\label{sec:approx:strongpruning}
 \citet{Johnson:2000:PCS:338219.338637} investigated three main modifications to the GW Algorithm:
 \begin{enumerate}
 \item an improved pruning procedure called Strong Pruning,
 \item how the the algorithm performs on the unrooted PCSTP, and
 \item how pertubation of the prize vector affects the algorithm.
 \end{enumerate}
 Furthermore, as a part of these investigations, they produced an implementation of the GW Algorithm.
 In this section we will detail the Strong Pruning procedure, and touch upon the results from the two other
  investigations.
 \paragraph{Strong Pruning} The Strong Pruning procedure
 (as shown in Algorithm \ref{alg:approx:gwsp})
 is a simple recursive procedure which walks the tree in post-order while calculating the net-worths
 of all subtrees. If --- along the way --- a subtree is encountered which is connected by an
  edge which has higher cost than the net-worth of the subtree, then the subtree is discarded.
 \begin{algorithm}[h!]
   \begin{algorithmic}[1]
     \Input{Feasible solution to (GW-IP), $T$ rooted in vertex $r$ to $G = (V, E, c, p)$} 
     \Output{A tree $T' \subseteq E$.}
     \State $T' \gets T$
     \Procedure{StrongPrune}{r}
     \State $nw(r) \gets p_r$
     \For{$v \in \Call{Children}{r}$}
     \State \Call{StrongPrune}{v}
     \If{$nw(v) \leq c_{rv}$}
     \State $T' \gets T' \setminus \{(r,v)\}$
     \Else
     \State $nw(r) \gets nw(r) + nw(v) - c_{rv}$
     \EndIf
     \EndFor
   \EndProcedure
 \end{algorithmic}
 \caption{Strong pruning for the GW Algorithm.}\label{alg:approx:gwsp}
\end{algorithm}

The Strong Pruning procedure clearly runs in $O(n)$ time, which is an improvement on the $O(n^2)$
guarantee for the pruning procedure presented by \citet{goemans1995general}. Additionally,
Strong Pruning does not require any additional information about the candidate solution, $T$,
which is in contrast with the labels required by the GW Pruning procedure.

However, computational experiments performed by \citet{Johnson:2000:PCS:338219.338637}
do not report any significant speedup gained by replacing GW Pruning with
Strong Pruning in the GW Algorithm
as the algorithm is dominated by its $O(n^2 \log n)$ growing phase.

However, they do report
improvements in objective value. On 12 instances based on street maps,
the vanilla GW Algorithm produced objective values which are in the range of $2-9\%$
worse than those produced by the GW Algorithm modified with Strong Pruning. On randomly
generated instances, this is reported to be up to $20\%$.

Thus, modifying the GW Algorithm to perform Strong Pruning results in better solutions at
 no extra cost.
\paragraph{GW  Algorithm for the unrooted PCSTP}
The GW Algorithm is stated and its approximation ratio proved
for the rooted variant of the PCSTP. While these apply directly to unrooted variants
when run for all possible roots, this comes with
a worsened $O(n^3 \log n)$ runtime.

\citet{Johnson:2000:PCS:338219.338637} verify that if the growth phase of the GW Algorithm
is replaced with a procedure which does not treat the root vertex as special, both the
approximation ratio and asymptotic runtime proven for the rooted variant are still valid.

Computationally, however, they report a $1.5-2.0$ slowdown for a unrooted GW Algorithm. They also
report that running the rooted GW Algorithm repeatedly with randomly selected roots give slightly
better objective value (less than $1\%$ difference) than the unrooted variant, although at a starkly
higher cost.
\paragraph{Prize Pertubation} The third thing investigated by \citet{Johnson:2000:PCS:338219.338637} was
the effect of pertubing vertex prizes by some constant $\alpha$, that is setting
$$p_v \gets \alpha p_v$$
for all $v \in V$. This ``tricks'' the GW Algorithm into over/undervaluing the importance of gathering prizes,
 leading to altered behaviour.

 Experiments performed by \citeauthor{Johnson:2000:PCS:338219.338637} resulted in $\alpha \approx 0.85$ giving a $\sim2\%$
 lower objective values compared to $\alpha = 1$ for the GW Algorithm with Strong Pruning on random instances.
  This kind of prize pertubation is explored further by \citet{canuto2001local} -- see Section \ref{sec:canuto-search}.


%%% Local Variables:
%%% TeX-master: "report"
%%% reftex-default-bibliography: ("lit.bib")
%%% End:


\section{Heuristics}\label{sec:solving:heuristics}
There has been a number of heuristics defined for the \gls{pcstp}:

\begin{itemize}
\item \citet{canuto2001local} presented a local search based heuristics which we
   will detail below.
 \item \citet{Ljubic:2004:memetic} presented a heuristics with inspiration in genetic
   algorithms.
 \item \citet{ljubic2005solving} presented a LP relation based heuristic for their branch
   and bound algorithm which we will detail below.
\item \citet{fu2014knowledge} presented a tabu search heuristic.
 \item \citet{akhmedov2016divide} presented a divide and conquer algorithm based on clustering
    for applications within bioinformatics.
\end{itemize}
We have chosen to show in this section, one search based and one LP based heuristic to give
an overview of general heuristics methods can be applied to the \gls{pcstp}.
\subsection{Local Search with Pertubation and Relinking}\label{sec:canuto-search}

\citet{canuto2001local} presented a multi-phase heuristics which makes use of the GW Algorithm.
 Algorithm \ref{alg:heuristics:canuto} sketches the full procedure. 

 The heuristics can be seen as a multi start heuristics which uses the GW Algorithm
 to generate good intitial
 solutions (line \ref{alg:canuto:line:gw})
 which it progressively refines using a hill climbing local search
 (line \ref{alg:canuto:line:hc}),
 and then a path relinking scheme against a previous good solution
 (line \ref{alg:canuto:line:relink}).
 Pertubation of the prize vector, $p$, is used to push the GW Algorithm to generate
 different solutions on multiple starts, and finally, a
 variable neighbourhood search is applied as a
 post processing step to refine the best candidate solution found.

 \begin{algorithm}[h!]
   \begin{algorithmic}[1]
     \Input{Graph $G = (V, E, c, p)$.}
     \Output{Tree $\tilde{T} \subseteq G$.}
     \Procedure{Canuto-LocalSearch}{}
     \State $E \gets \emptyset$, $\hat{p} \gets p$, $\hat{z}^* \gets \infty$
     \For{$i \in 1...\mathit{max\_iterations}$}
     \State $S \gets \Call{GW-Algorithm}{V,E, c, \hat{p}}$ \label{alg:canuto:line:gw}
     \If{$S$ hasn't been seen in previous iterations}
     \State $S' \gets \Call{Hill-Climbing}{S', G = (V, E, c, p)}$ \label{alg:canuto:line:hc}
     \State Add $S'$ to $E$ if it satisfies the necessary criteria \label{alg:canuto:line:elite}
     \State Pick random $Y \in E$
     \State $S' \gets \Call{Relink}{S', Y}$ \label{alg:canuto:line:relink}
     \If{$c(S') < \hat{z}^*$}
     \State $\hat{z}^* \gets c(S')$
     \State $S^* \gets S'$
     \EndIf
     \EndIf
     \State Pertubate $\hat{p}$
     \EndFor
     \State $S^* \gets \Call{VNS}{S^*, G = (V, E, c, p)}$\label{alg:canuto:line:vns}
     \State \Return $\Call{MST}{G[S^*]}$
     \EndProcedure
 \end{algorithmic}
 \caption{The heuristics defined by \citet{canuto2001local}.}\label{alg:heuristics:canuto}
 \end{algorithm}

 In this section, we will
 describe further the characteristics of the local search, the
 path relinking, and the postprocessing used in the algorithm.
\paragraph{Local Search with Pertubations}
\citet{canuto2001local} define their search space as the space of minimum spanning
trees on all graphs induced by a subset of vertices
 $S \subset V$. If $G[S]$ is the subgraph induced by the vertex subset $S \subseteq V$, then
 this search space can formally be defined as
\[\mathcal{S_G} = \{MST(G[S]) : S \subseteq V \}\mathnormal{.}\]
A neighbour to a given MST is then any MST of a subgraph induced by a subset which differs from $S$
 by exactly one vertex, that is,
\[\mathcal{N}(S) = \Big\{MST(G[S']) : S' \subseteq V \Bigm| |S' \triangle S| = 1 \Big\}\mathnormal{.}\] 
Clearly this neighbourhood structure means that all feasible solutions are reachable
given any initial subset as starting point. Additionally, they define the
$k$'th order neighbourhood as,
\[\mathcal{N}^k(S) = \Big\{MST(G[S']) : S' \subseteq V \Bigm| |S' \triangle S| = k \Big\}\mathnormal{.}\] 

To reduce clutter we will denote the cost
of a given subset, $S \subseteq V$, of vertices as the cost
of the MST of the graph induced by $S$, that is
\[c(S) = c(MST(G[S]))\mathnormal{.}\]

The search algorithm itself is just a plain hill-climbing algorithm which repeatedly
 moves to the first
 found neighbouring solution with lower cost until it arrives at a local minima.

 To overcome the problem of running into the same local minima repeatedly,
 \citet{canuto2001local} employ two pertubation schemes which are alternately applied
 on even and odds iterations:
 \begin{enumerate}
 \item A fixed percentage of vertices which have appear both in the initial solution generated
   by the GW Algorithm \textit{and}  in the solution subsequently generated by
    the hill climbing procedure have their prizes
    set to zero.
  \item A pertubation factor, $\alpha$, is randomly picked within a fixed interval $[1 -a, 1+a]$
    and prizes are set to $p_v \gets \alpha p_v$ for all $v \in V$.
 \end{enumerate}

\paragraph{Path Relinking}
For every iteration of the main loop, the relinking procedure in Algorithm \ref{heuristics:canuto:relink}
is applied to a solution generated by the hill climbing procedure and 
a random \textit{elite} solution which is picked from the
persistant pool of elite solutions, $E$.

The relinking procedure basically transforms $S$ into $Y$ one vertex at a time in a greedy fashion.
The solution along the way which has the best objective value is then returned.
 \begin{algorithm}[h!]
   \begin{algorithmic}[1]
     \Procedure{Apply}{S, v}
     \If{$v \in S$}
     \State \Return $S \setminus \{v\}$
     \Else
     \State \Return $S \cup \{v\}$
     \EndIf
     \EndProcedure
     \Procedure{Relink}{S, Y}
     \State $D \gets S \triangle Y$
     \State $S' \gets S$
     \State $\tilde{S} \gets S$
     \While{$S' \neq Y$}
     \State $v^* \gets \argmin_{v \in D} c(\Call{Apply}{S,v})$
     \State $S' \gets \Call{Apply}{S', v^*}$
     \State if $c(S') < c(\tilde{S})$ then set $\tilde{S} \gets S'$
     \State $D \gets D \setminus \{v\}$
     \EndWhile
     \State \Return $\tilde{S}$
     \EndProcedure
 \end{algorithmic}
 \caption{The relinking scheme used by \citet{canuto2001local}.}\label{heuristics:canuto:relink}
 \end{algorithm}

 Entry into $E$ for a subset $S$ is dermined by two criteria:
 either we must
 \[c(\tilde{S}) < \min_{S \in E} c(S)\mathnormal{,}\]
 or we must have both
 \[c(\tilde{S}) < \max_{S \in E} c(S)\]
  and that the Hamming distances between the characteristic vectors of $\tilde{S}$
   and \textit{all} solutions in $E$ are greater than $\rho |V|$ for some $\rho \leq 1$. 

\paragraph{Postprocessing}
Variable neighbourhood search (VNS) (See \citet{hansen2010variable}) is applied
to best candidate solution as a post-processing step. In principle, the VNS
procedure (shown in Algorithm \ref{alg:heuristics:canuto:vns}) is a hill climbing
search within a union of the first $k_{max}$ order neighbourhoods with an early stopping
 criteria.
\begin{algorithm}[h!]
   \begin{algorithmic}[1]
     \Procedure{VNS}{S}
     \For{$i \in 1...\mathit{max\_iterations}$}
     \State $k \gets 1$
     \While{$k  \leq k_{max}$}
     \State Pick random $S' \in \mathcal{N}^k(S)$
     \State $\tilde{S} \gets \Call{Hill-Climbing}{S'}$
     \If{$c(\tilde{S}) < c(S)$}
     \State $k \gets 1$
     \State $S \gets \tilde{S}$
     \Else
     \State $k \gets k + 1$
     \EndIf
     \EndWhile
     \EndFor
     \State \Return $S$
     \EndProcedure
 \end{algorithmic}
 \caption{The Variable Neighbourhood Search
   used by \citet{canuto2001local}.}\label{alg:heuristics:canuto:vns}
 \end{algorithm}

 As a postprocessing step, this procedure is only applied to the single best solution generated
  by all the previous steps.
\paragraph{Evaluation}
Based on the implementation of the GW Algorithm by \citet{Johnson:2000:PCS:338219.338637},
\citet{canuto2001local} performed experiments, detailing the objective function value
and running time of
\begin{itemize}
\item The GW Algorithm,
\item The multi-start local search procedure with pertubations
\item The above combined with path relinking, and
\item The full heuristics.
\end{itemize}

All three ``parts'' of the heuristics are shown to help improve upon the
objective value generated by the previous parts.
But in particular, \citeauthor{canuto2001local} report that adding the
path relinking scheme results in a stark increase
 in number of instances solved to optimality.

 \subsection{LP Based Heuristic}
 \label{sec:heuristics:lp}
 As part of their branch and bound algorithm for the \gls{pcstp}, \cite{ljubic2005solving} detail a primal
 heuristics which takes advantage of an optimal solution to the LP relaxation of the problem. While this
 is detailed for a directed version of the \gls{pcstp} (see Section~\ref{sec:exact:dhea}) we detail it
 here for the \gls{pcstp}.
 
 Given an optimal solution to the \gls{gsec} and LP relaxation of the ILP
 \[(\bd{ \bar x^*}, \bd{\bar y^*})\]
 of an instance of the \gls{pcstp} defined on the graph $G = (V, E, c, p)$,
 a \textit{good} feasible solution to the \gls{pcstp} as follows.

 Select an initial subset of vertices to be part of the solution based on their decision variable value,
 \[S = \{ i \in V \mid \bar y^*_i > 0.5 \}\mathnormal{.}\]
 \textit{Alternatively, use $\bar y^*_i$ as the probability for including $i$ in $S$.}
 We then augment $G$ to have costs relative to $\bd{\bar{x}}$. More specifically,
 we create the cost function
 \[c'_{ij} = 1 - \bar{x}^*_{ij}\]
 and let it replace $c$ in $G$ to give us the graph,
 \[G' = (V, E, c', p)\mathnormal{.}\]
 With $G'$ and $S$ in hand, we compute the \textit{distance network} of $S$ as
 \[G_S = (S, S \times S, d_S)\]
 where ${d_S}_{ij}$ is the length of the shortest path from $v_i$ to $v_j$ in $G'$.
 Since we use $G'$, a path is short if its edges have high values in the LP
 relaxation.

 Then we find a minimum spanning tree on $G_S$,
 \[T = (S, E_S, d_S) \mathnormal{.}\]
 Let $P_{ij}$ be the shortest path between vertices $i$ and $j$ in $G$, then
 we select a new --- larger --- subset of $V$ by selecting all vertices which
 are implicitly part of an edge in $T$ according to $d_S$, that is
 \[S' = \{ v \in V \mid \exists (i,j) \in E_S, v \in P_{ij}\}\mathnormal{.}\]

 Finally, let $G_H$ be the graph induced by $S'$ on $G$,
 \[G_H = (S', E_H, c, p)\]
 and a minimum spanning tree of $G_H$, $T' = (S', E_{T'}, c, p)$.

 The heuristic solution is then given by solving the \gls{pcstp}
 on $T'$. This can be done in linear time, e.g. by using the
 strong pruning procedure defined by
 \citet{Johnson:2000:PCS:338219.338637}
 (see Section~\ref{sec:approx:strongpruning}).

 
 
 
%%% Local Variables:
%%% TeX-master: "report"
%%% reftex-default-bibliography: ("lit.bib")
%%% End:


\section{Lower Bounding Procedures}
\label{sec:solving:lower}

\subsection{Generalised Subtour Elimination Constraints}
\label{sec:lower:gsec}
\citet{lucena2004strong} introduced a method for producing lower bounds for the PCSTP based on solving
the LP relaxations of the PCSTP using a constraint separation procedure.

Their main contributions
are the stating of a \textit{generalised subtour elimination constraint} ILP (see Formulation \ref{form:lower:gsec}) formulation
for the PCSTP, the separation procedure for solving the LP-relaxation, and strengthening of the formulation by considering
single vertex solution seperately.

Define the function
$$E(S) = \{(i,j) \in E \mid i \in S \wedge j \in S\}$$
then Formulation (\ref{form:lower:gsec}) is a ILP formulation of the PCSTP. Constraints (\ref{form:lower:gsec:gsec}) are
referred to as generalised subtour elimination constraints, and together with constraint (\ref{form:lower:gsec:sum})
ensure that the solution defined by $(x, y)$ is a tree which spans the vertices $y_v = 1$.

 \begin{formulation}[h!]
   \begin{subequations}
     \begin{alignat}{3} 
       &\underset{x,y}{\text{minimize}}
       & & \sum_{e \in E} c_e x_e +  \sum_{v \in V} p_v (1 - y_v)  & \\
       & \text{subject to}\quad
       & & x(E) \leq y(V) - 1 &&  \label{form:lower:gsec:sum}\\
       &&& x(E(S)) \leq y(S \setminus \{s\}) && \forall s \in S, S \subseteq V \label{form:lower:gsec:gsec}\\
       \intertext{ILP:}
       &&& x_e \in \BB  && \forall e \in E \\
       &&& y_v \in \BB  && \forall v \in V \\
       \intertext{LP:}
       &&& 0 \leq x_e \leq 1  && \forall e \in E \\
       &&& 0 \leq y_v \leq 1  && \forall v \in V \\
       \intertext{To exclude single vertex solutions:}
       &&& \sum_{i \in \delta(v)} x_{vi} \geq
       \begin{cases}
         y_v & p_v > 0 \\
         2y_v & p_v = 0
       \end{cases} \qquad && \forall v \in V \label{form:lower:gsec:single}
     \end{alignat}\label{form:lower:gsec}
   \end{subequations}
   \caption{(GSEC-IP) GSEC formulation of the PCSTP.}
 \end{formulation}

 Expanding the formulation with constraint (\ref{form:lower:gsec:single}), excludes single vertex solutions.
 \citet{lucena2004strong} found that adding these constraints and
 considering single vertex solutions in a separate routine results faster runtimes.

 \paragraph{Separation of GSECs} Each subset $S \in V$ produces $|S|$ GSECs. To reduce the size of the LP relaxation,
 \citet{lucena2004strong} suggest relaxing these constraints and readding them in a separation procedure.
 This separation procedure can be stated a max-flow problem as follows:

 Let $(\bar{x}, \bar{y})$ be a solution to the relaxed LP. Consider the support graph induced by vertices and edges with nonzero decision veriables,
  and vertex $l$ with decision variable $\bar y_l > 0$
  then iff
  $$\max_{S_l \subseteq V} \left\{ \bar x (E(S_l)) - (\bar y (S_l) - \bar y_l) \right\} > 0$$
  there is a violated GSEC for a subset containing $l$, and constraint corresponding to $S_l^{max}$ is the most violated constraint.

  We can model finding this subset for a given $l$ as follows. Let $\bar{N} = (\bar{V}, \bar{A})$ be a flow-network
  for the support graph, with vertices
  $$\bar V = \{s, t\} \cup \{v \in V \mid \bar y_v > 0 \}$$
  and arcs
  $$\bar A = \{(i, j) \in E \mid \bar x_{ij} > 0 \} \cup \{(j, i) \in E \mid \bar x_{ij} > 0\} \cup \{(s, v) \mid v \in \bar V\} \cup \{(v, t) \mid v \in \bar V\}\mathnormal{.}$$
  Additionally we have capacities original edges,
  $$\xi_{ij} = \xi_{ji} = \frac{\bar x_{ij}}{2} \qquad \forall (i,j) \in \bar A$$
  and if we let $\delta (v)$ be all vertices adjacent to $v$ in the support graph then node capacities are
  $$\xi_v = \sum_{u \in \delta(v)} \xi_{vu}\mathnormal{,}$$
  and finally
  for arcs from and to the source and the sink, we have
  $$\xi_{sv} =
  \begin{cases}
    \infty & v = l \\
    \max(\xi_v - \bar y_v, 0) & \text{otherwise}
  \end{cases}$$
  and
  $$\xi_{vt} = \max(\bar y - \xi_v, 0)\mathnormal{.}$$
  Then, similarly to \citet{padberg1983trees}, we can see that
  the minimum $(S,T)$ cut on the flow network will minimise
  \begin{align*}
    c(S,T) = &\sum_{v \in S} \max \left\{\bar y_v - \xi_v, 0\right\} +
               \sum_{i \in S, j \in T} \xi_{ij} +
               \sum_{v \in T} \max\left\{\xi_v - \bar y_v, 0\right\} \\
    =&\left(\sum_{v \in S} \max \left\{\bar y_v - \xi_v, 0\right\} - \sum_{v \in S} \max\left\{\xi_v - \bar y_v, 0\right\} \right) +
v       \sum_{i \in S, j \in T}  \xi_{ij} +
       \sum_{v \in V} \max\left\{\xi_v - \bar y_v, 0\right\} \\
    =&\sum_{v \in S} \bar y_v - \xi_v +
       \sum_{i \in S, j \in T} \xi_{ij} +
       \sum_{v \in V} \max\left\{\xi_v - \bar y_v, 0\right\} \\
    =&\sum_{v \in S} \bar y_v -
       \sum_{v \in S} \sum_{u \in \delta(v)} \frac{\bar x_{vu}}{2} +
       \sum_{i \in S, j \in T} \frac{\bar x_{ij}}{2} +
       \sum_{v \in V} \max\left\{\xi_v - \bar y_v, 0\right\} \\
    =&\sum_{v \in S} \bar y_v -
       \left(\sum_{i \in S} \sum_{j \in S \setminus \{i\}} \frac{\bar x_{ij}}{2} +
       \sum_{i \in S, j \in T} \frac{\bar x_{ij}}{2}\right) +
       \sum_{i \in S, j \in T} \frac{\bar x_{ij}}{2} +
       \sum_{v \in V} \max\left\{\xi_v - \bar y_v, 0\right\} \\
    =&\sum_{v \in S} \bar y_v -
       \bar x (E(S)) +
       \sum_{v \in V} \max\left\{\xi_v - \bar y_v, 0\right\}\mathnormal{.}
  \end{align*}
  Since $\sum_{v \in V} \max\left\{\xi_v - \bar y_v, 0\right\}$ is independent of
  any cuts, and
  $l$ \textit{must} be in $S$ due to $\xi_{sy} = \infty$, any minimum cut, $(S,T)$, will maximise
  $$\bar x (E(S_l)) - (\bar y (S_l) - \bar y_l)\mathnormal{.}$$
  Hence any set of vertices $S' = S \setminus \{s\}$ where $S$ induces a minimal $S,T$ cut in the flow network will correspond to
  a most violated GSEC for the solution $(\bar x, \bar y)$ to the relaxed LP.

  Finding violated GSECs then reduces to solving $|V|$ maximum flow problems, which can be done in polynomial time. However, there is guarantee that
  that the $|V|$ will result in different violated constraints. To avoid this, \citet{lucena2004strong} suggest that whenever a max flow problem has been
  solved for some $l = v \in V$, then we can set $\xi_{vt} = \infty$ to avoid $v$ being on the source-side of any subsequent min-cuts.
  \todo[inline]{Wolsey (I think) suggests deleting edges/vertices from the support graph, maybe mention that}

\clearpage
\section{Exact Algorithms}\label{sec:solving:exact}

\subsection{DHEA Solver}
\label{sec:exact:dhea}
The DHEA solver, defined and implemented
by \citet{ljubic2005solving}
(more concisely described in \citet{ljubic2006algorithmic}) finds exact solutions to the PCSTP
by first applying
reduction tests to the input graph (See Section \ref{sec:solving:pre}) and
then transforming the resulting problem into related problem of finding a minimum
aborescence in a directed graph. It is the first example of a procedure designed for
finding provably optimal solutions for the PCSTP.

The resulting SAP is then solved by branch-and-cut using a separation procedure
similar to the one defined in \citet{lucena2004strong}
(Section \ref{sec:lower:gsec}) and a new primal heuristics for
finding good incumbents at every node.

The solver has as a dependency on of the MIP solvers CPLEX or LEDA.
 \paragraph{Reducing the PCSTP to PCSAP}
 Citing stronger LP relaxations, \citet{ljubic2005solving} reduces an input PCSTP instance
 into a related problem in a directed graph which essentially is a constrained version
 of a Prize-Collecting Steiner Aborescence Problem (PCSAP).

 Given an instance of the PCSTP, with graph $G = (V, E, c, p)$ and terminals $N \subset V$,
 we define an instance
 of the PCSAP with directed graph, $G_A = (V_A, A, c')$, where $c'$ define the arc costs,
 by first adding an
 artificial root vector, $r$, to $V$
 $$V_A = \{r\} \cup V\mathnormal{.}$$
 Then we define the arc set by adding bi-directional arcs for each edge in $E$ and
 adding an arc from $r$ to each terminal,
 $$A = \{(i,j) \mid (i,j) \in E \} \cup \{ (j,i) \mid (i,j) \in E \} \cup \{(r, i) \mid i \in N\}\mathnormal{.}$$
 We define arc-costs by subtracting the prize of the arc's endpoint from the original cost of the
 corresponding edge in $G$. Arcs from the root have cost equal to the negative prize of the endpoint vertex. This gives us,
 $$c'_{ij} =
 \begin{cases}
   -p_j & i = r \\
   c_{ij} - p_j & \text{otherwise.}
 \end{cases}
 $$
Figure (\ref{fig:pcstptosap}) shows an example of this kind of reduction.
 \begin{figure}[h]\centering
  \begin{subfigure}[t]{0.47\linewidth}
    
    \begin{tikzpicture}[auto, node distance=1.5 cm]
      %Nodes
  \node[terminal, label={12}] (a) {a};
  \node[steiner] (b) [right=of a] {b};
  \node[steiner] (c) [right=of b] {c};
  \node[terminal, label={10}] (d) [above =of c] {d};
  \node[steiner] (e) [left=of d] {e};
  \node[steiner] (f) [left=of e] {f};
  \node[terminal, label={3}] (g) [above right=0.75 and 1.3 of c] {g};
  % Edges
  \begin{scope}[every edge/.style={draw=black, thick}]
    \draw (a) edge node[below]{4} (b);
    \draw (b) edge node[near start]{5} (d);
    \draw (b) edge node[below]{8} (c);
    \draw (c) edge node{3} (d);
    \draw (c) edge node{2} (g);
    \draw (c) edge node[near start]{5} (e);
    \draw (d) edge node{6} (e);
    \draw (d) edge node{10} (g);
    \draw (e) edge node{1} (f);
  \end{scope}
    \end{tikzpicture}
    \caption{PCSTP Instance (\ref{fig:pcstp:01})}
  \end{subfigure}
  \begin{subfigure}[t]{0.47\linewidth}
    \begin{tikzpicture}[auto, node distance=1.7 cm]
      % Nodes
      \node[terminal] (a) {a};
      \node[steiner] (b) [right=of a] {b};
      \node[steiner] (c) [right=of b] {c};
      \node[terminal] (d) [above =of c] {d};
      \node[steiner] (e) [left=of d] {e};
      \node[steiner] (f) [left=of e] {f};
      \node[terminal] (g) [above right=0.85 and 1.3 of c] {g};
      \node[terminal, root] (r) [below= of g] {r};
      % Edges
      \begin{scope}[every edge/.style={draw=black, thick},
        every node/.style={circle, inner sep=.7}]
        \draw[->] (a) edge[bend right=10] node[below]{4} (b);
        \draw[<-] (a) edge[bend left=10] node{-8} (b);
        \draw[->] (b) edge[bend right=10] node[near start,swap] {-5} (d);
        \draw[<-] (b) edge[bend left=10] node[near start] {5} (d);
        \draw[<->] (b) edge node[below]{8} (c);
        \draw[->] (c) edge[bend right=10] node[swap]{-7} (d);
        \draw[<-] (c) edge[bend left=10] node{3} (d);
        \draw[<-] (c) edge[bend left=10] node{2} (g);
        \draw[->] (c) edge[bend right=10] node[swap]{-1} (g);
        \draw[<->] (c) edge node[near end]{5} (e);
        \draw[->] (d) edge[bend right=10, swap] node{6} (e);
        \draw[<-] (d) edge[bend left=10] node{-4} (e);
        \draw[->] (d) edge[bend right=10] node[swap]{7} (g);
        \draw[<-] (d) edge[bend left=10] node{0} (g);
        \draw[<->] (e) edge node{1} (f);
        \draw[->] (r) edge[bend right=90, looseness=2] node{-10} (d);
        \draw[->] (r) edge node{-3} (g);
        \draw[->] (r) edge[bend left=40] node{-12} (a);
      \end{scope}
    \end{tikzpicture}
    \caption{Instance (\ref{fig:pcstp:01}) as a PCSAP instance. Root node is coloured blue.}
  \end{subfigure}
  \caption{Reduction from PCSTP to SAP.}
  \label{fig:pcstptosap}
\end{figure}

Given a solution, $SA = (V_{SA}, A_{SA}, c')$,
to the transformed problem, we can generate a solution, $T = (V_T, E_T, c)$, to the
original problem as follows.
Take all edges in $G$ which correspond to arcs in $SA$ -- that is, if either of the arcs
$(i, j) \in A$ or $(j, i) \in A$ are members of the solution $A_{SA}$ then we select the edge $(i,j) \in E$.
We select vertices by taking any vertex $v \in V_{SA} \cap V$ which has an in-degree higher than 0 in
$SA$.

However, to ensure that $T$ is indeed a feasible solution to the original problem we
cannot simply find the connected subgraph in $G_A$ which minimises
$$c(SA) = \sum_{a \in A_{SA}} c'_{a} + \sum_{v \in N} p_v\mathnormal{.}$$
Firstly, since all the arcs from the root have negative costs they are trivially part of any minimal arborescence.
As a result, whenever $G$ has more than one terminal, this may result in a disconnected -- hence infeasible --
solution in the original graph.
Furthermore, by representing vertex prizes within the cost of arcs,
 $G_A$  may contain addtional arcs with negative costs. This may result in an
optimal subgraph which is \textit{not} an aborescence as it may contain cycles.

\citet{ljubic2005solving} solve these two issues by adding extra constraints to the problem. Firstly, the
out-degree of the root vector is limited to a single arc, that is $\delta^+(r) = 1$. With this,
we can intepret the selection of an arc in the adjacency of $r$ as an ``initial'' selection,
 of a terminal, or a root,
in the original problem. Enforcing that any solution to the transformed problem must be an aborescence
 is done by constraining the in-degree of any ``selected'' in a feasible solution terminal to one.

 Single arc solutions in $G_A$ will then correspond 1-to-1 with singleton solutions in $G$.
 Connected
 solutions in $G_A$ containing the arc $(r,v)$ will then correspond to a tree in $G$ ``rooted'' in $v$.
 It is worth noting that for such a solution, the arc $(r, v)$ can be replaced with any other arc
 $(r, u), u \in N \cap V_{SA}$ where;
 hence there exists a many-to-one
 relationship between solutions in $G_A$ and solutions in $G$, that is one of each choice of $(r, v)$.
 To ensure that there exists a one-to-one relationship between solutions in $G_{SA}$ and $G$, only the selected 
 terminal
 with the lowest index in a given solution can be connected to the root.
 \begin{figure}[h!]
   \centering
    \begin{tikzpicture}[auto, node distance=1.7 cm]
      % Nodes
      \node[terminal] (a) {a};
      \node[steiner] (b) [right=of a] {b};
      \node[steiner] (c) [right=of b] {c};
      \node[terminal] (d) [above =of c] {d};
      \node[steiner] (e) [left=of d] {e};
      \node[steiner] (f) [left=of e] {f};
      \node[terminal] (g) [above right=0.85 and 1.3 of c] {g};
      \node[terminal, root] (r) [below= of g] {r};
      % Edges
      \begin{scope}[every edge/.style={draw=black, thick},
        every node/.style={circle, inner sep=.7}]
        \draw[->] (a) edge[bend right=10, selected] node[below]{4} (b);
        \draw[<-] (a) edge[bend left=10] node{-8} (b);
        \draw[->] (b) edge[bend right=10, selected] node[near start,swap] {-5} (d);
        \draw[<-] (b) edge[bend left=10] node[near start] {5} (d);
        \draw[<->] (b) edge node[below]{8} (c);
        \draw[->] (c) edge[bend right=10] node[swap]{-7} (d);
        \draw[<-] (c) edge[bend left=10] node{3} (d);
        \draw[<-] (c) edge[bend left=10] node{2} (g);
        \draw[->] (c) edge[bend right=10] node[swap]{-1} (g);
        \draw[<->] (c) edge node[near end]{5} (e);
        \draw[->] (d) edge[bend right=10, swap] node{6} (e);
        \draw[<-] (d) edge[bend left=10] node{-4} (e);
        \draw[->] (d) edge[bend right=10] node[swap]{7} (g);
        \draw[<-] (d) edge[bend left=10] node{0} (g);
        \draw[<->] (e) edge node{1} (f);
        \draw[->] (r) edge[bend right=90, looseness=2] node{-10} (d);
        \draw[->] (r) edge node{-3} (g);
        \draw[->] (r) edge[bend left=40, selected] node{-12} (a);
      \end{scope}
    \end{tikzpicture}
    \caption{Solution to the transformed instance corresponding to the solution in Figure (\ref{fig:pcstp:01:opt}).}
    \label{fig:exact:sap:opt}
  \end{figure}

  With these extra constraints in place, any feasible solution in the transformed, directed graph $G_{A}$ will correspond
  to a feasible solution in the original graph $G$ with the same cost. In other words there exists a bijection between feasible
  solutions to corresponding instances of the constrained SAP and the PCSTP. Figure \ref{fig:exact:sap:opt} shows this for our
   original PCSTP (see Figure \ref{fig:pcstp:01:opt}). This optimal solution has cost
   $$c(SA) = \sum_{ij \in A_{SA}} c'_{ij} + \sum_{v \in V} p_v = (-12 + 4 -5) + (12 + 10 + 3) = 12$$
   which is the same as the corresponding problem to the PCSTP. Note that solution starting with $(r,d)$ would have the same cost,
   $$(-10 + 5 -8) + (12 + 10 + 3) = 12$$
   but is infeasible due to $d > a$.
  \paragraph{ILP Formulation}

Finding such a minimum ``prize-collecting'' aborescence is formalised by \citet{ljubic2005solving}
as the ILP in Fomulation (\ref{form:exact:cut}). 
 \begin{formulation}[h!]
   \begin{subequations}
     \begin{alignat}{3} 
       &\underset{x,y}{\text{minimize}}
       & & \sum_{ij \in A} c'_{ij} x_{ij} +  \sum_{v \in V} p_v  & \\
       & \text{subject to}\quad
       & & \sum_{j \in A} x_{rj} = 1 &&  \label{form:exact:pcsap:root}\\
       &&& \sum_{ji \in A} x_{ji} = y_i  && \forall i \in V_A \label{form:exact:pcsap:prize}\\
       &&& x(\delta^-(S)) \geq y_k && \forall k \in S, S \subseteq V_A \setminus \{r\}
       \label{form:exact:pcsap:conn}\\
       \intertext{ILP:}
       &&& x_a \in \BB  && \forall a \in A \\
       &&& y_v \in \BB  && \forall v \in V_A \\
       \intertext{LP:}
       &&& 0 \leq x_a \leq 1  && \forall a \in A \\
       &&& 0 \leq y_v \leq 1  && \forall v \in V_A \\
       \intertext{To ensure bijection:}
       &&& x_{rj} \leq 1 - y_i && \forall i,j \in V_A \setminus \{r\}; i < j
       \label{form:exact:bijection}\\
       \intertext{Nonterminal Degree:}
       &&& \sum_{j \in A}x_{ij} \leq \sum_{j \in A}x_{ji}
       && \forall i \in V_A \setminus (N \cup \{r\})
       \label{form:exact:strength}\\
       \intertext{Single Orientation:}
       &&& x_{ij} + x_{ji} \leq y_i  && \forall i \in V_A \setminus \{r\}; \forall (i,j) \in A
       \label{form:exact:orientation}
     \end{alignat}\label{form:exact:cut}
   \end{subequations}
   \caption{(CUT-IP) GSEC formulation of the constrained PCSAP.}
 \end{formulation}

 As usual, the ILP formulation has two decision vectors, $x$ and $y$, where the former denotes arcs
 in the solution arborescence and the latter which vertices are in the arborescence. Constraints-wise,
 we have
 constraint (\ref{form:exact:pcsap:root}), which ensures that exactly one arc begins in the artificial
 root in any feasible aborescence. Constraint (\ref{form:exact:pcsap:prize}) ensures
 that a vertex can be selected iff it has an incoming arc.
 Constraints (\ref{form:exact:pcsap:conn}), called \textit{connectivity constraints},
 ensures the connectivity of feasible aborescences by requiring that
 each vertex subset containing a selected vertex must have an incoming arc.
 Together, these constraints ensure that any feasible solution is an aborescence rooted
 in $r$ with $\delta^+(r) = 1$.

 It is worth noting that for a given subset $S \subseteq V$, adding constraints of type
 (\ref{form:exact:pcsap:prize}) for each $v \in S$ to a corresponding connectivity constraint
  gives us generalised subtour elimination constraints:
 \begin{align*}
   -x(\delta^-(S)) &\leq -y_k \\
   \sum_{i \in S} \sum_{ji \in A} x_{ji} - x (\delta^-(S)) &\leq \sum_{v \in S} - y_k \\
   \sum_{i \in S} \sum_{j \in S} x_{ij} &\leq \sum_{v \in S} - y_k \mathnormal{.}
 \end{align*}

 Adding only constraints (\ref{form:exact:pcsap:root})-(\ref{form:exact:pcsap:conn})
 has the unfortunate consequence of having multiple 
 PCSAP solutions
 correspond
 to a single PCSTP solution. To achieve a bijection between the PCSAP solutions and the
 PCSTP solutions, constraint (\ref{form:exact:bijection}) allows only the chosen vertex
 with the lowest index to be adjacent to the root vertex.
 
 Additionally, \citet{ljubic2005solving} observe that each nonterminal in an optimal solution
 \textit{must not} be a leaf in the arborescence. Hence, they add constraint
 (\ref{form:exact:strength}) to strengthen the formulation. And finally, they
 also capture the fact that there cannot be anti-parallel arcs in a feasible aborescence in
 constraint (\ref{form:exact:orientation}). While this constraint is captured implicitly
 in the previous constraints and adding them increases the size of the formulation,
 \citet{ljubic2005solving} report improved running times for adding them as it avoids
 having to implicitly separate the in the separation procedure.
 
\paragraph{Branch \& Bound}

For finding lower bounds, the LP relaxation of (\ref{form:exact:cut}) is solved at each node.
Constraints of type (\ref{form:exact:pcsap:conn}) are separated through solving the max flow (min cut)
problem on a support graph (similar to the procedure in Section \ref{sec:lower:gsec}). 
The max flow problem is solved for every terminal $v \in N$ and returns both the minimum cut closest to
the root, and the minimum cut $(S,T)$ ``closest'' to the sink with $v \in T$.
Violated constraints are iteratively found by increasing the
capacity of all edges crossing a min cut to $1$ and rerunning the max flow procedure.

Primal heuristics based on the solution of LP relaxations are used at each node in the branch and bound
tree to discover improved incumbents.

The rest of the branch and bound procedure is handled by an underlying MIP solver (CPLEX or LEDA).
\subsection{SCIP-Jack Solver}
\label{sec:exact:scipj}
The SCIP-Jack solver is a set of plugins for the SCIP general purpose optimisation suite capable
of generating exact solutions to a variety of STP variants in graphs \cite{gamrath2017scip}.

As with the DHEA solver (Section \ref{sec:exact:dhea}), SCIP-Jack solves the PCSTP by first applying
reduction tests (See section \ref{sec:pre:summary-usage} for details) to the input graph before transforming problem
instances to a constrained version of the SAP and solving it with branch and cut.
The main difference here is in that the DHEA solver works with a prize-collecting variant
 of the SAP problem whereas SCIP-Jack works with the regular version of the SAP.

\paragraph{Reduction to SAP}
As with the DHEA solver, SCIP-Jack internally works on the SAP instead of the solving native
versions of the STP variants. Different from the DHEA solver, SCIP-Jack, however, does not
transform the PCSTP into a prize-collecting SAP. This means that the reduction is slightly
different, as the prize-collecting aspect of the PCSTP has to be described in an alternative
way in the SAP.

Given an instance of the PCSTP with graph $G = (V,E,c,p)$ and terminal set $N$, we wish to define
 a directed graph $G_A = (V_A, A, c')$ with a terminal set $N_A$. The first
 step of this reduction is to add a copy of each terminal to the vertex set, $V_A$.
 To do this, we define a
set of auxiliary terminals as
$$N_A = \{v' \mid v \in N\}$$
which are the terminal set for our SAP.
The original set of vertices $V$ is combined with this set and an
artificial root vector to define the vertex-set in the directed graph
$$V_A = V \cup \{r\} \cup N_A\mathnormal{.}$$
We then add antiparallel arcs for each edge in $E$ as well as arcs from the artifical root vertex, $r$,
to each original terminal and auxiliary terminal and arcs from the original terminals to the auxiliary
terminals,
\begin{align*}
A = &\{(i,j) \mid (i,j) \in E \} \cup \{ (j,i) \mid (i,j) \in E \} \:\cup \\
&\{(r, i) \mid i \in N\} \cup \{(r,i) \mid i \in N_A\} \cup \{(i,j) \mid i \in N, j \in N_A\}\mathnormal{.}
\end{align*}
Finally, the arc-costs in the directed graphs are defined as follows,
$$c'_{ij} =
\begin{cases}
  c_{ij} & (i,j) \in E \\
  p_v & i = r, j \in N_A , j = v' \\
  0 & i = r, j \in N \\
  0 & i \in N, j \in N_A
\end{cases}\textnormal{.}$$
Arcs corresponding to edges in the original graph have the same costs in $G_A$, arcs from the root
to vertices in $N_A$ cost the prize of the vertex, arcs from the root to the original terminals
have no cost and the same goes for arcs from the original terminals to the auxiliary terminals.

\begin{figure}[h]\centering
  \begin{subfigure}[t]{0.47\linewidth}
    
    \begin{tikzpicture}[auto, node distance=1.5 cm]
      %Nodes
  \node[terminal, label={12}] (a) {a};
  \node[steiner] (b) [right=of a] {b};
  \node[steiner] (c) [right=of b] {c};
  \node[terminal, label={10}] (d) [above =of c] {d};
  \node[steiner] (e) [left=of d] {e};
  \node[steiner] (f) [left=of e] {f};
  \node[terminal, label={3}] (g) [above right=0.75 and 1.3 of c] {g};
  % Edges
  \begin{scope}[every edge/.style={draw=black, thick}]
    \draw (a) edge node[below]{4} (b);
    \draw (b) edge node[near start]{5} (d);
    \draw (b) edge node[below]{8} (c);
    \draw (c) edge node{3} (d);
    \draw (c) edge node{2} (g);
    \draw (c) edge node[near start]{5} (e);
    \draw (d) edge node{6} (e);
    \draw (d) edge node{10} (g);
    \draw (e) edge node{1} (f);
  \end{scope}
    \end{tikzpicture}
    \caption{PCSTP Instance (\ref{fig:pcstp:01})}
  \end{subfigure}
  \begin{subfigure}[t]{0.47\linewidth}
    \begin{tikzpicture}[auto, node distance=1.7 cm]
      % Nodes
      \node[steiner] (a) {a};
      \node[terminal] (aa) [below=1cm of a] {a'};
      \node[steiner] (b) [right=of a] {b};
      \node[steiner] (c) [right=of b] {c};
      \node[steiner] (d) [above =of c] {d};
      \node[terminal] (dd) [above= 0.8cm of d] {d'};
      \node[steiner] (e) [left=of d] {e};
      \node[steiner] (f) [left=of e] {f};
      \node[steiner] (g) [above right=0.85 and 1.3 of c] {g};
      \node[terminal] (gg) [below= 0.8cm of g] {g'};
      \node[terminal, root] (r) [right= of g] {r};
      % Edges
      \begin{scope}[every edge/.style={draw=black, thick},
        every node/.style={circle, inner sep=.7}]
        \draw[<->] (a) edge node[below]{4} (b);
        \draw[->] (a) edge node {0} (aa);
        \draw[<->] (b) edge node[near start] {5} (d);
        \draw[<->] (b) edge node[below]{8} (c);
        \draw[<->] (c) edge node{3} (d);
        \draw[<->] (c) edge node{2} (g);
        \draw[<->] (c) edge node[near end]{5} (e);
        \draw[->] (d) edge node {0} (dd);
        \draw[<->] (d) edge node{6} (e);
        \draw[<->] (d) edge node[swap]{7} (g);
        \draw[<->] (e) edge node{1} (f);
        \draw[->] (g) edge node{0} (gg);
        \draw[->] (r) edge[bend right=40] node{0} (d);
        \draw[->] (r) edge[bend right=40] node{10} (dd);
        \draw[->] (r) edge node{0} (g);
        \draw[->] (r) edge node{3} (gg);
        \draw[->] (r) edge[bend left=45, looseness=1.0] node{0} (a);
        \draw[->] (r) edge[bend left=50, looseness=0.7] node{12} (aa);
      \end{scope}
    \end{tikzpicture}
    \caption{Instance (\ref{fig:pcstp:01}) as a SAP instance. Root node is coloured blue.}
  \end{subfigure}
  \caption{Reduction from PCSTP to SAP.}
  \label{fig:scip:pcstptosap}
\end{figure}

Figure \ref{fig:scip:pcstptosap} shows this transfomation. Intuitively, we see that
picking an arc which goes from the root to a vertex in $v' \in N_A$
corresponds to not picking the terminal $v$ in the original graph. Selecting an arc from the root
to some vertex $v \in N$ (and the arc $(v, v')$) then corresponds to selecting $v$
in the original graph.. Selecting any arcs between two vertices which are found in $G$ corresponds
directly to selecting the corresponding edge in $G$.

\begin{figure}[t]
  \centering
    \begin{tikzpicture}[auto, node distance=1.7 cm]
      % Nodes
      \node[steiner] (a) {a};
      \node[terminal] (aa) [below=1cm of a] {a'};
      \node[steiner] (b) [right=of a] {b};
      \node[steiner] (c) [right=of b] {c};
      \node[steiner] (d) [above =of c] {d};
      \node[terminal] (dd) [above= 0.8cm of d] {d'};
      \node[steiner] (e) [left=of d] {e};
      \node[steiner] (f) [left=of e] {f};
      \node[steiner] (g) [above right=0.85 and 1.3 of c] {g};
      \node[terminal] (gg) [below= 0.8cm of g] {g'};
      \node[terminal, root] (r) [right= of g] {r};
      % Edges
      \begin{scope}[every edge/.style={draw=black, thick},
        every node/.style={circle, inner sep=.7}]
        \draw[<-] (a) edge (b);
        \draw[->] (a) edge[selected] node[below]{4} (b);
        \draw[->] (a) edge[selected] node {0} (aa);
        \draw[<-] (b) edge (d);
        \draw[->] (b) edge[selected] node[near start] {5} (d);
        \draw[<->] (b) edge node[below]{8} (c);
        \draw[<->] (c) edge node{3} (d);
        \draw[<->] (c) edge node{2} (g);
        \draw[<->] (c) edge node[near end]{5} (e);
        \draw[->] (d) edge [selected] node {0} (dd);
        \draw[<->] (d) edge node{6} (e);
        \draw[<->] (d) edge node[swap]{7} (g);
        \draw[<->] (e) edge node{1} (f);
        \draw[->] (g) edge node{0} (gg);
        \draw[->] (r) edge[bend right=40] node{0} (d);
        \draw[->] (r) edge[bend right=40] node{10} (dd);
        \draw[->] (r) edge node{0} (g);
        \draw[->] (r) edge[selected] node{3} (gg);
        \draw[->] (r) edge[bend left=45, looseness=1.0, selected] node{0} (a);
        \draw[->] (r) edge[bend left=50, looseness=0.7] node{12} (aa);
      \end{scope}
    \end{tikzpicture}
    \caption{Optimal solution to Figure (\ref{fig:exact:sap:opt}) corresponding to solution (\ref{fig:pcstp:01:opt}).}
    \label{fig:scip:sap:opt}
  \end{figure}

  Figure (\ref{fig:scip:sap:opt}) shows the optimal solution in Figure (\ref{fig:pcstp:01:opt}) in a transformed graph.
  Here we see how terminal $g$ not being selected is represented by selecting the arc $(r, g')$, incurring its prize as
  a penalty. The solution has total cost,
  $$c(SA) = 0 + 0 + 4 + 5 + 0 + 3 = 12$$
  which is the same as the corresponding solution in the optimal graph.

 \paragraph{ILP Formulation}
 \cite{gamrath2017scip} define the ILP formulation of the constrained SAP in Formulation (\ref{form:sap:flow})
 named \textit{flow-balance directed cut formulation}. Constraint (\ref{form:exact:sap:cut}) ensures that there exists a
  path from the root to any terminal by ensuring that any
 cut in the graph which does not contain all terminals must have outgoing arcs. Constraint (\ref{form:exact:sap:cut})
 ensures that all terminals have exactly one incoming arc, nonterminals have at most one incoming arc and the root has
 no incoming arcs. Constraint (\ref{form:exact:sap:indeg}) ensures that nonterminals cannot be leafs in a solution.
 Constraint (\ref{form:exact:sap:deg}) ensures that the solution is connected by ensuring that an arc can only be selected
 if its starting point is pointed to by a selected arc.

 Finally, as with the DHEA solver constraint (\ref{form:sap:root})
 ensures that exact one ``free'' arc from the root can be selected, and (\ref{form:sap:bijection}) creates a bijection
  between solutions in the SAP and PCSTP.
 \begin{formulation}[h!]
   \begin{subequations}
     \begin{alignat}{3} 
       &\underset{x}{\text{minimize}}
       & & \sum_{ij \in A} c'_{ij} x_{ij} & \\
       & \text{subject to}\quad
       && \sum_{v \in \delta^+(r), c_{rv} = 0} x_{rv} = 1 && 
       \label{form:sap:root}\\
       &&& x(\delta^+(S)) \geq 1 && \forall S \subset V, r \in S, N \not\subset S
       \label{form:exact:sap:cut}\\
       &&& x(\delta^-(v))
       \begin{cases}
         = 0 & v = r \\
         = 1 & v \in N \\
         \leq 1 & v \in V_A \setminus (N \cup \{r\})
       \end{cases}
       && \forall v \in V_A \label{form:exact:sap:indeg}\\
       &&& x(\delta^-(v)) \leq x(\delta^+(v)) && \forall v \in V \setminus N
       \label{form:exact:sap:deg}\\
       &&& x(\delta^-(v)) \geq x_{vu} && \forall v \in V \setminus N, \forall u \in \delta^+(v)
       \label{form:exact:sap:conn}\\
       \intertext{ILP:}
       &&& x_a \in \BB  && \forall a \in A \\
       \intertext{LP:}
       &&& 0 \leq x_a \leq 1  && \forall a \in A \\
       \intertext{To ensure bijection:}
       &&& \sum_{i \in \delta^-(j)} x_{ij} + x_{rv}
       && \forall v = 1,...,|V_A|, j = 1,...,i-1
       \label{form:sap:bijection}
     \end{alignat}\label{form:sap:flow}
   \end{subequations}
   \caption{(FLW-IP) Flow-balance directed cut formulation of the constrained SAP.}
 \end{formulation}

 Given a solution to the an instance of the constrained SAP as defined here, a solution to the PCSTP can then
 be obtained by taking an edge for all arcs in the solution for which both endpoints are vertices
 in the original graph. The tree induced by this set of edges then is the corresponding solution in the original
 PCSTP instance.
 
  \paragraph{Branch \& Bound}
  While node selection is left to the underlying MIP solver, SCIP-Jack branches on vertices in the following way. For each node in the
  b\&b tree with optimal solution $x^*$ to the LP relaxation, the nonterminal, $v \in V_A \setminus N_A$, which minimises
  $$v_b = \argmin_{v \in V_A \setminus N_A} \left| \sum_{u \in \delta^-(v)} x^*_{uv} - 0.5 \right|$$
  is added to the set of terminals in one child of the b\&b node and excluded completely in the other.
  \cite{gamrath2017scip} claim that this gives stronger performance than
  to branching on single arcs.

  Futhermore, SCIP-Jack contains an array of heuristics for the SAP which we will not delve further into.

\subsection{Difference in Approach}

% Both use SAP transformations
% SCIP uses an actual SAP instead of the PCSAPish favoured by Ljubic

% Both make use of multipurpose solvers
% Both use the max flow separation of the GSECs
% Both heavily inspired by STP research


%%% Local Variables:
%%% TeX-master: "report"
%%% reftex-default-bibliography: ("lit.bib")
%%% End:


%%% Local Variables:
%%% TeX-master: "report"
%%% reftex-default-bibliography: ("lit.bib")
%%% End:
