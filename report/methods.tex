\chapter{Solving the Prize-Collecting Steiner Tree Problem}
\label{chap:solving}
\section{Preprocessing}
Applying preprocessing routines to heavily reduce input graphs is a common technique which has been proven succesful in many cases both for instances of the STP
\citep{koch1998solving}
and instances of the PCSTP
\citep{ljubic2005solving, gamrath2017scip}. %TODO: MORE
These routines make use of proven invariants to remove and contract edges as well as choose edges before applying the any main procedure.
A set of common preprocessing routines are applied in different manners in literature, mainly differentiated by:
\begin{enumerate}[label=\alph*)]
\item which routines to apply,
\item in which order, and
\item when to recursively apply routines and how many times.
\end{enumerate}
Preprocessing routines can be very effective. For example, the preprocesing routine presented by \cite{koch1998solving} for the STP removes
up to 98\% of edges in some instances. In this section, we will give an overview of the preprocessing methods applied in
 recent literature.
\subsection{Simple Reduction Tests}
When an edge or vertex is provably part of an optimal solution, we say that the edge/vertex is \textit{choosable}, and similarly when
 an edge or vertex is provably not-part of an optimal solution, we say that the edge/vertex is \textit{redundant}.

\subsubsection{Non-Terminals of Degree 1}
\label{sec:red:test:deg1}
Let $G = (V, E, c, p)$ be a PCST instance and let $v \in V$ be any non-terminal with degree 1, then
 clearly -- since edges have positive weights -- $v$ can not be part of an optimal solution. Thus $v$ is redundant. In other words,
 all vertices in the set
 $$\{v \mid v \in V \setminus N \wedge |\delta(v)| = 1\}$$
 are redundant and can be removed from the graph along with their adjacent edges.

\begin{figure}[h]\centering
    \begin{tikzpicture}[auto, node distance=1.3 cm]
      % Pre
      \begin{scope}[shift={(-0.5,0)}]
        \node[terminal] (a) at (0, 0) {a};
        \node[steiner] (b) [left=of a] {v};
        \node (sg) [above right = 1.3 and 0.1 of a] {};
        \node (sg2) [below right= 1.3 and 0.1 of a] {};

        %Edges
        \draw (a) edge node {$k$} (b);
        \draw[dashed] (a) to (sg);
        \draw[dashed] (a) to (sg2);
      \end{scope}

      \draw [->,decorate, thick,
      decoration={snake,amplitude=.4mm,segment length=2mm,post length=1mm}]
      (1.0,0) -- (3,0);
      % Post      
      \begin{scope}[shift={(4cm, 0)}]

        \node[terminal] (a) at (0, 0) {a};
        \node (sg) [above right = 1.3 and 0.1 of a] {};
        \node (sg2) [below right= 1.3 and 0.1 of a] {};

        %Edges
        \draw[dashed] (a) to (sg);
        \draw[dashed] (a) to (sg2);
      \end{scope}

  \end{tikzpicture}
  \caption{Removing a non-terminal with degree 1.}
  \label{fig:red:test:deg1}
\end{figure}

 While this reduction test is originally stated for the STP \citep{hwang1992steiner}, it is also applicable to the PCSTP. Figure (\ref{fig:red:test:deg1})
  shows an example of removing a degree 1 non-terminal.

  \subsubsection{Non-Terminals of Degree 2}
    \label{fig:red:test2}
\todo{These two sections probably use too different methods in defining their reductions}
    Similarly, let $G = (V,E,c,p)$ be an instance of the PCSTP, let $v \in G$ be a non-terminal with degree $|\delta(v)| = 2$, and let
    $u$ and $w$ be the two vertices adjacent to $v$. Then we can obtain a reduced, equivalent graph,
    $$G' = (V', E', c', p)$$
    where
    $$V' = V - v \mathnormal{,}$$
    $$E' = (E \setminus \{(u,v),(w,v)\}) \cup \{(u,w)\}\mathnormal{,}$$
    and
    $$c_{uw} =
    \begin{cases}
      \min(c_{uw}, c_{uv} + c_{vw}) & (u,w) \in E\mathnormal{,} \\
      c_{uv} + c_{vw} & \text{otherwise.}
    \end{cases}$$

    In other words, if $c_{uv} + c_{vw} <  c_{uw}$ then $(u,w)$ is redundant, can be removed, and $v$ and its edges
    can be contracted to a single edge.
    Otherwise $v$ and its edges are redundant and can be removed. Figure (\ref{fig:red:test:deg2})
    shows an example this reduction test.

\begin{figure}[h]\centering
    \begin{tikzpicture}[auto, node distance=1.3 cm]
      % Pre
      \begin{scope}
        \node[terminal] (a) at (0, -0.65) {u};
        \node[terminal] (b) [above=of a] {w};
        \node[steiner] (c) [above left= 0.65 and 1.3 of a] {v};
        \node (sg) [above right = 1.3 and 0.1 of b] {};
        \node (sg2) [below right= 1.3 and 0.1 of a] {};

        %Edges
        \draw (a) edge node {$k$} (b);
        \draw (a) edge node {$c_1$} (c);
        \draw (b) edge node[swap] {$c_2$} (c);
        \draw[dashed] (b) to (sg);
        \draw[dashed] (a) to (sg2);
      \end{scope}

      \draw [->,decorate, thick,
      decoration={snake,amplitude=.4mm,segment length=2mm,post length=1mm}]
      (1,0) -- (3.25,0);
      % Post      
      \begin{scope}[shift={(4.5cm, 0)}]
        \node[terminal] (a) at (0, -0.65) {u};
        \node[terminal] (b) [above=of a] {w};
        \node (sg) [above right = 1.3 and 0.1 of b] {};
        \node (sg2) [below right= 1.3 and 0.1 of a] {};

        %Edges
        \draw (a) edge node[swap] {$\min(k, c_1 + c_2)$} (b);
        \draw[dashed] (b) to (sg);
        \draw[dashed] (a) to (sg2);
      \end{scope}

  \end{tikzpicture}
  \caption{Removing a non-terminal with degree 2.}
  \label{fig:red:test:deg2}
\end{figure}

This test is another example of a reduction test for the STP which can by directly applied to
 the PCSTP.
\subsubsection{Terminals of Degree 1}
\label{sec:red:test:tdeg1}

\begin{figure}[h!]\centering
    \begin{tikzpicture}[auto, node distance=1.3 cm]
      % Pre
      \begin{scope}[shift={(-0.5,0)}]
        \node[terminal, label={$p_a$}] (a) at (0, 0) {a};
        \node[terminal, label={$p_v$}] (b) [left=of a] {v};
        \node (sg) [above right = 1.3 and 0.1 of a] {};
        \node (sg2) [below right= 1.3 and 0.1 of a] {};

        %Edges
        \draw (a) edge node {$k$} (b);
        \draw[dashed] (a) to (sg);
        \draw[dashed] (a) to (sg2);
      \end{scope}

      \draw [->,decorate, thick,
      decoration={snake,amplitude=.4mm,segment length=2mm,post length=1mm}]
      (1.0,1) -- node [above=1mm,midway,text width=3cm, sloped, align=center] {$k\geq p_v$}
      (3,2);
      % Post      
      \begin{scope}[shift={(4cm, 3cm)}]

        \node[terminal, label=right:{$p_a$}] (a) at (0, 0) {a};
        \node (sg) [above right = 1.3 and 0.1 of a] {};
        \node (sg2) [below right= 1.3 and 0.1 of a] {};

        %Edges
        \draw[dashed] (a) to (sg);
        \draw[dashed] (a) to (sg2);
      \end{scope}

      \draw [->,decorate, thick,
      decoration={snake,amplitude=.4mm,segment length=2mm,post length=1mm}]
      (1.0,-1) -- node [above=1mm,midway,text width=3cm, sloped, align=center] {$k < p_v$}
      (3,-2);

      % Post2      
      \begin{scope}[shift={(4cm, -3cm)}]

        \node[terminal, label=right:{$p_a + (p_v - k)$}] (a) at (0, 0) {a};
        \node (sg) [above right = 1.3 and 0.1 of a] {};
        \node (sg2) [below right= 1.3 and 0.1 of a] {};

        %Edges
        \draw[dashed] (a) to (sg);
        \draw[dashed] (a) to (sg2);
      \end{scope}

  \end{tikzpicture}
  \caption{Removing a non-terminal with degree 1.}
  \label{fig:red:test:deg1}
\end{figure}

\subsection{Steiner Distance Reduction Tests}
% $$d(u,v) = \min \{ c(P) \mid P \in \mathcal{P}_{uv}\}$$
More complex tests can be described in terms
of Steiner distances and Bottleneck distances. There were originally stated for
the STP in, amongst others, \cite{duin1989edge,duin1989reduction} and later
adapted for the PCSTP in \cite{uchoa2006reduction}.

First we must make some definitions.
If $P = (..., u, ..., w, ...)$ is a simple path in $G$, then $P_{uw}$ is
the subpath in $P$ starting at vertex $u$ and ending in vertex $w$,
\todo{Some of this should probably go in a notations section}
that is $P_{u,w} = (u, ...,w)$. Then we define the \textit{Steiner distance} of
 $P_{uw}$ as,

 $$sd(P_{uw}) = \sum_{(i,j) \in E(P_{u,w})} c_{i,j} -
 \sum_{v \in V(P) \setminus \{u,w\}} p_{v}\mathnormal{.}$$
 We then denote the Steiner distance of a simple path, $P$, as the maximal Steiner
 distance found among subpaths of $P$,
 $$sd(P) = \max_{u,w \in P} sd(P_{uw})\mathnormal{.}$$
 Let $\mathcal{P}_{uw}$ be the set of all simple paths
 connecting vertices $u$ and $w$ in
 $G$, then we denote the \textit{bottleneck distance} between $u$ and $w$ as minimal Steiner
  distance among paths in $\mathcal{P}_{uw}$,
  $$B(u,w) = \min_{P \in  \mathcal{P}_{uw}} sd(P)\mathnormal{.}$$
  The bottleneck distance is a measure of the worst case \textit{additional cost}
  \todo{Make decision on whether I should replicated proofs for claims like this.}
 of connecting
 two disjoint subgraphs in $G$.


  Finally, we denote the bottleneck distance between vertices $u$ and $w$ \textit{excluding}
  the edge $e$ as, \missingfigure{Some visual intuition on the bottleneck distance}
  $$B(u,w)^{-e} = \min_{P \in  \mathcal{P}_{uw}, e \not \in P} sd(P)\mathnormal{.}$$


 \cite{uchoa2006reduction} shows that calculating the bottleneck distance
 is an NP hard problem by reduction from the Hamiltonian Path problem. Hence,
 exactly calculating the bottleneck distance is infeasible. However, \cite{uchoa2006reduction}
 also claims that existing heuristics are fast and give strong upper bounds.
 \subsubsection{Special Distance Test}
 \label{sec:red:test:sd}
 \begin{theorem}
 Consider any edge $(u,v) \in E$. If we have
 $$B(u,v)^{-(u,v)} \leq c_{uv}$$
 then $(u,v)$ is redundant.
\end{theorem}
 \begin{proof}
   Let $T  \subseteq G$ be an optimal solution to the PCSTP in graph $G$
   where we have
   $$B(u,v)^{-(u,v)} \leq c_{uv}$$
   for some edge $(u,v) \in E_T$, and let $(T_1, T_2)$ be the cut bridged
   by $(u,v)$.

   Consider then the simple path $P \in \mathcal{P}_{uv}$ from $u$ to $v$ which doesn't
    contain $(u,v)$ and has
   $$sd(P) = B(u,v)^{-(u,v)}\mathnormal{.}$$

   Since $T$ is a tree, and $P$ by definition doesn't contain the edge $(u,v)$
   then we must have that $P$ consists of a subpath contained in $T_1$, followed by
   a subpath not contained in $T$, followed by a subpath contained in $T_2$.
   In other words, we have,
   $$P = (u, P_1, w, P_2, z, P_3, v)$$
   as shown in Figure \ref{fig:red:test:sd:thm}.

   Then by definition we have,
   $$sd(P_2) \leq sd(P) = B(u,v)^{-(u,v)} \leq c_{uv}$$
   and we construct another solution $T'$ by replacing $(u,v)$ with the
   vertices and edges in $P_2$ which has cost
   $$c(T') = c(T) - c_{uv} + sd(P_2) \leq c(T)\mathnormal{.}$$
   Hence $T'$ also optimal in $G$ and $(u,v)$ is by definition redundant.
\end{proof}
\begin{figure}[h!]\centering
    \begin{tikzpicture}[auto, node distance=1.3 cm]
      % Pre
      \begin{scope}[shift={(-0.5,0)}]
        \node[terminal] (u) at (0, 0) {u};
        \node[subgraph, above left= 0.05cm and 1.4cm of u] {$T_1$};
        \node[steiner, densely dashed] (w1) [above=of u] {w};
        
        \node[terminal] (v) [right= of u] {v};
        \node[subgraph,above right= 0.05 and 1.4cm  of v] {$T_2$};
        \node[steiner, densely dashed] (w2) [above=of v] {z};
        \node (sga) [left= of u] {};
        \node (sga2) [below left= 1.3 and 0.1 of u] {};

        \node (sgb) [right=  of v] {};
        \node (sgb2) [below right= 1.3 and 0.1 of v] {};

        
        
        %Edges
        \draw[selected] (u) edge node {$k$} (v);
        \draw[dashed, selected] (u) to (sga);
        \draw[dashed, selected] (u) to (sga2);

        \draw[dashed, selected] (v) to (sgb);
        \draw[dashed, selected] (v) to (sgb2);

        \draw (u) edge[snake it, red, bend left] node[text=black] {$P_1$} (w1);
        \draw (w1) edge[snake it] node {$P_2$} (w2);
        \draw (w2) edge[snake it, red, bend left] node[text=black] {$P_3$} (v);
      \end{scope}

      % \draw [->,decorate, thick,
      % decoration={snake,amplitude=.4mm,segment length=2mm,post length=1mm}]
      % (4,0) -- node [above=1mm,midway,text width=3cm, sloped, align=center] {}
      % (6,0);
      % Post      

  \end{tikzpicture}
  \caption{Edge in an optimal solution with $c(u,v) \geq B(u,v)^{-(u,v)}$.}
  \label{fig:red:test:sd:thm}
\end{figure}

\subsubsection{Non-terminals of degree 3}
\label{sec:red:test:deg3}

\section{Primal Heuristics}

\section{Exact Algorithms}
%%% Local Variables:
%%% TeX-master: "report"
%%% reftex-default-bibliography: ("lit.bib")
%%% End:
