\chapter{Solving the Prize-Collecting Steiner Tree Problem}
\label{chap:solving}
\section{Preprocessing}
Applying preprocessing routines to heavily reduce input graphs is a common technique which has been proven succesful in many cases both for instances of the STP
\citep{koch1998solving}
and instances of the PCSTP
\citep{ljubic2005solving, gamrath2017scip}. %TODO: MORE
These routines make use of proven invariants to remove and contract edges as well as choose edges before applying the any main procedure.
A set of common preprocessing routines are applied in different manners in literature, mainly differentiated by:
\begin{enumerate}[label=\alph*)]
\item which routines to apply,
\item in which order, and
\item when to recursively apply routines and how many times.
\end{enumerate}
Preprocessing routines can be very effective. For example, the preprocesing routine presented by \cite{koch1998solving} for the STP removes
up to 98\% of edges in some instances. In this section, we will give an overview of the preprocessing methods applied in
 recent literature.
\subsection{Simple Reduction Tests}
When an edge or vertex is provably part of an optimal solution, we say that the edge/vertex is \textit{choosable}, and similarly when
 an edge or vertex is provably not-part of an optimal solution, we say that the edge/vertex is \textit{redundant}.

\subsubsection{Non-Terminals of Degree 1}
\label{sec:red:test:deg1}
Let $G = (V, E, c, p)$ be a PCST instance and let $v \in V$ be any non-terminal with degree 1, then
 clearly -- since edges have positive weights -- $v$ can not be part of an optimal solution. Thus $v$ is redundant. In other words,
 all vertices in the set
 $$\{v \mid v \in V \setminus N \wedge |\delta(v)| = 1\}$$
 are redundant and can be removed from the graph along with their adjacent edges.

\begin{figure}[h]\centering
    \begin{tikzpicture}[auto, node distance=1.3 cm]
      % Pre
      \begin{scope}[shift={(-0.5,0)}]
        \node[terminal] (a) at (0, 0) {a};
        \node[steiner] (b) [left=of a] {v};
        \node (sg) [above right = 1.3 and 0.1 of a] {};
        \node (sg2) [below right= 1.3 and 0.1 of a] {};

        %Edges
        \draw (a) edge node {$k$} (b);
        \draw[dashed] (a) to (sg);
        \draw[dashed] (a) to (sg2);
      \end{scope}

      \draw [->,decorate, thick,
      decoration={snake,amplitude=.4mm,segment length=2mm,post length=1mm}]
      (1.0,0) -- (3,0);
      % Post      
      \begin{scope}[shift={(4cm, 0)}]

        \node[terminal] (a) at (0, 0) {a};
        \node (sg) [above right = 1.3 and 0.1 of a] {};
        \node (sg2) [below right= 1.3 and 0.1 of a] {};

        %Edges
        \draw[dashed] (a) to (sg);
        \draw[dashed] (a) to (sg2);
      \end{scope}

  \end{tikzpicture}
  \caption{Removing a non-terminal with degree 1.}
  \label{fig:red:test:deg1}
\end{figure}

 While this reduction test is originally stated for the STP \citep{hwang1992steiner}, it is also applicable to the PCSTP. Figure (\ref{fig:red:test:deg1})
  shows an example of removing a degree 1 non-terminal.

  \subsubsection{Non-Terminals of Degree 2}
    \label{fig:red:test2}

    Similarly, let $G = (V,E,c,p)$ be an instance of the PCSTP, let $v \in G$ be a non-terminal with degree $|\delta(v)| = 2$, and let
    $u$ and $w$ be the two vertices adjacent to $v$. Then we can obtain a reduced, equivalent graph,
    $$G' = (V', E', c', p)$$
    where
    $$V' = V - v \mathnormal{,}$$
    $$E' = (E \setminus \{(u,v),(w,v)\}) \cup \{(u,w)\}\mathnormal{,}$$
    and
    $$c_{uw} =
    \begin{cases}
      \min(c_{uw}, c_{uv} + c_{vw}) & (u,w) \in E\mathnormal{,} \\
      c_{uv} + c_{wv} & \mathnormal{otherwise.}
    \end{cases}$$

    In other words, if $c_{uv} + c_{vw} <  c_{uw}$ then $(u,w)$ is redundant, can be removed, and $v$ and its edges
    can be contracted to a single edge.
    Otherwise $v$ and its edges are redundant and can be removed. Figure (\ref{fig:red:test:deg2})
    shows an example this reduction test.

\begin{figure}[h]\centering
    \begin{tikzpicture}[auto, node distance=1.3 cm]
      % Pre
      \begin{scope}
        \node[terminal] (a) at (0, -0.65) {u};
        \node[terminal] (b) [above=of a] {w};
        \node[steiner] (c) [above left= 0.65 and 1.3 of a] {v};
        \node (sg) [above right = 1.3 and 0.1 of b] {};
        \node (sg2) [below right= 1.3 and 0.1 of a] {};

        %Edges
        \draw (a) edge node {$k$} (b);
        \draw (a) edge node {$c_1$} (c);
        \draw (b) edge node[swap] {$c_2$} (c);
        \draw[dashed] (b) to (sg);
        \draw[dashed] (a) to (sg2);
      \end{scope}

      \draw [->,decorate, thick,
      decoration={snake,amplitude=.4mm,segment length=2mm,post length=1mm}]
      (1,0) -- (3.25,0);
      % Post      
      \begin{scope}[shift={(4.5cm, 0)}]
        \node[terminal] (a) at (0, -0.65) {u};
        \node[terminal] (b) [above=of a] {w};
        \node (sg) [above right = 1.3 and 0.1 of b] {};
        \node (sg2) [below right= 1.3 and 0.1 of a] {};

        %Edges
        \draw (a) edge node[swap] {$\min(k, c_1 + c_2)$} (b);
        \draw[dashed] (b) to (sg);
        \draw[dashed] (a) to (sg2);
      \end{scope}

  \end{tikzpicture}
  \caption{Removing a non-terminal with degree 2.}
  \label{fig:red:test:deg2}
\end{figure}

This test is another example of a reduction test for the STP which can by directly applied to
 the PCSTP.
\subsubsection{Terminals of Degree 1}
\label{sec:red:test:tdeg1}

\begin{figure}[h]\centering
    \begin{tikzpicture}[auto, node distance=1.3 cm]
      % Pre
      \begin{scope}[shift={(-0.5,0)}]
        \node[terminal, label={$p_a$}] (a) at (0, 0) {a};
        \node[terminal, label={$p_v$}] (b) [left=of a] {v};
        \node (sg) [above right = 1.3 and 0.1 of a] {};
        \node (sg2) [below right= 1.3 and 0.1 of a] {};

        %Edges
        \draw (a) edge node {$k$} (b);
        \draw[dashed] (a) to (sg);
        \draw[dashed] (a) to (sg2);
      \end{scope}

      \draw [->,decorate, thick,
      decoration={snake,amplitude=.4mm,segment length=2mm,post length=1mm}]
      (1.0,1) -- node [above=1mm,midway,text width=3cm, sloped, align=center] {$k\geq p_v$}
      (3,2);
      % Post      
      \begin{scope}[shift={(4cm, 3cm)}]

        \node[terminal, label=right:{$p_a$}] (a) at (0, 0) {a};
        \node (sg) [above right = 1.3 and 0.1 of a] {};
        \node (sg2) [below right= 1.3 and 0.1 of a] {};

        %Edges
        \draw[dashed] (a) to (sg);
        \draw[dashed] (a) to (sg2);
      \end{scope}

      \draw [->,decorate, thick,
      decoration={snake,amplitude=.4mm,segment length=2mm,post length=1mm}]
      (1.0,-1) -- node [above=1mm,midway,text width=3cm, sloped, align=center] {$k < p_v$}
      (3,-2);

      % Post2      
      \begin{scope}[shift={(4cm, -3cm)}]

        \node[terminal, label=right:{$p_a + (p_v - k)$}] (a) at (0, 0) {a};
        \node (sg) [above right = 1.3 and 0.1 of a] {};
        \node (sg2) [below right= 1.3 and 0.1 of a] {};

        %Edges
        \draw[dashed] (a) to (sg);
        \draw[dashed] (a) to (sg2);
      \end{scope}

  \end{tikzpicture}
  \caption{Removing a non-terminal with degree 1.}
  \label{fig:red:test:deg1}
\end{figure}

\subsubsection{Terminals of Degree 2}
\subsection{Special Distance Reduction Tests}


\section{Primal Heuristics}

\section{Exact Algorithms}
%%% Local Variables:
%%% TeX-master: "report"
%%% reftex-default-bibliography: ("lit.bib")
%%% End:
