\clearpage
\section{Exact Algorithms}\label{sec:solving:exact}

\subsection{DHEA Solver}
\label{sec:exact:dhea}
The DHEA solver, defined and implemented
by \citet{ljubic2005solving}
(more concisely described in \citet{ljubic2006algorithmic}) finds exact solutions to the PCSTP
by first applying
reduction tests to the input graph (See Section \ref{sec:solving:pre}) and
then transforming the resulting problem into related problem of finding a minimum
aborescence in a directed graph. It is the first example of a procedure designed for
finding provably optimal solutions for the PCSTP.

The resulting SAP is then solved by branch-and-cut using a separation procedure
similar to the one defined in \citet{lucena2004strong}
(Section \ref{sec:lower:gsec}) and a new primal heuristics for
finding good incumbents at every node.

The solver has as a dependency on of the MIP solvers CPLEX or LEDA.
 \paragraph{Reducing the PCSTP to PCSAP}
 Citing stronger LP relaxations, \citet{ljubic2005solving} reduces an input PCSTP instance
 into a related problem in a directed graph which essentially is a constrained version
 of a Prize-Collecting Steiner Aborescence Problem (PCSAP).

 Given an instance of the PCSTP, with graph $G = (V, E, c, p)$ and terminals $N \subset V$,
 we define an instance
 of the PCSAP with directed graph, $G_A = (V_A, A, c')$, where $c'$ define the arc costs,
 by first adding an
 artificial root vector, $r$, to $V$
 $$V_A = \{r\} \cup V\mathnormal{.}$$
 Then we define the arc set by adding bi-directional arcs for each edge in $E$ and
 adding an arc from $r$ to each terminal,
 $$A = \{(i,j) \mid (i,j) \in E \} \cup \{ (j,i) \mid (i,j) \in E \} \cup \{(r, i) \mid i \in N\}\mathnormal{.}$$
 We define arc-costs by subtracting the prize of the arc's endpoint from the original cost of the
 corresponding edge in $G$. Arcs from the root have cost equal to the negative prize of the endpoint vertex. This gives us,
 $$c'_{ij} =
 \begin{cases}
   -p_j & i = r \\
   c_{ij} - p_j & \text{otherwise.}
 \end{cases}
 $$
Figure (\ref{fig:pcstptosap}) shows an example of this kind of reduction.
 \begin{figure}[h]\centering
  \begin{subfigure}[t]{0.47\linewidth}
    
    \begin{tikzpicture}[auto, node distance=1.5 cm]
      %Nodes
  \node[terminal, label={12}] (a) {a};
  \node[steiner] (b) [right=of a] {b};
  \node[steiner] (c) [right=of b] {c};
  \node[terminal, label={10}] (d) [above =of c] {d};
  \node[steiner] (e) [left=of d] {e};
  \node[steiner] (f) [left=of e] {f};
  \node[terminal, label={3}] (g) [above right=0.75 and 1.3 of c] {g};
  % Edges
  \begin{scope}[every edge/.style={draw=black, thick}]
    \draw (a) edge node[below]{4} (b);
    \draw (b) edge node[near start]{5} (d);
    \draw (b) edge node[below]{8} (c);
    \draw (c) edge node{3} (d);
    \draw (c) edge node{2} (g);
    \draw (c) edge node[near start]{5} (e);
    \draw (d) edge node{6} (e);
    \draw (d) edge node{10} (g);
    \draw (e) edge node{1} (f);
  \end{scope}
    \end{tikzpicture}
    \caption{PCSTP Instance (\ref{fig:pcstp:01})}
  \end{subfigure}
  \begin{subfigure}[t]{0.47\linewidth}
    \begin{tikzpicture}[auto, node distance=1.7 cm]
      % Nodes
      \node[terminal] (a) {a};
      \node[steiner] (b) [right=of a] {b};
      \node[steiner] (c) [right=of b] {c};
      \node[terminal] (d) [above =of c] {d};
      \node[steiner] (e) [left=of d] {e};
      \node[steiner] (f) [left=of e] {f};
      \node[terminal] (g) [above right=0.85 and 1.3 of c] {g};
      \node[terminal, root] (r) [below= of g] {r};
      % Edges
      \begin{scope}[every edge/.style={draw=black, thick},
        every node/.style={circle, inner sep=.7}]
        \draw[->] (a) edge[bend right=10] node[below]{4} (b);
        \draw[<-] (a) edge[bend left=10] node{-8} (b);
        \draw[->] (b) edge[bend right=10] node[near start,swap] {-5} (d);
        \draw[<-] (b) edge[bend left=10] node[near start] {5} (d);
        \draw[<->] (b) edge node[below]{8} (c);
        \draw[->] (c) edge[bend right=10] node[swap]{-7} (d);
        \draw[<-] (c) edge[bend left=10] node{3} (d);
        \draw[<-] (c) edge[bend left=10] node{2} (g);
        \draw[->] (c) edge[bend right=10] node[swap]{-1} (g);
        \draw[<->] (c) edge node[near end]{5} (e);
        \draw[->] (d) edge[bend right=10, swap] node{6} (e);
        \draw[<-] (d) edge[bend left=10] node{-4} (e);
        \draw[->] (d) edge[bend right=10] node[swap]{7} (g);
        \draw[<-] (d) edge[bend left=10] node{0} (g);
        \draw[<->] (e) edge node{1} (f);
        \draw[->] (r) edge[bend right=90, looseness=2] node{-10} (d);
        \draw[->] (r) edge node{-3} (g);
        \draw[->] (r) edge[bend left=40] node{-12} (a);
      \end{scope}
    \end{tikzpicture}
    \caption{Instance (\ref{fig:pcstp:01}) as a PCSAP instance. Root node is coloured blue.}
  \end{subfigure}
  \caption{Reduction from PCSTP to SAP.}
  \label{fig:pcstptosap}
\end{figure}

Given a solution, $SA = (V_{SA}, A_{SA}, c')$,
to the transformed problem, we can generate a solution, $T = (V_T, E_T, c)$, to the
original problem as follows.
Take all edges in $G$ which correspond to arcs in $SA$ -- that is, if either of the arcs
$(i, j) \in A$ or $(j, i) \in A$ are members of the solution $A_{SA}$ then we select the edge $(i,j) \in E$.
We select vertices by taking any vertex $v \in V_{SA} \cap V$ which has an in-degree higher than 0 in
$SA$.

However, to ensure that $T$ is indeed a feasible solution to the original problem we
cannot simply find the connected subgraph in $G_A$ which minimises
$$c(SA) = \sum_{a \in A_{SA}} c'_{a} + \sum_{v \in N} p_v\mathnormal{.}$$
Firstly, since all the arcs from the root have negative costs they are trivially part of any minimal arborescence.
As a result, whenever $G$ has more than one terminal, this may result in a disconnected -- hence infeasible --
solution in the original graph.
Furthermore, by representing vertex prizes within the cost of arcs,
 $G_A$  may contain addtional arcs with negative costs. This may result in an
optimal subgraph which is \textit{not} an aborescence as it may contain cycles.

\citet{ljubic2005solving} solve these two issues by adding extra constraints to the problem. Firstly, the
out-degree of the root vector is limited to a single arc, that is $\delta^+(r) = 1$. With this,
we can intepret the selection of an arc in the adjacency of $r$ as an ``initial'' selection,
 of a terminal, or a root,
in the original problem. Enforcing that any solution to the transformed problem must be an aborescence
 is done by constraining the in-degree of any ``selected'' in a feasible solution terminal to one.

 Single arc solutions in $G_A$ will then correspond 1-to-1 with singleton solutions in $G$.
 Connected
 solutions in $G_A$ containing the arc $(r,v)$ will then correspond to a tree in $G$ ``rooted'' in $v$.
 It is worth noting that for such a solution, the arc $(r, v)$ can be replaced with any other arc
 $(r, u), u \in N \cap V_{SA}$ where;
 hence there exists a many-to-one
 relationship between solutions in $G_A$ and solutions in $G$, that is one of each choice of $(r, v)$.
 To ensure that there exists a one-to-one relationship between solutions in $G_{SA}$ and $G$, only the selected 
 terminal
 with the lowest index in a given solution can be connected to the root.
 \begin{figure}[h!]
   \centering
    \begin{tikzpicture}[auto, node distance=1.7 cm]
      % Nodes
      \node[terminal] (a) {a};
      \node[steiner] (b) [right=of a] {b};
      \node[steiner] (c) [right=of b] {c};
      \node[terminal] (d) [above =of c] {d};
      \node[steiner] (e) [left=of d] {e};
      \node[steiner] (f) [left=of e] {f};
      \node[terminal] (g) [above right=0.85 and 1.3 of c] {g};
      \node[terminal, root] (r) [below= of g] {r};
      % Edges
      \begin{scope}[every edge/.style={draw=black, thick},
        every node/.style={circle, inner sep=.7}]
        \draw[->] (a) edge[bend right=10, selected] node[below]{4} (b);
        \draw[<-] (a) edge[bend left=10] node{-8} (b);
        \draw[->] (b) edge[bend right=10, selected] node[near start,swap] {-5} (d);
        \draw[<-] (b) edge[bend left=10] node[near start] {5} (d);
        \draw[<->] (b) edge node[below]{8} (c);
        \draw[->] (c) edge[bend right=10] node[swap]{-7} (d);
        \draw[<-] (c) edge[bend left=10] node{3} (d);
        \draw[<-] (c) edge[bend left=10] node{2} (g);
        \draw[->] (c) edge[bend right=10] node[swap]{-1} (g);
        \draw[<->] (c) edge node[near end]{5} (e);
        \draw[->] (d) edge[bend right=10, swap] node{6} (e);
        \draw[<-] (d) edge[bend left=10] node{-4} (e);
        \draw[->] (d) edge[bend right=10] node[swap]{7} (g);
        \draw[<-] (d) edge[bend left=10] node{0} (g);
        \draw[<->] (e) edge node{1} (f);
        \draw[->] (r) edge[bend right=90, looseness=2] node{-10} (d);
        \draw[->] (r) edge node{-3} (g);
        \draw[->] (r) edge[bend left=40, selected] node{-12} (a);
      \end{scope}
    \end{tikzpicture}
    \caption{Solution to the transformed instance corresponding to the solution in Figure (\ref{fig:pcstp:01:opt}).}
    \label{fig:exact:sap:opt}
  \end{figure}

  With these extra constraints in place, any feasible solution in the transformed, directed graph $G_{A}$ will correspond
  to a feasible solution in the original graph $G$ with the same cost. In other words there exists a bijection between feasible
  solutions to corresponding instances of the constrained SAP and the PCSTP. Figure \ref{fig:exact:sap:opt} shows this for our
   original PCSTP (see Figure \ref{fig:pcstp:01:opt}). This optimal solution has cost
   $$c(SA) = \sum_{ij \in A_{SA}} c'_{ij} + \sum_{v \in V} p_v = (-12 + 4 -5) + (12 + 10 + 3) = 12$$
   which is the same as the corresponding problem to the PCSTP. Note that solution starting with $(r,d)$ would have the same cost,
   $$(-10 + 5 -8) + (12 + 10 + 3) = 12$$
   but is infeasible due to $d > a$.
  \paragraph{ILP Formulation}

Finding such a minimum ``prize-collecting'' aborescence is formalised by \citet{ljubic2005solving}
as the ILP in Fomulation (\ref{form:exact:cut}). 
 \begin{formulation}[h!]
   \begin{subequations}
     \begin{alignat}{3} 
       &\underset{x,y}{\text{minimize}}
       & & \sum_{ij \in A} c'_{ij} x_{ij} +  \sum_{v \in V} p_v  & \\
       & \text{subject to}\quad
       & & \sum_{j \in A} x_{rj} = 1 &&  \label{form:exact:pcsap:root}\\
       &&& \sum_{ji \in A} x_{ji} = y_i  && \forall i \in V_A \label{form:exact:pcsap:prize}\\
       &&& x(\delta^-(S)) \geq y_k && \forall k \in S, S \subseteq V_A \setminus \{r\}
       \label{form:exact:pcsap:conn}\\
       \intertext{ILP:}
       &&& x_a \in \BB  && \forall a \in A \\
       &&& y_v \in \BB  && \forall v \in V_A \\
       \intertext{LP:}
       &&& 0 \leq x_a \leq 1  && \forall a \in A \\
       &&& 0 \leq y_v \leq 1  && \forall v \in V_A \\
       \intertext{To ensure bijection:}
       &&& x_{rj} \leq 1 - y_i && \forall i,j \in V_A \setminus \{r\}; i < j
       \label{form:exact:bijection}\\
       \intertext{Nonterminal Degree:}
       &&& \sum_{j \in A}x_{ij} \leq \sum_{j \in A}x_{ji}
       && \forall i \in V_A \setminus (N \cup \{r\})
       \label{form:exact:strength}\\
       \intertext{Single Orientation:}
       &&& x_{ij} + x_{ji} \leq y_i  && \forall i \in V_A \setminus \{r\}; \forall (i,j) \in A
       \label{form:exact:orientation}
     \end{alignat}\label{form:exact:cut}
   \end{subequations}
   \caption{(CUT-IP) CUT formulation of the constrained PCSAP.}
 \end{formulation}

 As usual, the ILP formulation has two decision vectors, $x$ and $y$, where the former denotes arcs
 in the solution arborescence and the latter which vertices are in the arborescence. Constraints-wise,
 we have
 constraint (\ref{form:exact:pcsap:root}), which ensures that exactly one arc begins in the artificial
 root in any feasible aborescence. Constraint (\ref{form:exact:pcsap:prize}) ensures
 that a vertex can be selected iff it has an incoming arc.
 Constraints (\ref{form:exact:pcsap:conn}), called \textit{connectivity constraints},
 ensures the connectivity of feasible aborescences by requiring that
 each vertex subset containing a selected vertex must have an incoming arc.
 Together, these constraints ensure that any feasible solution is an aborescence rooted
 in $r$ with $\delta^+(r) = 1$.

 It is worth noting that for a given subset $S \subseteq V$, adding constraints of type
 (\ref{form:exact:pcsap:prize}) for each $v \in S$ to a corresponding connectivity constraint
  gives us generalised subtour elimination constraints:
 \begin{align*}
   -x(\delta^-(S)) &\leq -y_k \\
   \sum_{i \in S} \sum_{ji \in A} x_{ji} - x (\delta^-(S)) &\leq \sum_{v \in S} - y_k \\
   \sum_{i \in S} \sum_{j \in S} x_{ij} &\leq \sum_{v \in S} - y_k \mathnormal{.}
 \end{align*}

 Adding only constraints (\ref{form:exact:pcsap:root})-(\ref{form:exact:pcsap:conn})
 has the unfortunate consequence of having multiple 
 PCSAP solutions
 correspond
 to a single PCSTP solution. To achieve a bijection between the PCSAP solutions and the
 PCSTP solutions, constraint (\ref{form:exact:bijection}) allows only the chosen vertex
 with the lowest index to be adjacent to the root vertex.
 
 Additionally, \citet{ljubic2005solving} observe that each nonterminal in an optimal solution
 \textit{must not} be a leaf in the arborescence. Hence, they add constraint
 (\ref{form:exact:strength}) to strengthen the formulation. And finally, they
 also capture the fact that there cannot be anti-parallel arcs in a feasible aborescence in
 constraint (\ref{form:exact:orientation}). While this constraint is captured implicitly
 in the previous constraints and adding them increases the size of the formulation,
 \citet{ljubic2005solving} report improved running times for adding them as it avoids
 having to implicitly separate the in the separation procedure.
 
\paragraph{Branch \& Bound}

For finding lower bounds, the LP relaxation of (\ref{form:exact:cut}) is solved at each node.
Constraints of type (\ref{form:exact:pcsap:conn}) are separated through solving the max flow (min cut)
problem on a support graph (similar to the procedure in Section \ref{sec:lower:gsec}). 
The max flow problem is solved for every terminal $v \in N$ and returns both the minimum cut closest to
the root, and the minimum cut $(S,T)$ ``closest'' to the sink with $v \in T$.
Violated constraints are iteratively found by increasing the
capacity of all edges crossing a min cut to $1$ and rerunning the max flow procedure.

Primal heuristics based on the solution of LP relaxations are used at each node in the branch and bound
tree to discover improved incumbents.

The rest of the branch and bound procedure is handled by an underlying MIP solver (CPLEX or LEDA).
\subsection{SCIP-Jack Solver}
\label{sec:exact:scipj}
The SCIP-Jack solver is a set of plugins for the SCIP general purpose optimisation suite capable
of generating exact solutions to a variety of STP variants in graphs \cite{gamrath2017scip}.

As with the DHEA solver (Section \ref{sec:exact:dhea}), SCIP-Jack solves the PCSTP by first applying
reduction tests (See section \ref{sec:pre:summary-usage} for details) to the input graph before transforming problem
instances to a constrained version of the SAP and solving it with branch and cut.
The main difference here is in that the DHEA solver works with a prize-collecting variant
 of the SAP problem whereas SCIP-Jack works with the regular version of the SAP.

\paragraph{Reduction to SAP}
As with the DHEA solver, SCIP-Jack internally works on the SAP instead of the solving native
versions of the STP variants. Different from the DHEA solver, SCIP-Jack, however, does not
transform the PCSTP into a prize-collecting SAP. This means that the reduction is slightly
different, as the prize-collecting aspect of the PCSTP has to be described in an alternative
way in the SAP.

Given an instance of the PCSTP with graph $G = (V,E,c,p)$ and terminal set $N$, we wish to define
 a directed graph $G_A = (V_A, A, c')$ with a terminal set $N_A$. The first
 step of this reduction is to add a copy of each terminal to the vertex set, $V_A$.
 To do this, we define a
set of auxiliary terminals as
$$N_A = \{v' \mid v \in N\}$$
which are the terminal set for our SAP.
The original set of vertices $V$ is combined with this set and an
artificial root vector to define the vertex-set in the directed graph
$$V_A = V \cup \{r\} \cup N_A\mathnormal{.}$$
We then add antiparallel arcs for each edge in $E$ as well as arcs from the artifical root vertex, $r$,
to each original terminal and auxiliary terminal and arcs from the original terminals to the auxiliary
terminals,
\begin{align*}
A = &\{(i,j) \mid (i,j) \in E \} \cup \{ (j,i) \mid (i,j) \in E \} \:\cup \\
&\{(r, i) \mid i \in N\} \cup \{(r,i) \mid i \in N_A\} \cup \{(i,j) \mid i \in N, j \in N_A\}\mathnormal{.}
\end{align*}
Finally, the arc-costs in the directed graphs are defined as follows,
$$c'_{ij} =
\begin{cases}
  c_{ij} & (i,j) \in E \\
  p_v & i = r, j \in N_A , j = v' \\
  0 & i = r, j \in N \\
  0 & i \in N, j \in N_A
\end{cases}\textnormal{.}$$
Arcs corresponding to edges in the original graph have the same costs in $G_A$, arcs from the root
to vertices in $N_A$ cost the prize of the vertex, arcs from the root to the original terminals
have no cost and the same goes for arcs from the original terminals to the auxiliary terminals.

\begin{figure}[h]\centering
  \begin{subfigure}[t]{0.47\linewidth}
    
    \begin{tikzpicture}[auto, node distance=1.5 cm]
      %Nodes
  \node[terminal, label={12}] (a) {a};
  \node[steiner] (b) [right=of a] {b};
  \node[steiner] (c) [right=of b] {c};
  \node[terminal, label={10}] (d) [above =of c] {d};
  \node[steiner] (e) [left=of d] {e};
  \node[steiner] (f) [left=of e] {f};
  \node[terminal, label={3}] (g) [above right=0.75 and 1.3 of c] {g};
  % Edges
  \begin{scope}[every edge/.style={draw=black, thick}]
    \draw (a) edge node[below]{4} (b);
    \draw (b) edge node[near start]{5} (d);
    \draw (b) edge node[below]{8} (c);
    \draw (c) edge node{3} (d);
    \draw (c) edge node{2} (g);
    \draw (c) edge node[near start]{5} (e);
    \draw (d) edge node{6} (e);
    \draw (d) edge node{10} (g);
    \draw (e) edge node{1} (f);
  \end{scope}
    \end{tikzpicture}
    \caption{PCSTP Instance (\ref{fig:pcstp:01})}
  \end{subfigure}
  \begin{subfigure}[t]{0.47\linewidth}
    \begin{tikzpicture}[auto, node distance=1.7 cm]
      % Nodes
      \node[steiner] (a) {a};
      \node[terminal] (aa) [below=1cm of a] {a'};
      \node[steiner] (b) [right=of a] {b};
      \node[steiner] (c) [right=of b] {c};
      \node[steiner] (d) [above =of c] {d};
      \node[terminal] (dd) [above= 0.8cm of d] {d'};
      \node[steiner] (e) [left=of d] {e};
      \node[steiner] (f) [left=of e] {f};
      \node[steiner] (g) [above right=0.85 and 1.3 of c] {g};
      \node[terminal] (gg) [below= 0.8cm of g] {g'};
      \node[terminal, root] (r) [right= of g] {r};
      % Edges
      \begin{scope}[every edge/.style={draw=black, thick},
        every node/.style={circle, inner sep=.7}]
        \draw[<->] (a) edge node[below]{4} (b);
        \draw[->] (a) edge node {0} (aa);
        \draw[<->] (b) edge node[near start] {5} (d);
        \draw[<->] (b) edge node[below]{8} (c);
        \draw[<->] (c) edge node{3} (d);
        \draw[<->] (c) edge node{2} (g);
        \draw[<->] (c) edge node[near end]{5} (e);
        \draw[->] (d) edge node {0} (dd);
        \draw[<->] (d) edge node{6} (e);
        \draw[<->] (d) edge node[swap]{7} (g);
        \draw[<->] (e) edge node{1} (f);
        \draw[->] (g) edge node{0} (gg);
        \draw[->] (r) edge[bend right=40] node{0} (d);
        \draw[->] (r) edge[bend right=40] node{10} (dd);
        \draw[->] (r) edge node{0} (g);
        \draw[->] (r) edge node{3} (gg);
        \draw[->] (r) edge[bend left=45, looseness=1.0] node{0} (a);
        \draw[->] (r) edge[bend left=50, looseness=0.7] node{12} (aa);
      \end{scope}
    \end{tikzpicture}
    \caption{Instance (\ref{fig:pcstp:01}) as a SAP instance. Root node is coloured blue.}
  \end{subfigure}
  \caption{Reduction from PCSTP to SAP.}
  \label{fig:scip:pcstptosap}
\end{figure}

Figure \ref{fig:scip:pcstptosap} shows this transfomation. Intuitively, we see that
picking an arc which goes from the root to a vertex in $v' \in N_A$
corresponds to not picking the terminal $v$ in the original graph. Selecting an arc from the root
to some vertex $v \in N$ (and the arc $(v, v')$) then corresponds to selecting $v$
in the original graph.. Selecting any arcs between two vertices which are found in $G$ corresponds
directly to selecting the corresponding edge in $G$.

\begin{figure}[t]
  \centering
    \begin{tikzpicture}[auto, node distance=1.7 cm]
      % Nodes
      \node[steiner] (a) {a};
      \node[terminal] (aa) [below=1cm of a] {a'};
      \node[steiner] (b) [right=of a] {b};
      \node[steiner] (c) [right=of b] {c};
      \node[steiner] (d) [above =of c] {d};
      \node[terminal] (dd) [above= 0.8cm of d] {d'};
      \node[steiner] (e) [left=of d] {e};
      \node[steiner] (f) [left=of e] {f};
      \node[steiner] (g) [above right=0.85 and 1.3 of c] {g};
      \node[terminal] (gg) [below= 0.8cm of g] {g'};
      \node[terminal, root] (r) [right= of g] {r};
      % Edges
      \begin{scope}[every edge/.style={draw=black, thick},
        every node/.style={circle, inner sep=.7}]
        \draw[<-] (a) edge (b);
        \draw[->] (a) edge[selected] node[below]{4} (b);
        \draw[->] (a) edge[selected] node {0} (aa);
        \draw[<-] (b) edge (d);
        \draw[->] (b) edge[selected] node[near start] {5} (d);
        \draw[<->] (b) edge node[below]{8} (c);
        \draw[<->] (c) edge node{3} (d);
        \draw[<->] (c) edge node{2} (g);
        \draw[<->] (c) edge node[near end]{5} (e);
        \draw[->] (d) edge [selected] node {0} (dd);
        \draw[<->] (d) edge node{6} (e);
        \draw[<->] (d) edge node[swap]{7} (g);
        \draw[<->] (e) edge node{1} (f);
        \draw[->] (g) edge node{0} (gg);
        \draw[->] (r) edge[bend right=40] node{0} (d);
        \draw[->] (r) edge[bend right=40] node{10} (dd);
        \draw[->] (r) edge node{0} (g);
        \draw[->] (r) edge[selected] node{3} (gg);
        \draw[->] (r) edge[bend left=45, looseness=1.0, selected] node{0} (a);
        \draw[->] (r) edge[bend left=50, looseness=0.7] node{12} (aa);
      \end{scope}
    \end{tikzpicture}
    \caption{Optimal solution to Figure (\ref{fig:exact:sap:opt}) corresponding to solution (\ref{fig:pcstp:01:opt}).}
    \label{fig:scip:sap:opt}
  \end{figure}

  Figure (\ref{fig:scip:sap:opt}) shows the optimal solution in Figure (\ref{fig:pcstp:01:opt}) in a transformed graph.
  Here we see how terminal $g$ not being selected is represented by selecting the arc $(r, g')$, incurring its prize as
  a penalty. The solution has total cost,
  $$c(SA) = 0 + 0 + 4 + 5 + 0 + 3 = 12$$
  which is the same as the corresponding solution in the optimal graph.

 \paragraph{ILP Formulation}
 \cite{gamrath2017scip} define the ILP formulation of the constrained SAP in Formulation (\ref{form:sap:flow})
 named \textit{flow-balance directed cut formulation}. Constraint (\ref{form:exact:sap:cut}) ensures that there exists a
  path from the root to any terminal by ensuring that any
 cut in the graph which does not contain all terminals must have outgoing arcs. Constraint (\ref{form:exact:sap:cut})
 ensures that all terminals have exactly one incoming arc, nonterminals have at most one incoming arc and the root has
 no incoming arcs. Constraint (\ref{form:exact:sap:indeg}) ensures that nonterminals cannot be leafs in a solution.
 Constraint (\ref{form:exact:sap:deg}) ensures that the solution is connected by ensuring that an arc can only be selected
 if its starting point is pointed to by a selected arc.

 Finally, as with the DHEA solver constraint (\ref{form:sap:root})
 ensures that exact one ``free'' arc from the root can be selected, and (\ref{form:sap:bijection}) creates a bijection
  between solutions in the SAP and PCSTP.
 \begin{formulation}[h!]
   \begin{subequations}
     \begin{alignat}{3} 
       &\underset{x}{\text{minimize}}
       & & \sum_{ij \in A} c'_{ij} x_{ij} & \\
       & \text{subject to}\quad
       && \sum_{v \in \delta^+(r), c_{rv} = 0} x_{rv} = 1 && 
       \label{form:sap:root}\\
       &&& x(\delta^+(S)) \geq 1 && \forall S \subset V, r \in S, N \not\subset S
       \label{form:exact:sap:cut}\\
       &&& x(\delta^-(v))
       \begin{cases}
         = 0 & v = r \\
         = 1 & v \in N \\
         \leq 1 & v \in V_A \setminus (N \cup \{r\})
       \end{cases}
       && \forall v \in V_A \label{form:exact:sap:indeg}\\
       &&& x(\delta^-(v)) \leq x(\delta^+(v)) && \forall v \in V \setminus N
       \label{form:exact:sap:deg}\\
       &&& x(\delta^-(v)) \geq x_{vu} && \forall v \in V \setminus N, \forall u \in \delta^+(v)
       \label{form:exact:sap:conn}\\
       \intertext{ILP:}
       &&& x_a \in \BB  && \forall a \in A \\
       \intertext{LP:}
       &&& 0 \leq x_a \leq 1  && \forall a \in A \\
       \intertext{To ensure bijection:}
       &&& \sum_{i \in \delta^-(j)} x_{ij} + x_{rv}
       && \forall v = 1,...,|V_A|, j = 1,...,i-1
       \label{form:sap:bijection}
     \end{alignat}\label{form:sap:flow}
   \end{subequations}
   \caption{(FLW-IP) Flow-balance directed cut formulation of the constrained SAP.}
 \end{formulation}

 Given a solution to the an instance of the constrained SAP as defined here, a solution to the PCSTP can then
 be obtained by taking an edge for all arcs in the solution for which both endpoints are vertices
 in the original graph. The tree induced by this set of edges then is the corresponding solution in the original
 PCSTP instance.
 
  \paragraph{Branch \& Bound}
  While node selection is left to the underlying MIP solver, SCIP-Jack branches on vertices in the following way. For each node in the
  b\&b tree with optimal solution $x^*$ to the LP relaxation, the nonterminal, $v \in V_A \setminus N_A$, which minimises
  $$v_b = \argmin_{v \in V_A \setminus N_A} \left| \sum_{u \in \delta^-(v)} x^*_{uv} - 0.5 \right|$$
  is added to the set of terminals in one child of the b\&b node and excluded completely in the other.
  \cite{gamrath2017scip} claim that this gives stronger performance than
  to branching on single arcs.

  Futhermore, SCIP-Jack contains an array of heuristics for the SAP which we will not delve further into.

\subsection{Difference in Approach}

% Both use SAP transformations
% SCIP uses an actual SAP instead of the PCSAPish favoured by Ljubic

% Both make use of multipurpose solvers
% Both use the max flow separation of the GSECs
% Both heavily inspired by STP research


%%% Local Variables:
%%% TeX-master: "report"
%%% reftex-default-bibliography: ("lit.bib")
%%% End:
