 \section{Preprocessing}
 \label{sec:solving:pre}
 Applying preprocessing routines to heavily reduce input graphs is a common technique which has been proven successful in many cases,
 both for instances of the \gls{stp}
\citep{koch1998solving}
and instances of the \gls{pcstp}
\citep{lucena2004strong,canuto2001local,ljubic2005solving, gamrath2017scip}. %TODO: MORE
These routines make use of invariants in the problem to make transformations of the
input graph to reduce its size without removing optimal solutions.

A set of common preprocessing routines are applied in different manners in literature.
The way in which this is done is mainly differentiated by:
\begin{enumerate}[label=\alph*)]
\item which routines to apply,
\item in which order, and
\item when to recursively apply routines and how many times.
\end{enumerate}
Preprocessing routines can be very effective. For example, the preprocessing routine
presented by \citet{koch1998solving} for the \gls{stp} removes
up to 98\% of edges in some instances. In other words, making good use of preprocessing
routines, has been shown to increase the size of solvable instances by up to 50 times in some
cases.

In this section, we first give a full overview of
existing preprocessing methods for the \gls{pcstp},
 including proofs of validity,
 and then, in the end of the section,
 summarise their usage in recent literature.

 In the following, we denote any edge or vertex (or part of a graph) as
 \textit{redundant} if there exists an optimal solution
 which does not contain it.
  Additionally, we say that a 
  transformation of a problem instance is \textit{valid} when it produces an
  equivalent problem where all optimal solutions map back to optimal solutions
  in the original problem.
  
  Thus, making a series of valid transformation of the input graph may actually
  hide optimal solutions, but will leave at least one optimal solution
  which will map directly to an optimal solution in the untouched input graph.
 \subsection{Local Reduction Tests}\label{sec:pre:local}
 The first type of reductions we look at, we denote as \textit{local} reduction tests.
 These are tests which at most require knowledge
 of the neighbourhood of a couple of vertexes in the graph
 and static global information
 such as maximum prize. As such, the tests
 presented in this section are generally of low computational cost,
 allowing for testing a full graph in linear time.

 Note that the first four test we present (i.e. NTD1, NTD2, TD1, TD2) are known collectively
 in recent literature as \textit{Degree Tests}, for example by \citet{rehfeldt2016reduction}.
\paragraph{Non-Terminals of Degree 1 (NTD1)}\label{sec:red:test:deg1}
Let $G = (V, E, c, p)$ be a \gls{pcstp} instance and let $v \in V$ be any non-terminal of degree 1, then
clearly
---since edges have positive weights---
$v$ can not be part of an optimal solution.
Thus, $v$ is redundant, and
 all vertices in the set
 \[\{v \mid v \in V \setminus N \wedge |\delta(v)| = 1\}\]
 are redundant. Hence it is valid to remove them from $G$ along with their adjacent edges.

\begin{figure}[h]\centering
    \begin{tikzpicture}[auto, node distance=1.3 cm]
      % Pre
      \begin{scope}[shift={(-0.5,0)}]
        \node[terminal] (a) at (0, 0) {a};
        \node[steiner] (b) [left=of a] {v};
        \node (sg) [above right = 1.3 and 0.1 of a] {};
        \node (sg2) [below right= 1.3 and 0.1 of a] {};

        %Edges
        \draw (a) edge node {$k$} (b);
        \draw[dashed] (a) to (sg);
        \draw[dashed] (a) to (sg2);
      \end{scope}

      \draw [->,decorate, thick,
      decoration={snake,amplitude=.4mm,segment length=2mm,post length=1mm}]
      (1.0,0) -- (3,0);
      % Post      
      \begin{scope}[shift={(4cm, 0)}]

        \node[terminal] (a) at (0, 0) {a};
        \node (sg) [above right = 1.3 and 0.1 of a] {};
        \node (sg2) [below right= 1.3 and 0.1 of a] {};

        %Edges
        \draw[dashed] (a) to (sg);
        \draw[dashed] (a) to (sg2);
      \end{scope}

  \end{tikzpicture}
  \caption{Removing a non-terminal of degree 1.}\label{fig:red:test:deg1}
\end{figure}

 While this reduction test is originally stated for the \gls{stp} \citep{hwang1992steiner}, it is also applicable to the \gls{pcstp}. Figure~(\ref{fig:red:test:deg1})
  shows an example of removing a degree 1 non-terminal.

  \paragraph{Non-Terminals of Degree 2 (NTD2)}
    \label{fig:red:test2}
    Similarly, let $G = (V,E,c,p)$ be an instance of the \gls{pcstp}, let $v \in G$ be a non-terminal of degree $|\delta(v)| = 2$, and let
    $u$ and $w$ be the two vertices adjacent to $v$. Then we can obtain a reduced, equivalent graph,
    \[G' = (V', E', c', p)\]
    where
    \[V' = V - v \mathnormal{,}\]
    \[E' = (E \setminus \{(u,v),(w,v)\}) \cup \{(u,w)\}\mathnormal{,}\]
    and $c' = c$ except for
    $$c'_{uw} =
    \begin{cases}
      \min(c_{uw}, c_{uv} + c_{vw}) & (u,w) \in E\mathnormal{,} \\
      c_{uv} + c_{vw} & \text{otherwise.}
    \end{cases}$$

    In other words, if $c_{uv} + c_{vw} <  c_{uw}$ then $(u,w)$ is redundant, can be removed,
    and the path from $u$ along $v$ to $w$
    can be contracted into a single edge.
    Otherwise $v$ and its edges are redundant and can be removed. Figure~(\ref{fig:pre:ntd2})
    shows an example this reduction test.

\begin{figure}[h!]\centering
    \begin{tikzpicture}[auto, node distance=1.3 cm]
      % Pre
      \begin{scope}
        \node[terminal] (a) at (0, -0.65) {u};
        \node[terminal] (b) [above=of a] {w};
        \node[steiner] (c) [above left= 0.65 and 1.3 of a] {v};
        \node (sg) [above right = 1.3 and 0.1 of b] {};
        \node (sg2) [below right= 1.3 and 0.1 of a] {};

        %Edges
        \draw (a) edge node {$k$} (b);
        \draw (a) edge node {$c_1$} (c);
        \draw (b) edge node[swap] {$c_2$} (c);
        \draw[dashed] (b) to (sg);
        \draw[dashed] (a) to (sg2);
      \end{scope}

      \draw [->,decorate, thick,
      decoration={snake,amplitude=.4mm,segment length=2mm,post length=1mm}]
      (1,0) -- (3.25,0);
      % Post      
      \begin{scope}[shift={(4.5cm, 0)}]
        \node[terminal] (a) at (0, -0.65) {u};
        \node[terminal] (b) [above=of a] {w};
        \node (sg) [above right = 1.3 and 0.1 of b] {};
        \node (sg2) [below right= 1.3 and 0.1 of a] {};

        %Edges
        \draw (a) edge node[swap] {$\min(k, c_1 + c_2)$} (b);
        \draw[dashed] (b) to (sg);
        \draw[dashed] (a) to (sg2);
      \end{scope}

  \end{tikzpicture}
  \caption{Removing a non-terminal of degree 2.}  \label{fig:pre:ntd2}
\end{figure}

This test is another example of a reduction test for the \gls{stp} which can by directly applied to
 the \gls{pcstp}.
 \paragraph{Terminals of Degree 1 (TD1)}\label{sec:pre:td1}
 When we have a terminal, $u$, of degree one, we can make the following two observations:
 \begin{enumerate}
 \item If the cost of the edge adjacent to $u$ is higher than $p_u$, then it is
   never worth it to connect $u$ to a solution, hence the adjacent edge is redundant, and
 \item if the above is the case and there exists
   a vertex with at least as high prize as $u$, the single vertex solution containing $u$
   can \textit{at best} share objective value with other optimal solutions. Thus $u$ is redundant.
 \end{enumerate}
 These observations were stated first by \citet{uchoa2006reduction} ---
 however without making the second observation,
 making the redundancy of $u$ potentially invalid
 --- and later corrected and restated by \citet{rehfeldt2016reduction} as a combined test.

 We state here only the first part here as the TD1 test to allow for the disconnection of vertices with maximal prize.
 \begin{theorem}[Terminals of Degree 1 Test]
   Let $u$ be a terminal in $G$ of degree of 1 with adjacent edge $(u, v)$. If
   $p_u < c_{uv}$ then $(u,v)$ is redundant.
 \end{theorem}

 Applying this definition of TD1 in conjunction with the UDV test (defined in Section~\ref{sec:pre:udv})
 allows for the potential removal of terminals of degree one (see Figure~\ref{fig:red:test:deg1}), giving the
  test defined by \citet{uchoa2006reduction} and \citet{rehfeldt2016reduction}.

\begin{figure}[h!]\centering
    \begin{tikzpicture}[auto, node distance=1.3 cm]
      % Pre
      \begin{scope}[shift={(-4,0)}]
        \node[terminal, label={$p_a$}] (a) at (0, 0) {v};
        \node[terminal, label={$p_v$}] (b) [left=of a] {u};
        \node (sg) [above right = 1.3 and 0.1 of a] {};
        \node (sg2) [below right= 1.3 and 0.1 of a] {};

        %Edges
        \draw (a) edge node {$k$} (b);
        \draw[dashed] (a) to (sg);
        \draw[dashed] (a) to (sg2);
      \end{scope}

      \draw [->,decorate, thick,
      decoration={snake,amplitude=.4mm,segment length=2mm,post length=1mm}]
      (-3,1) -- node [above=1mm,midway,text width=3cm, sloped, align=center] {$k\geq p_v$}
      (-1,2.3);
      \draw [->,decorate, thick,
      decoration={snake,amplitude=.4mm,segment length=2mm,post length=1mm}]
      (-3,-1) -- node [above=1mm,midway,text width=3cm, sloped, align=center] {$k < p_v$}
      (-1,-2.3);

      % Post      

      \begin{scope}[shift={(1.5cm, 0)}]

        \begin{scope}[shift={(0, 3cm)}]
          
          \node[terminal, label=right:{$p_a$}] (a) at (0, 0) {v};
          \node[terminal, label={$p_v$}] (b) [left=of a] {u};
        
          \node (sg) [above right = 1.3 and 0.1 of a] {};
          \node (sg2) [below right= 1.3 and 0.1 of a] {};

          % Edges
          \draw[dashed] (a) to (sg);
          \draw[dashed] (a) to (sg2);
        \end{scope}


        % Post2      
        \begin{scope}[shift={(0, -3cm)}]
          \node[terminal, label=right:{$p_a + (p_v - k)$}] (a) at (0, 0) {v};
          \node[terminal, label={$p_v$}] (b) [left=of a] {u};
          
          \node (sg) [above right = 1.3 and 0.1 of a] {};
          \node (sg2) [below right= 1.3 and 0.1 of a] {};
          
          % Edges
          \draw[dashed] (a) to (sg);
          \draw[dashed] (a) to (sg2);
        \end{scope}
      \end{scope}

      \draw [->,decorate, thick,
      decoration={snake,amplitude=.4mm,segment length=2mm,post length=1mm}]
      (4,3) -- node [above=1mm,midway,text width=3cm, sloped, align=center] {$\exists p_u \geq p_v$}
      (5,3);

      \draw [->,decorate, thick,
      decoration={snake,amplitude=.4mm,segment length=2mm,post length=1mm}]
      (4,-3) -- node [above=1mm,midway,text width=3cm, sloped, align=center] {$\exists p_u \geq p_v$}
      (5,-3);

      \begin{scope}[shift={(6.5cm, 0)}]

        \begin{scope}[shift={(0, 3cm)}]
          
          \node[terminal, label=right:{$p_a$}] (a) at (0, 0) {v};
        
          \node (sg) [above right = 1.3 and 0.1 of a] {};
          \node (sg2) [below right= 1.3 and 0.1 of a] {};

          % Edges
          \draw[dashed] (a) to (sg);
          \draw[dashed] (a) to (sg2);
        \end{scope}


        % Post3      
        \begin{scope}[shift={(0, -3cm)}]
          \node[terminal, label=right:{$p_a + (p_v - k)$}] (a) at (0, 0) {v};
          
          \node (sg) [above right = 1.3 and 0.1 of a] {};
          \node (sg2) [below right= 1.3 and 0.1 of a] {};
          
          % Edges
          \draw[dashed] (a) to (sg);
          \draw[dashed] (a) to (sg2);
        \end{scope}
    \end{scope}

  \end{tikzpicture}
  
  \caption{Removing a terminal of degree 1 (TD1 + UDV).}
  \label{fig:red:test:deg1}
\end{figure}

\paragraph{Terminals of Degree 2 (TD2)} 
Similarly to TD1, we can disconnect terminals of degree 2 where both edges have costs higher than the prize
 of the terminal. Formally we state this as,
 \begin{theorem}[Terminals of Degree 2 Test]
   Let $u$ be a vertex of degree 2, such that vertices $v$ and $w$ are adjacent to $u$. Suppose that
   \[p_u \leq \min(c_{uv}, c_{uw})\]
   then any optimal solution must \textit{either} contain both edges or neither.
   Thus it is valid to replace the edges $(u,v)$ and $(u,w)$ with a single edge $(w,v)$ with cost
   \[c_{wv} = c_{uv} + c_{uw} - p_u\mathnormal{.}\]
 \end{theorem}

\begin{figure}[h!]\centering
    \begin{tikzpicture}[auto, node distance=1.3 cm]
      % Pre
      \begin{scope}
        \node[terminal] (a) at (0, -0.65) {v};
        \node[terminal] (b) [above=of a] {w};
        \node[terminal] (c) [above left= 0.65 and 1.3 of a] {u};
        \node (sg) [above right = 1.3 and 0.1 of b] {};
        \node (sg2) [below right= 1.3 and 0.1 of a] {};

        %Edges
        \draw (a) edge node {$k$} (b);
        \draw (a) edge node {$c_1$} (c);
        \draw (b) edge node[swap] {$c_2$} (c);
        \draw[dashed] (b) to (sg);
        \draw[dashed] (a) to (sg2);
      \end{scope}

      \draw [->,decorate, thick,
      decoration={snake,amplitude=.4mm,segment length=2mm,post length=1mm}]
      (1,0) -- node {$p_u \leq \min(c_1, c_2)$}
      (3.25,0);
      % Post      
      \begin{scope}[shift={(6.25cm, 0)}]
        \node[terminal] (a) at (0, -0.65) {v};
        \node[terminal] (b) [above=of a] {w};
        \node[terminal] (c) [above left= 0.65 and 1.3 of a] {u};
        \node (sg) [above right = 1.3 and 0.1 of b] {};
        \node (sg2) [below right= 1.3 and 0.1 of a] {};

        %Edges
        \draw (a) edge node[swap] {$\min(k, c_1 + c_2 - p_u)$} (b);
        \draw[dashed] (b) to (sg);
        \draw[dashed] (a) to (sg2);
      \end{scope}

      \draw [->,decorate, thick,
      decoration={snake,amplitude=.4mm,segment length=2mm,post length=1mm}]
      (9,0) -- node {$\exists p_v \geq p_u$}
      (10,0);

      \begin{scope}[shift={(11cm, 0)}]
        \node[terminal] (a) at (0, -0.65) {v};
        \node[terminal] (b) [above=of a] {w};
        \node (sg) [above right = 1.3 and 0.1 of b] {};
        \node (sg2) [below right= 1.3 and 0.1 of a] {};

        %Edges
        \draw (a) edge node[swap] {$\min(k, c_1 + c_2 - p_u)$} (b);
        \draw[dashed] (b) to (sg);
        \draw[dashed] (a) to (sg2);
      \end{scope}

  \end{tikzpicture}
  \caption{Removing a terminal of degree 2 (TD2 + UDV).}\label{fig:red:td2}
\end{figure}

\paragraph{Unconnected Dominated Vertex (UDV)}\label{sec:pre:udv}
Proposed by \citet{rehfeldt2016reduction},
the \textit{unconnected dominated vertex test} reduces the
graph by removing any subgraph which contains at most one terminal which has less than maximum prize.
Stated formally as in the following theorem, which we will not prove.
\begin{theorem}[Unconnected Dominated Vertex Test]
  Consider a connected subgraph $S = (V_S, E_S, c, p)$ of $G$. Let $N_S = \{v \in V_s \mid p_v > 0\}$
  be the set of terminals in $S$. Then $S$ is redundant and removing it is valid if either
  \begin{enumerate}
  \item $N = \emptyset$, or
  \item $N = \{u\}$ and $p_u \leq \max_{v \in V \setminus \{u\}} p_v$.
  \end{enumerate}
\end{theorem}

This test can easily be applied as a special case for subgraphs consisting of a single,
 unconnected vertex.
 Figure~(\ref{fig:red:test:deg1}) shows the application
 of TD1 followed directly by UDV.\@
\paragraph{Minimum Adjacency (MA)}
Again defined by \citet{duin1989reduction} for the \gls{stp}
but also applicable for the \gls{pcstp}, is
the \textit{Minimum Adjacency test}
which also known as the \textit{$V \setminus K$ test}.
It is a reduction test which contracts any adjacent
terminals which are connected by an edge with lower cost than either of their prizes.
 This is shown in Figure~(\ref{fig:red:test:ma}).
 
\begin{figure}[h!]\centering
    \begin{tikzpicture}[auto, node distance=1.3 cm]
      % Pre
      \begin{scope}[shift={(-0.5,0)}]
        \node[terminal, label={$p_u$}] (u) at (0, 0) {u};
        \node[terminal, label={$p_v$}] (v) [left=of u] {v};
        \node (sgu) [above right = 1.3 and 0.1 of u] {};
        \node (sgu2) [below right= 1.3 and 0.1 of u] {};

        \node (sgv) [above left = 1.3 and 0.1 of v] {};
        \node (sgv2) [below left= 1.3 and 0.1 of v] {};

        %Edges
        \draw (u) edge node {$k$} (v);
        \draw[dashed] (u) to (sgu);
        \draw[dashed] (u) to (sgu2);
        \draw[dashed] (v) to (sgv);
        \draw[dashed] (v) to (sgv2);
      \end{scope}

      \draw [->,decorate, thick,
      decoration={snake,amplitude=.4mm,segment length=2mm,post length=1mm}]
      (1.0,0) -- node [above=1mm,midway,text width=3cm, sloped, align=center] {$k \leq \min(p_v, p_u)$}
      (3,0);
      % Post      
      \begin{scope}[shift={(6cm, 0cm)}]
        \node[terminal, label={$p_u + p_v - k$}] (u) at (0, 0) {uv};
        \node (sgu) [above right = 1. and 2.0 of u] {};
        \node (sgu2) [below right= 1. and 2.0 of u] {};
        \node (sgv) [above left = 1. and 2.0 of u] {};
        \node (sgv2) [below left= 1. and 2.0 of u] {};

        %Edges
        \draw[dashed] (u) to (sgu);
        \draw[dashed] (u) to (sgu2);
        \draw[dashed] (u) to (sgv);
        \draw[dashed] (u) to (sgv2);
      \end{scope}

  \end{tikzpicture}
  \caption{Minimum adjacency test.}
  \label{fig:red:test:ma}
\end{figure}

\begin{theorem}[Minimum Adjacency]
  Let $u$ and $v$ be adjacent terminals in $G$. If we have
  \[c_{uv} \leq \min(p_u, p_v)\]
  and
  \[c_{uv} = \min_{(u, w) \in \delta(u)}c_{uw}\]
  then it is valid to contract $u$ and $v$.
\end{theorem}
\begin{proof}
  It can be shown that any solution $T$ for the \gls{pcstp} problem defined by a graph $G$
  which contains $u$ but not $(u,v)$ can
  be transformed into a solution $T'$ which contains $(u,v)$ where $c(T') \leq c(T)$.

  \paragraph{Case 1: ($v \in V_T$)}
  Let $(u, w) \in E_T$ be the first edge in the simple path from $u$ to $v$ in $T$
  ($T$ is a tree, so there is only
  one such path). By assumption, we
  have that $c_{uv} \leq c_{uw}$, and the tree $T'$ constructed by removing $(u,w)$ from $T$ and
  adding $(u,v)$ has cost
  \[c(T') = c(T) - c_{uw} + c_{uv} \leq T\mathnormal{.}\]
  \paragraph{Case 2: ($v  \not\in V_T$)}
  Let $T'$ be the tree obtained by adding $(u,v)$ to $T$. As per the assumption that $\min(p_u, p_v) \geq c_{uv}$,
  we have that $T'$ has cost,
  \[c(T') = c(T) + c_{uv} - p_{uv} \leq c(T)\mathnormal{.}\]
\end{proof}


\subsection{Steiner Distance Reduction Tests}\label{sec:sd-red-test}
% \[d(u,v) = \min \{ c(P) \mid P \in \mathcal{P}_{uv}\}\]
More complex tests can be described in terms
of the concepts Steiner distance and Bottleneck distance. These were originally stated for
the \gls{stp} by, amongst others, \citet{duin1989edge,duin1989reduction} and later
adapted for the \gls{pcstp} by \citet{uchoa2006reduction}.

Before describing the reduction tests, we must make some definitions.
If $P = (\ldots, u, \ldots, w, \ldots)$ is a simple path in $G$, then $P_{uw}$ is
the subpath of $P$ which starts at vertex $u$ and ends in vertex $w$,
that is $P_{u,w} = (u, \ldots,w)$. Then we define the \textit{Steiner distance} of
$P_{uw}$ as the cost of edges in $P_{uw}$ subtracted by the prize collected along
 the way,
 $$sd(P_{uw}) = \sum_{(i,j) \in E(P_{u,w})} c_{i,j} -
 \sum_{v \in V(P) \setminus \{u,w\}} p_{v}\mathnormal{.}$$
 We then denote the Steiner distance of a simple path, $P$, as the maximal Steiner
 distance found among subpaths of $P$,
 \[sd(P) = \max_{u,w \in P} sd(P_{uw})\mathnormal{.}\]
 Let $\mathcal{P}_{uw}$ be the set of all simple paths
 connecting vertices $u$ and $w$ in
 $G$, then we denote the \textit{bottleneck distance} between $u$ and $w$ as the minimal Steiner
  distance among paths in $\mathcal{P}_{uw}$,
  \[B(u,w) = \min_{P \in  \mathcal{P}_{uw}} sd(P)\mathnormal{.}\]
  The bottleneck distance is a measure of the worst case \textit{opportunity cost}
 of bridging some cut $(S, T)$
 where $u \in S$ and $v \in T$.

 Finally, we denote ${B(u,w)}^{-e}$
 as the minimum Steiner distance of a path between $u$ and $w$ which does
   not include $e$, that is
  \[{B(u,w)}^{-e} = \min_{P \in  \mathcal{P}_{uw}, e \not \in P} sd(P)\mathnormal{.}\]

 \citet{uchoa2006reduction} shows that calculating the bottleneck distance
 is an NP hard problem by reduction from the Hamiltonian Path problem, suggesting that
 exact bottleneck distances are infeasible to calculate. However, \citet{uchoa2006reduction}
 also claims that existing heuristics are fast and give strong upper bounds.

It is worth noting that due to
\begin{enumerate*}[label={\alph*)}]
\item the relatively new proposal of the Steiner/bottleneck distances by \citet{uchoa2006reduction} for the \gls{pcstp}, and
\item their computational complexities,
\end{enumerate*}
the reduction tests stated in this section have been applied in literature
using ordinary distance (i.e.\ ignoring prizes) and shortests paths
instead of the Steiner and bottleneck distances. This corresponds to the direct application of
reduction tests
proposed for the \gls{stp} by \citet{duin1989edge,duin1989reduction}. Since this results in
working with
upper bounds on the bottleneck distance, the original
tests are still valid for the \gls{pcstp},
but are weaker than the natively tests proposed by \citet{uchoa2006reduction}.

When we refer to
\textit{shortest path relaxations} of the following tests, we refer to application of the test
 as described above.

 The tests shown below, while computationally more costly than the local tests, can potentially
 ``break'' a graph, which previously could not be reduced further with basic tests, in such a
 way that the basic tests can be applied.
 \paragraph{Special Distance (SD)}\label{sec:red:test:sd}
 Using the \gls{pcstp} version of the Steiner and bottleneck distances,
 \citet{uchoa2006reduction} proves
  the validity of the more general \textit{special distance test}.
 \begin{theorem}[Special Distance Test]
 Consider any edge $(u,v) \in E$. If we have
 \[{B(u,v)}^{-(u,v)} \leq c_{uv}\]
 then $(u,v)$ is redundant.
\end{theorem}
 \begin{proof}
   Let $T  \subseteq G$ be an optimal solution to the \gls{pcstp} in graph $G$
   where we have
   \[B(u,v)^{-(u,v)} \leq c_{uv}\]
   for some edge $(u,v) \in E_T$, and let $(T_1, T_2)$ be the cut bridged
   by $(u,v)$ in $T$.

   Consider then the simple path $P \in \mathcal{P}_{uv}$ in $G$ from $u$ to $v$ which doesn't
    contain $(u,v)$ and has
   \[sd(P) = {B(u,v)}^{-(u,v)}\mathnormal{.}\]

\begin{figure}[h!]\centering
    \begin{tikzpicture}[auto, node distance=1.3 cm]
      % Pre
      \begin{scope}[]
        \path[fill=black!5,use Hobby shortcut,closed=true]
        (-2.5, -2) .. (0.3,-1) .. (.5,1) .. (.5,2)  .. (-2.5,3.5);
        \path[fill=black!5,use Hobby shortcut,closed=true]
        (2.5, -2) .. (1.3,-1) .. (1.5,1) .. (1.5,2)  .. (2.5,3.5);

        \node[terminal] (u) at(0, 0) {u};
        \node[subgraph, above left= 0.05cm and 1.4cm of u] {$T_1$};
        \node[steiner, densely dashed] (w1) [above=of u] {w};
        
        \node[terminal] (v) [right= of u] {v};
        \node[subgraph,above right= 0.05 and 1.4cm  of v] {$T_2$};
        \node[steiner, densely dashed] (w2) [above=of v] {z};
        \node (sga) [left= of u] {};
        \node (sga2) [below left= 1.3 and 0.1 of u] {};

        \node (sgb) [right=  of v] {};
        \node (sgb2) [below right= 1.3 and 0.1 of v] {};

        
        
        %Edges
        \draw[selected] (u) edge node {$k$} (v);
        \draw[dashed, selected] (u) to (sga);
        \draw[dashed, selected] (u) to (sga2);

        \draw[dashed, selected] (v) to (sgb);
        \draw[dashed, selected] (v) to (sgb2);

        \draw (u) edge[snake it, red, bend left] node[text=black] {$P_1$} (w1);
        \draw (w1) edge[snake it] node {$P_2$} (w2);
        \draw (w2) edge[snake it, red, bend left] node[text=black] {$P_3$} (v);
      \end{scope}

      % \draw [->,decorate, thick,
      % decoration={snake,amplitude=.4mm,segment length=2mm,post length=1mm}]
      % (4,0) -- node [above=1mm,midway,text width=3cm, sloped, align=center] {}
      % (6,0);
      % Post      

  \end{tikzpicture}
  \caption{The optimal solution, $T = T_1 \cup T_2$ connected by $(u,v)$.
    Since $c(u,v) \geq B(u,v)^{-(u,v)}$,
  the simple path $P_2$ has cost no larger than $(u,v)$ and can replace it in $T$.}\label{fig:red:test:sd:thm}
\end{figure}
   
   Since $T$ is a tree, and $P$ by definition doesn't contain the edge $(u,v)$
   then we must have that $P$ consists of a subpath contained in $T_1$, followed by
   a subpath not contained in $T$, followed by a subpath contained in $T_2$.
   In other words, we have,
   \[P = (u, P_1, w, P_2, z, P_3, v)\]
   as shown in Figure~\ref{fig:red:test:sd:thm}.

   Then by definition we have,
   \[sd(P_2) \leq sd(P) = B(u,v)^{-(u,v)} \leq c_{uv}\]
   and we can construct another solution $T'$ by replacing the edge
   $(u,v)$ with the
   vertices and edges of $P_2$ which then must have the cost
   \[c(T') = c(T) - c_{uv} + sd(P_2) \leq c(T)\mathnormal{.}\]
   Hence $T'$ also optimal in $G$ and $(u,v)$ is by definition redundant.
\end{proof}

\paragraph{Non-Terminals of Degree 3 (NTD3)}\label{sec:red:test:deg3}
Another bottleneck distance based test,
also proved valid for the \gls{pcstp} by \citet{uchoa2006reduction},
is the \textit{non-terminals of degree 3 test}.
\begin{theorem}[Non-Terminals of Degree 3 Test]\label{thm:ntd3}
  Let $u$ be a nonterminal of degree 3 in $G = (V, E, c, p)$,
  and let $v$, $w$, and $z$ be its adjacent
  vertices (see Figure~\ref{fig:red:test:ntd3:thm}). If we have
  $$\min\left(B(v,w) + B(v,z), B(w,v) + B(w,z),  B(z, v)+ B(z, w)\right) \leq
  c_{uv} + c_{uw} + c_{uz}$$
  then there exists an optimal solution to $G$ where $u$ has degree of
  \textit{at most} 2, that is $|\delta(u)| \leq 2$. Thus $u$ and its three edges, can be replaced by
  the edges $\{(v, w), (w,z), (z,v)\}$ with costs
  \[c_{vw} = c_{vu} + c_{uw},\quad c_{wz} = c_{wu} + c_{uz},\quad c_{zv} = c_{zu} + c_{uv}\mathnormal{.}\]
\end{theorem}
\begin{figure}[h!]\centering
    \begin{tikzpicture}[auto, node distance=1.3 cm]
      % Pre
      \begin{scope}[]
        \path[fill=black!5,use Hobby shortcut,closed=true]
        (-0.5, 1) .. (-4, 3) .. (-0.5,3);
        \path[fill=black!5,use Hobby shortcut,closed=true]
        (.5, 1) .. (4,3) .. (.5,3);
        \path[fill=black!5,use Hobby shortcut,closed=true]
        (0, -.5) .. (-2,-3) .. (2,-3);

        \node[steiner] (u) at(0, 0) {u};
        \node[terminal] (w) [above left=1.3 and 1.3 of u] {w};
        \node[terminal] (v) [below= of u] {v};
        \node[terminal] (z) [above right= 1.3 and 1.3 of u] {z};

        \node[subgraph, above left= 0.05cm and 1 cm of w] {$T_w$};
        \node[subgraph,above right= 0.05cm and 1 cm of z] {$T_z$};
        \node[subgraph, below right= 0.05cm and 1 cm of v] {$T_v$};
        
        \node (sgw) [above left= 0.7 of w] {};
        \node (sgv) [below= 0.7 of v] {};
        \node (sgz) [above right = 0.7 of z] {};
        
        %Edges
        \draw[selected] (u) edge node {$c_{uv}$} (v);
        \draw[selected] (u) edge node {$c_{uw}$} (w);
        \draw[selected] (u) edge node[swap] {$c_{uz}$} (z);
        \draw[dashed, selected] (v) to (sgv);
        \draw[dashed, selected] (w) to (sgw);
        \draw[dashed, selected] (z) to (sgz);

        \draw (v) edge[snake it, bend left] node[text=black] {$P_1$} (w);
        \draw (v) edge[snake it, bend right] node[swap,text=black] {$P_2$} (z);
      \end{scope}

      % \draw [->,decorate, thick,
      % decoration={snake,amplitude=.4mm,segment length=2mm,post length=1mm}]
      % (4,0) -- node [above=1mm,midway,text width=3cm, sloped, align=center] {}
      % (6,0);
      % Post      

  \end{tikzpicture}
  \caption{Non-Terminal of $|\delta(u)| = 3$ which connects the subtrees $T_v$, $T_w$, and $T_z$. The simple paths $P_1$ and $P_2$ provide an
     alternate way of reconnecting $T$ with at least as good cost.}\label{fig:red:test:ntd3:thm}
\end{figure}
\begin{proof}   
  W.l.o.g.\ consider the case where $B(v,w) + B(v,z) \leq c_{uv} + c_{uw} + c_{uz}$,
  and let $T$ be an optimal solution which contains $(u,v)$, $(u,w)$, and $(u,z)$.

  Let $P_1 = (v, ..., w)$ be the simple path with Steiner distance
  \[sd(P_1) = B(v,w)\]
  and similarly let $P_2 = (v, ..., z)$ be the simple path with Steiner distance
  \[sd(P_2) = B(v,z)\mathnormal{.}\]
  This gives us the situation in Figure~(\ref{fig:red:test:ntd3:thm}). Note that $u$ may be a part of either paths.

  Let $T_v$, $T_w$, and $T_z$ be the subtrees of $T$ obtained by
  removing the edges adjacent to $u$ from $T$, and construct a new solution with
   total cost no-larger than $T$ as,
   \[T' = T_v \cup T_w \cup T_z \cup P_1 \cup P_2\mathnormal{.}\]
   We must have $|\delta_{T'}(u)| \leq 2$. If we had $|\delta_{T'}| = 3$ then
   we would have
   \[P_1 = \left[v, (v,u), u, (u,w), w \right]\]
   and
   \[P_2 = \left[v, (v,u), u, (u,z), z \right]\]
   giving us
   \[B(v,w) + B(v, z) = sd(P_1) + sd(P_2) = 2 c_{vu} + c_{uw} + c_{uz} > c_{vu} + c_{uw} + c_{uz}\]
   which is a contradiction to our original assumption that
   \[B(v,w) + B(v,z) \leq c_{uv} + c_{uw} + c_{uz}\mathnormal{.}\]
   Thus $T'$ is an optimal solution to $G$ with $|\delta_{T'}(u)| \leq 2$.
\end{proof}

\subsection{Voronoi Reductions}

Reduction test based on a Voronoi decomposition of the input graph was introduced for the
\gls{pcstp} by \citet{gamrath2017scip}. These reduction tests
involve using Voronoi decomposition of the graph to generate conditional lower bounds based on
vertex inclusion/exclusion. Given a strong upper bound, these can then be used
to reduce the graph.

\paragraph{Voronoi Decomposition of a PCSTP Graph}

Given an instance of the \gls{pcstp} with the undirected graph $G = (V, E, c, p)$ and terminal set
$N \subseteq V$, we define a Voronoi region for a given terminal, $i \in N$, as the set of nonterminals
which are closer to $i$ than any other terminal in terms of simple shortests paths, that is,
\[R_i = \left\{v \in V \setminus N \mid i = \argmin_{u \in N} d_{vu}\right\}\]
where $R_i$ is the Voronoi region covered by the terminal $i$ and $d_{vu}$ is the shortest path between vertices
$v$ and $u$.

We then define the \textit{radius} of the $i$th Voronoi region as the shortest path
from its centre (the $i$th terminal)
to any vertex not in the region,
\[\radius(i) = \min_{v \not\in R_i} d_{iv}\mathnormal{.}\]
From this we define a \textit{prize-collecting radius} of a region as,
\[\pcradius(i) = \min(\radius(i), p_i)\mathnormal{.}\]
We can intuitively see the \textit{pc radius} of a region as a kind of lower bound
on the cost contribution
 of that region to any
 \textit{connected}, multi-vertex solution. Either the prize for the respective terminal must be paid as a penalty
 \textit{or} the terminal must be connected to the rest of the tree by a path which must have some cost \textit{higher}
 than the \textit{radius} of the region.

 Finally, let
 \[z_1, z_2, \ldots, z_{|N|}\]
 be a monotonically increasing ordering of terminals such that
\[\pcradius(z_1) \leq \pcradius(z_2) \leq ... \leq \pcradius(z_{|N|})\mathnormal{.}\]

\paragraph{Lower Bound Reductions}

The first reduction we can present based on Voronoi regions is based on a lower bound given that some nonterminal is
part of a solution. This was first introduced for the \gls{stp} by \citet{polzin2001improved} and later adapted by
\citet{rehfeldt2016reduction} for the \gls{pcstp}.

\begin{theorem}\label{thm:vor:1}
  Let $T = (V_T, E_T, c, p)$ be a solution to the \gls{pcstp} in the graph $G = (V, E, c, p)$ with terminal set $N$ such that
  $v_k \in V_T$ for some $v_k \in V \setminus N$. Finally, let $v_a \in N$ and $v_b \in N$ be the two terminals \textit{closest}
  to $v_k$, then we have
  \[d(v_a, v_k) + d(v_b, v_k) + \sum_{i = 1}^{|N| - 2} \pcradius(z_i) \leq c(T)\mathnormal{.}\]
\end{theorem}
\begin{proof}
  Consider a solution $T$ as described above which contains some nonterminal $u \in V_T$.
  For all $i \in (N \cap V_T)$ let $P_i$ be the
  path from $v_i$ to $u$ in $T$ (this must be well defined as $T$ is a tree). Furthermore,
  let $P_i'$ be the subpath of $P_i$ which starts in $v_i$ and ends in the first vertex along
  $P_i$ not in $R_i$, and
  let $\Delta(P_i)$ be the number of times a Voronoi border is crossed by the path $P_i$ where
  we have $\Delta(P_i) = \infty$ if $v_i \not\in V_T$.

  \begin{figure}[h!]
     \centering
     \begin{tikzpicture}[auto, node distance=1.3cm]
       \node[steiner] (vi) {$v_i$};
       \node[steiner] (vib) [below= of vi] {$\phantom{v_i}$};

       \node[terminal] (vj) [right= of vi] {$v_m$};
       \node[terminal] (vl) [left= of vib] {$v_l$};
       \node [above left= 0.2 of vib] {$P_l$};

       
       \node[steiner]  (vlal) [above left= 1.2 and 1 of vl] {$\phantom{v_i}$};
       \node [above right= 0.05 and 0.1 of vlal] {$P_{i_3}$};
       \node[terminal]  (vlala) [above= of vlal] {$\phantom{v_i}$};

    \node[steiner] (vll) [below left=  of vl] {$\phantom{v_i}$};
    \node[terminal] (vlll) [below left=  of vll] {$\phantom{v_i}$};

    \node[terminal] (vibr) [below right= 0.5 and 1.3 of vib] {$\phantom{v_i}$};

    \begin{scope}
      % VORONOI :(
            \draw[dashed] let \p1=(vll),\p2=(vlll) in 
            ({(\x1+\x2)/2-1cm},{(\y1+\y2)/2+1cm}) coordinate(A)
            --
            ({(\x1+\x2)/2+1cm},{(\y1+\y2)/2-1cm}) coordinate(AA);
            \draw[dashed] let \p1=(vl),\p2=(vlal) in 
            ({(\x1+\x2)/2+1cm},{(\y1+\y2)/2+1cm}) coordinate(B)
            --
            (A);

            \draw[dashed] let \p1=(vib),\p2=(vibr) in 
            ({(\x1+\x2)/2+1cm},{(\y1+\y2)/2+1cm}) coordinate(C)
            --
            (AA);

            \draw[dashed] (B) -- (C);
            \draw[dashed] (C) -- ([xshift=2cm]C);
            \draw[dashed] (B) -- ([yshift=2cm]B);
            \draw[dashed] (AA) -- ([yshift=-2cm]AA);
            \draw[dashed] (A) -- ([xshift=-2cm]A);
   \end{scope}
   \begin{scope}
     % Edges
     \draw[selected, blue] (vlala) edge (vlal);
     \draw[selected, blue] (vlal) edge (vl);

     \draw[selected, green] (vlll) edge node[black] {$P_{i_2}$} (vll);
     \draw[selected, yellow] (vibr) edge node[black] {$P_{i_1}$} (vib);

     \draw[selected, black] (vll) edge (vl);
     
     \draw[selected] (vj) edge node[above] {$P_m$} (vi);

     \draw[selected, orange] (vl) edge (vib);
     \draw[selected, orange] (vib) edge (vi);
   \end{scope}
  \end{tikzpicture}
  
  \caption{Example solution of the \gls{pcstp} split into paths.}\label{fig:pre:vor:pro}
\end{figure}

  Let $v_m$ and $v_l$ be two terminals such that $\Delta(P_m) + \Delta(P_l)$ is minimal among
  all pairs of terminals where $P_m$ and $P_l$ are disjoint paths. Since $T$ is a tree and
  $v_i$ must have degree of at least $2$ (See Section~\ref{sec:red:test:deg1}) there must
  be at least one such pair.

  Consider the set of paths
  $$\mathcal{P} = \{P_i' \mid i \in (V_T \cap N) \setminus \{v_m, v_l\} \}
  \mathnormal{.}$$
  These paths must be disjoint as $T$ contains no cycles. The paths in $\mathcal{P}$
  alongside $P_m$ and $P_l$ are then all pairwise disjoint paths which cover a subset of the
  edges
  in $E_T$.

An example of this partition of $T$ is shown in Figure~(\ref{fig:pre:vor:pro}). The two ``closest''
terminals $v_l$ and $v_m$ have disjoint paths $P_l$ and $P_m$ to the interesting nonterminal $v_i$.
The rest of the solution is partitioned by $P_i'$ paths, these paths run from a terminal towards $v_i$
until ---inclusive--- the first edge which crosses a Voronoi boundary. No two paths in such a decomposition
can share an edge. This follows directly from the fact that each Voronoi region contains
\textit{only} a single terminal, and that there can be only a single path from vertex to $v_i$ (no cycles).
Hence a border edge can only be crossed in one ``direction'' this way.

With this, clearly, the total edge costs of these paths cannot be greater than the edge cost of $T$, as
we have only selected a subset of the edges of $T$, and never select an edge twice. In other words, we have
  \begin{equation}
    c(E(P_m)) + c(E(P_l)) + \sum_{P_i \in \mathcal{P}} c(E(P)) \leq c(E_T)\mathnormal{.}\label{eq:vor:one}
  \end{equation}

  
  Additionally, by definition
  we must have that $P_i' \leq \pcradius(i)$, $d(v_m, v_i) \leq c(E(P_m))$, and
  $d(v_l, v_i) \leq c(E(P_l))$. This gives us

  \begin{equation}
d(v_l, v_i) + d(v_m, v_i) + \sum_{P_i \in \mathcal{P}} \pcradius(i) \leq c(E(P_m)) + c(E(P_l)) + \sum_{P_i \in \mathcal{P}} c(E(P))\label{eq:vor:two}
\end{equation}

  Finally, for the set of terminals not covered by $T$, we must ---again by the definition of $\pcradius$--- have
  \begin{equation}
  \sum_{i \in N \setminus V_T} \pcradius(i) \leq \sum_{i \in N \setminus V_T} p_i \mathnormal{.}\label{eq:vor:three}
  \end{equation}

  By combining the equations (\ref{eq:vor:one}), (\ref{eq:vor:two}), and (\ref{eq:vor:three}), and
   by the definition of vertices $v_a$ and $v_b$ being the terminals closest to $v_i$ we have,
   \[d(v_a, v_k) + d(v_b, v_k) + \sum_{i = 1}^{|N| - 2} \pcradius(z_i) \leq c(T)\]
   and we are done.
 \end{proof}
 If a nonterminal is found such that the bound in Theorem \ref{thm:vor:1} is larger than the objective value of
 any found feasible solution, then that nonterminal is redundant and it is valid to remove it and its edges.

 We can produce a similar lower bound if the ``interesting'' vertex is instead a terminal.
\begin{theorem}\label{thm:vor:2}
  Let $T = (V_T, E_T, c, p)$ be a solution to the \gls{pcstp} in the graph $G = (V, E, c, p)$ with terminal set $N$ such that
  $v_k \in V_T$ for some $v_k \in N$ and $V_T \neq \{v_k\}$. Finally, let $v_b \in N$ be the terminal \textit{closest}
  to $v_k$, and let $z'_1, \ldots, z_{|N|-1}$ be an ordering of the set $N \setminus \{v_k\}$ such that
  \[\pcradius(z_1) \leq \cdots \leq \pcradius(i) \leq \cdots \leq \pcradius(z'_{|N|-1})\]
  then we have
  \[d(v_b, v_k) + \sum_{i = 1}^{|N| - 2} \pcradius(z'_i) \leq c(T)\mathnormal{.}\]
\end{theorem}
The proof of Theorem~\ref{thm:vor:2} is symmetric to the proof of Theorem~\ref{thm:vor:1}. If a terminal is found such that
the respective lower bound is higher than the objective value of a given incumbent, then the terminal is redundant and it is
 valid to remove it from $G$ along with its adjacent edges.

 \subsection{Summary of Usage}\label{sec:pre:summary-usage}
There's a general consensus in literature that preprocessing is an important part of producing solutions
for the \gls{pcstp}. Preprocessing described in this section is applied natively to the \gls{pcstp}
by all of
\citet{lucena2004strong, Ljubic:2004:memetic, ljubic2005solving,akhmedov2016divide,gamrath2017scip}.
Additionally, preprocessing is applied to the \gls{sap} by \citet{leitner2016dual}, and proposed
as further improvements by \citet{fu2014knowledge}.

\paragraph{Combinations and Effectiveness}

\citet{lucena2004strong} applies all four of the degree tests (NTD1, NTD2, TD1+UDV, TD2+UDV)
alongside shortest path relaxations of SD and NTDk (for ``small'' numbers of $k$)
tests. They do not
present in which order tests are applied or how often they are repeated, but they
do present the graph sizes of their benchmark cases before and after applying preprocessing.

\citet{Ljubic:2004:memetic} and \citet{ljubic2005solving} share a preprocessing routine
 which applies in order:
\begin{enumerate}
\item degree one tests: NTD1+TD1+UDV,
\item degree two tests: NTD2+TD2,
\item shortest path relaxation of the special distance test (here called the least cost test),
\item shortest path relaxation of NTDk for $k = 3,\ldots,8$, and
\item the minimum adjacency test.
\end{enumerate}
These are repeated until a fix point is reached.
They present both numbers on graph reduction and preprocessing
time. Across all problem instances, they report on average
a 45\% reduction in graph size.

\citet{akhmedov2016divide} applies the four degree tests until a fix point is reached. They
report reduction in graph sizes for their test cases as well as preprocessing time and
total computation time with/without preprocessing.

\citet{rehfeldt2016reduction} present an in depth study of preprocessing routines for the
\gls{pcstp}. They report an on average reduction of vertex and edge counts by $91.5\%$ and
$88.8\%$ respectively for their ``strongest'' reduction package across DIMACs instances.
\begin{example}
Finally, we give an example of the power of
these preprocessing methods. Figure~\ref{fig:pre:ex} shows how
the local tests defined in Section~\ref{sec:pre:local} alone are enough
to solve the ---admittedly simplistic--- instance of the \gls{pcstp}
from Figure~\ref{fig:pcstp:01}.

\begin{figure}[h!]\centering
  \begin{tikzpicture}[auto,
    node/.style={minimum width=10cm, inner sep={(10cm, 5cm)}}]
    \node[minimum width=5cm,
      minimum height=2.5cm] (1) {
      \begin{tikzpicture}[node distance=1.5 cm,
            every node/.style={minimum width=0, minimum height=0}]]
    % Nodes
    \node[terminal, label={12}] (a) at (0,0) {a};
    \node[steiner] (b) [right=of a] {b};
    \node[steiner] (c) [right=of b] {c};
    \node[terminal, label={10}] (d) [above =of c] {d};
    \node[steiner] (e) [left=of d] {e};
    \node[steiner] (f) [left=of e] {f};
    \node[terminal, label={3}] (g) [above right=0.75 and 1.3 of c] {g};
    % Edges

    \draw (a) edge node[below]{4} (b);
    \draw (b) edge node[near start]{5} (d);
    \draw (b) edge node[below]{8} (c);
    \draw (c) edge node{3} (d);
    \draw (c) edge node{2} (g);
    \draw (c) edge node[near start]{5} (e);
    \draw (d) edge node {6} (e);
    \draw (d) edge node{10} (g);
    \draw (e) edge node{1} (f);
    \end{tikzpicture}
  };
  
  \node[minimum width=5cm,
      minimum height=2.5cm] (2) at (9, 0) {
    \begin{tikzpicture}[node distance=1.5 cm,
          every node/.style={minimum width=0, minimum height=0}]]
    % Nodes
    \node[terminal, label={12}] (a) at (0,0){a};
    \node[steiner] (b) [right=of a] {b};
    \node[steiner] (c) [right=of b] {c};
    \node[terminal, label={10}] (d) [above =of c] {d};
    \node[steiner] (e) [left=of d] {e};
    \node[terminal, label={3}] (g) [above right=0.75 and 1.3 of c] {g};
    % Edges

    \draw (a) edge node[below]{4} (b);
    \draw (b) edge node[near start]{5} (d);
    \draw (b) edge node[below]{8} (c);
    \draw (c) edge node{3} (d);
    \draw (c) edge node{2} (g);
    \draw (c) edge node[near start]{5} (e);
    \draw (d) edge node{6} (e);
    \draw (d) edge node{10} (g);
  \end{tikzpicture}
};

\node[minimum width=5cm,
      minimum height=2.5cm](3) at (0, -4) {
  \begin{tikzpicture}[node distance=1.5 cm,
        every node/.style={minimum width=0, minimum height=0}]]
    % Nodes
    \node[terminal, label={12}] (a) at (0,0){a};

    \node[terminal] (b) [right=of a, label={8}] {b};
    \node[steiner] (c) [right=of b] {c};
    \node[terminal, label={10}] (d) [above =of c] {d};
    \node[steiner] (e) [left=of d] {e};
    \node[terminal, label={3}] (g) [above right=0.75 and 1.3 of c] {g};
    % Edges

    \draw (b) edge node[near start]{5} (d);
    \draw (b) edge node[below]{8} (c);
    \draw (c) edge node{3} (d);
    \draw (c) edge node{2} (g);
    \draw (c) edge node[near start]{5} (e);
    \draw (d) edge node{6} (e);
    \draw (d) edge node{10} (g);
  \end{tikzpicture}};
\node[minimum width=5cm,
      minimum height=2.5cm] (4) at (9, -4) {
  \begin{tikzpicture}[node distance=1.5 cm,
        every node/.style={minimum width=0, minimum height=0}]]
    % Nodes
    \node[terminal, label={12}] (a) at (0,0){a};

    \node[steiner] (c) [right=of a] {c};
    \node[terminal, label={13}] (d) [above =of c] {d};
    \node[steiner] (e) [left=of d] {e};
    \node[terminal, label={3}] (g) [above right=0.75 and 1.3 of c] {g};
    % Edges

    \draw (c) edge node{3} (d);
    \draw (c) edge node{2} (g);
    \draw (c) edge node[near start]{5} (e);
    \draw (d) edge node{6} (e);
    \draw (d) edge node{10} (g);
  \end{tikzpicture}};
\node[minimum width=5cm,
      minimum height=2.5cm] (5) at (0, -8) {
  \begin{tikzpicture}[node distance=1.5 cm,
        every node/.style={minimum width=0, minimum height=0}]]
    % Nodes
    \node[terminal, draw=none, fill=none] (a) at (0,0){\phantom{a}};
    \node[steiner] (c) [right =of a] {c};
    \node[terminal, label={13}] (d) [above =of c] {d};
    \node[steiner] (e) [left=of d] {e};
    \node[terminal, label={3}] (g) [above right=0.75 and 1.3 of c] {g};
    % Edges

    \draw (c) edge node{3} (d);
    \draw (c) edge node{2} (g);
    \draw (c) edge node[near start]{5} (e);
    \draw (d) edge node{6} (e);
    \draw (d) edge node{10} (g);
  \end{tikzpicture}};
\node[minimum width=5cm,
      minimum height=2.5cm] (6) at (9, -8) {
        \begin{tikzpicture}[node distance=1.5 cm,
          every node/.style={minimum width=0, minimum height=0}]]
    % Nodes

    \node[steiner] (c) at (0,0) {c};
    \node[terminal, label={13}] (d) [above =of c] {d};
    \node[terminal, label={3}] (g) [above right=0.75 and 1.3 of c] {g};
    % Edges

    \draw (c) edge node{3} (d);
    \draw (c) edge node{2} (g);
    \draw (d) edge node{10} (g);
  \end{tikzpicture}
};
\node[minimum width=5cm,
minimum height=2.5cm] (7) at (0, -12) {
  \begin{tikzpicture}[node distance=1.5 cm,
    every node/.style={minimum width=0, minimum height=0}]
    % Nodes
    \node[terminal, label={13}] (d) {d};
    \node[terminal, label={3}] (g) [below right=0.75 and 1.3 of d] {g};
    % Edges

    \draw (d) edge node{5} (g);
  \end{tikzpicture}};
\node[minimum width=5cm,
      minimum height=2.5cm] (8) at (9, -12) {
  \begin{tikzpicture}[node distance=1.5 cm,
        every node/.style={minimum width=0, minimum height=0}]
    % Nodes
    \node[terminal, label={13}] (d) {d};
  \end{tikzpicture}};

% \draw [->,decorate, thick,
% decoration={snake,amplitude=.4mm,segment length=2mm,post length=1mm}]
% (1)
% edge node [above=1mm,midway,text width=3cm, sloped, align=center] {hej}
% (2);

\draw (1) edge[snake it] node[snake node] {1: $NTD_1$} (2);
\draw (2) edge[snake it] node[snake node] {2: $TD_1$} (3);
\draw (3) edge[snake it] node[snake node] {3: $MA$} (4);
\draw (4) edge[snake it] node[snake node] {4: $UV$} (5);
\draw (5) edge[snake it] node[snake node] {5: $NTD_2$} (6);
\draw (6) edge[snake it] node[snake node] {6: $NTD_2$} (7);
\draw (7) edge[snake it] node[snake node] {7: $TD_2 + UV$} (8);
\end{tikzpicture}
\caption{Graph reduction on the \gls{pcstp} instance in Figure~\ref{fig:pcstp:01}. In the final graph,
 the node $d$ represents nodes $\{a,b,d\}$.}
\label{fig:pre:ex}
\end{figure}

This example shows how successive application of preprocessing routines can be at
reducing the size of an input graph. These routines have a cascading effect as one
graph reduction makes another one legal. In Figure~(\ref{fig:pre:ex}) this comes to
pass when the third reduction -- the \textit{Minimum Adjacency} test -- contracts
the edge between nodes $b$ and $d$, causing the degree of nodes $c$ and $e$ to
 drop to 2, which allows for the fifth and sixth reductions.
\end{example}

It is an open question which selection of graph reduction tests are
 most efficient, and likewise with the ordering of graph reductions.
%%% Local Variables:
%%% TeX-master: "report"
%%% reftex-default-bibliography: ("lit.bib")
%%% End:
