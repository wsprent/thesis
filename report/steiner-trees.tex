\chapter{Steiner Trees in Graphs}
\label{chap:steiner-trees}

\section{Steiner Tree Problem}
The Steiner Tree Problem (STP) is defined as follows.
Given an undirected graph
$$G = (V, E, c)$$
 where $c: E \to \RR^+$ defines edge weights, and additionally a set of terminals $N \subseteq V$ then
a Steiner Tree is a tree, $T = (V_T, E_T, c) \subseteq G$, which spans $N$ with minimal cost. We denote vertices which must be part of a feasible solution,
that is $v \in N$, as \textit{terminals}, vertices which are not terminals as \textit{non-terminals},
and vertices which are not required to be part of a solution but are so anyway, that is
$v \in E_T \cap (V \setminus N)$, as \textit{Steiner vertices}.

Furthermore, let $S \subset V$ define a cut in $G$. If we have $S \cap N \neq \emptyset$ and $(V \setminus S) \cap \emptyset$ then we
 call $S$ a \textit{Steiner cut}.

When $N = V$, the STP is equivalent to the Minimum Spanning Tree Problen, and when $|N| = 2$ the STP is equivalent to finding the
shortest path between two vertices. However, while these problems both have polynomial-time solutions, the STP is an NP-Hard problem,
 the decision variant being part of Edmund Karp's original 21 NP-Complete problems \citep{karp1972reducibility}.

\begin{figure}[h]\centering
\begin{tikzpicture}[auto, node distance=1.5 cm]
  %Nodes
  \node[terminal] (a) {a};
  \node[steiner] (b) [right=of a] {b};
  \node[steiner] (c) [right=of b] {c};
  \node[terminal] (d) [above =of c] {d};
  \node[steiner] (e) [left=of d] {e};
  \node[steiner] (f) [left=of e] {f};
  \node[terminal] (g) [above right=0.75 and 1.3 of c] {g};
  % Edges
  \begin{scope}[every edge/.style={draw=black, thick}]
    \draw (a) edge node[below]{4} (b);
    \draw (b) edge node[near start]{5} (d);
    \draw (b) edge node[below]{8} (c);
    \draw (c) edge node{3} (d);
    \draw (c) edge node{2} (g);
    \draw (c) edge node[near start]{5} (e);
    \draw (d) edge node{6} (e);
    \draw (d) edge node{10} (g);
    \draw (e) edge node{1} (f);
  \end{scope}
\end{tikzpicture}
\caption{Instance of the Steiner Tree Problem. Terminals are coloured black and non-terminals coloured white.}
\label{fig:stp:01}
\end{figure}

Figure (\ref{fig:stp:01}) shows an instance of the STP with three terminals and four Steiner points. Since vertices
A, C, and D are terminals, they must be spanned by any feasible solution. Figure (\ref{fig:stp:01:feasible}) shows
a feasible solution (which we here denote as a set of edges)
$$E_T = \{(a,b), (b,d), (d,g)\}$$
with cost
$$c(T) = 4 + 5 + 10 = 19\mathnormal{.}$$
However, $T$ is not a Steiner tree as there exists at least one feasible solution with lower cost, i.e. the solution
$$E_{T'} = \{(a,b), (b,d), (d,c), (c,g)\}$$
in Figure (\ref{fig:stp:01:min}) which has cost
$$c(T') = 4 + 5 + 3 + 2 = 14\mathnormal{.}$$
% Suboptimal and Optimal STP solutions 
\begin{figure}[h]\centering
  \begin{subfigure}{0.47\linewidth}
    \begin{tikzpicture}[auto, node distance=1.5 cm]
      % Nodes
      \node[terminal] (a) {a};
      \node[steiner] (b) [right=of a] {b};
      \node[steiner] (c) [right=of b] {c};
      \node[terminal] (d) [above =of c] {d};
      \node[steiner] (e) [left=of d] {e};
      \node[steiner] (f) [left=of e] {f};
      \node[terminal] (g) [above right=0.75 and 1.3 of c] {g};
      % Edges
      \begin{scope}[every edge/.style={draw=black, thick}]
        \draw (a) edge[selected] node[below]{4} (b);
        \draw (b) edge[selected] node[near start]{5} (d);
        \draw (b) edge node[below]{8} (c);
        \draw (c) edge node{3} (d);
        \draw (c) edge node{2} (g);
        \draw (c) edge node[near start]{5} (e);
        \draw (d) edge node{6} (e);
        \draw (d) edge[selected] node{10} (g);
        \draw (e) edge node{1} (f);
      \end{scope}
    \end{tikzpicture}
    \caption{Feasible but not optimal.}
    \label{fig:stp:01:feasible}
  \end{subfigure}
  \quad
  \begin{subfigure}{0.47\linewidth}
    \begin{tikzpicture}[auto, node distance=1.5 cm]
      % Nodes
      \node[terminal] (a) {a};
      \node[steiner] (b) [right=of a] {b};
      \node[steiner] (c) [right=of b] {c};
      \node[terminal] (d) [above =of c] {d};
      \node[steiner] (e) [left=of d] {e};
      \node[steiner] (f) [left=of e] {f};
      \node[terminal] (g) [above right=0.75 and 1.3 of c] {g};
      % Edges
      \begin{scope}[every edge/.style={draw=black, thick}]
        \draw (a) edge[selected] node[below]{4} (b);
        \draw (b) edge node[below]{8} (c);
        \draw (b) edge[selected] node[near start]{5} (d);
        \draw (c) edge[selected] node{3} (d);
        \draw (c) edge[selected] node{2} (g);
        \draw (c) edge node[near start]{5} (e);
        \draw (d) edge node{6} (e);
        \draw (d) edge node{10} (g);
        \draw (e) edge node{1} (f);
      \end{scope}
    \end{tikzpicture}

    \caption{Minimal weight, optimal.}
    \label{fig:stp:01:min}
  \end{subfigure}
  \caption{Solutions to the STP in (\ref{fig:stp:01}). Red edges are part of the solution.}
\end{figure} 

\subsection{ILP Formulations}

\paragraph{Cut Formulation} Let $x$ be a decision vector of length $|E|$ where
$x_{ij} = 1$ implies that $(i,j) \in T$ and $x_{ij} = 0$ implies that $(i,j) \not\in T$,
 and let $c$ be a vector of node-weight s.t. $c_{ij} = c((i,j))$.
Then define the function $x : E \to \ZZ$ as
$$x(E') = \sum_{i,j \in E'} x_{ij}$$
that is, $x(E)$ is equal to the number of selected edges in $E$.
Finally let,
$$\delta(S) = \{(i, j) \mid i \in S \wedge j \in (E \setminus S)\}$$
be all edges which span the cut defined by $S$. Then we can formulate
 the STP as in ILP in terms of cuts (see Formulation (\ref{form:stp:cut})).
 \begin{formulation}[h!]
   \begin{subequations}
     \begin{alignat}{3} %TODO: Find reference for this formulation
       &\underset{x}{\text{minimize}}
       & & c^T x & \\
       & \text{subject to}\quad
       & & x(\delta(S)) \geq 1 \qquad&& \forall S \subset V \label{form:stp:cut:cut}\\
       &&&&& S \cap N \neq \emptyset \nonumber\\
       &&&&& S \cap (V \setminus N) \neq \emptyset \nonumber\\
       &&& x \in \BB. &&
     \end{alignat}\label{form:stp:cut}
   \end{subequations}
   \caption{The \textit{Cut Formulation} of the STP \citep{koch1998solving}.}
 \end{formulation}

 Constraints (\ref{form:stp:cut:cut}) ensures that any feasible solution, $T$,  must span all terminals in $G$, by
 requiring that every Steiner cut in $G$ must be crossed by an edge in $T$. This, combined with objective function
  ensures that any optimal solution to (\ref{form:stp:cut}) must be a Steiner tree in $G$ and vice versa.

\section{Steiner Aborescence Problem}
The Steiner Aborescence Problem (SAP) is the directed version of the Steiner Tree Problem.
Given a \textit{directed} graph,
$$G = (V, A, c)$$
a non-empty terminal set $N \subseteq V$, a root terminal $r \in N$, and arc-weights $c : A \to \RR^+$,
then a Steiner Aborescence, $T \subseteq A$, is an aborescence
in $G$, rooted in $r$, which spans $N$ and has minimal cost.

As with the STP, we denote vertices in $N$ as \textit{terminals}, and any non-terminals spanned by a Steiner Aborescence as
 \textit{Steiner vertices} or \textit{Steiner Points}.
\begin{figure}[h]\centering
\begin{tikzpicture}[auto, node distance=1.7 cm]
  %Nodes
  \node[terminal, root] (r) at (0,0) {r};
  \node[steiner] (b) [right= of r] {b};
  \node[terminal] (c) [above= of b] {c};
  \node[steiner] (d) [below= of b] {d};
  \node[steiner] (e) [left= of d] {e};
  \node[terminal] (f) [above right= of d] {f};
  \node[terminal] (g) [above right= of b] {g};
  % Edges
  \begin{scope}[every edge/.style={draw=black, thick}]
    \draw[->] (r) edge [bend left=20] node[above]{4} (b);
    \draw[<-] (r) edge [bend right=20] node[below]{1} (b);
    \draw[->] (r) edge [bend left=20] node[above]{6} (c);
    \draw[->] (r) edge  node{2} (e);
    \draw[->] (b) edge [bend left=20] node{3} (c);
    \draw[<-] (b) edge [bend right=20] node[swap]{2.5} (c);
    \draw[<-] (b) edge  node{3} (d);
    \draw[->] (b) edge  node{5} (f);
    \draw[->] (c) edge [bend left=20] node{3.5} (g);
    \draw[<-] (d) edge  node{1.5} (e);
    \draw[->] (d) edge [bend right] node{2} (f);
    \draw[->] (f) edge node{2} (g);
  \end{scope}
\end{tikzpicture}
\caption{Instance of the Steiner Aborescence Problem. Terminals are coloured black, non-terminals are coloured white, and the root terminal
  has a blue outline.}
\label{fig:sap:01}
\end{figure}

Figure (\ref{fig:sap:01}) shows an instance of the SAP with four terminals. Figure (\ref{fig:sap:01:opt}) shows a Steiner Aborescence in (\ref{fig:sap:01})
with cost $c(T) = 13.5$. It is worth noting that since all arcs must point away from the root in an aborescence, while the path $p = (c, b, r)$  has
 lower cost than the arc $(r, c)$, it cannot be part of a solution. Similarly, solutions to the SAP in the same graph but with root $c$ have lower cost.

\begin{figure}[h]\centering
\begin{tikzpicture}[auto, node distance=1.7 cm]
  %Nodes
  \node[terminal, root] (r) at (0,0) {r};
  \node[steiner] (b) [right= of r] {b};
  \node[terminal] (c) [above= of b] {c};
  \node[steiner] (d) [below= of b] {d};
  \node[steiner] (e) [left= of d] {e};
  \node[terminal] (f) [above right= of d] {f};
  \node[terminal] (g) [above right= of b] {g};
  % Edges
  \begin{scope}[every edge/.style={draw=black, thick}]
    \draw[->] (r) edge [bend left=20] node[above]{4} (b);
    \draw[<-] (r) edge [bend right=20] node[below]{1} (b);
    \draw[->] (r) edge [bend left=20, selected] node[above]{6} (c);
    \draw[->] (r) edge [selected] node{2} (e);
    \draw[->] (b) edge [bend left=20] node{3} (c);
    \draw[<-] (b) edge [bend right=20] node[swap]{2.5} (c);
    \draw[<-] (b) edge  node{3} (d);
    \draw[->] (b) edge  node{5} (f);
    \draw[->] (c) edge [bend left=20] node{3.5} (g);
    \draw[<-] (d) edge [selected, selected] node{1.5} (e);
    \draw[->] (d) edge [bend right, selected] node{2} (f);
    \draw[->] (f) edge [selected] node{2} (g);
  \end{scope}
\end{tikzpicture}
\caption{Steiner Aborescence in (\ref{fig:sap:01}) with cost $13.5$.}
\label{fig:sap:01:opt}
\end{figure}

The SAP is relevant mainly due to a tendency of reformulating the undirected STP into the SAP \citep{koch1998solving}, as well as reformulating
the PCSTP into the SAP \citep{gamrath2017scip, Ljubic:2004:memetic} and
prize-collecting variants of the SAP \citep{leitner2016dual, ljubic2005solving} before stating them as integer programs. This is due to results which show that LP relaxations of
 directed Steiner Tree Problem variants behave better than undirected variants \citep{Chopra:1994}.

\subsection{Reduction from Other Variants}

\subsubsection{Steiner Tree Problem}
\begin{figure}[h]\centering
  \begin{subfigure}{0.47\linewidth}
    \begin{tikzpicture}[auto, node distance=1.5 cm]
      % Nodes
      \node[terminal] (a) {a};
      \node[steiner] (b) [right=of a] {b};
      \node[steiner] (c) [right=of b] {c};
      \node[terminal] (d) [above =of c] {d};
      \node[steiner] (e) [left=of d] {e};
      \node[steiner] (f) [left=of e] {f};
      \node[terminal] (g) [above right=0.75 and 1.3 of c] {g};
      % Edges
      \begin{scope}[every edge/.style={draw=black, thick}]
        \draw (a) edge node[below]{4} (b);
        \draw (b) edge node[near start]{5} (d);
        \draw (b) edge node[below]{8} (c);
        \draw (c) edge node{3} (d);
        \draw (c) edge node{2} (g);
        \draw (c) edge node[near start]{5} (e);
        \draw (d) edge node{6} (e);
        \draw (d) edge node{10} (g);
        \draw (e) edge node{1} (f);
      \end{scope}
    \end{tikzpicture}
    \caption{Problem Instance (\ref{fig:stp:01})}
  \end{subfigure}
  \begin{subfigure}{0.47\linewidth}
    \begin{tikzpicture}[auto, node distance=1.5 cm]
      % Nodes
      \node[terminal] (a) {a};
      \node[steiner] (b) [right=of a] {b};
      \node[steiner] (c) [right=of b] {c};
      \node[terminal, root] (d) [above =of c] {d};
      \node[steiner] (e) [left=of d] {e};
      \node[steiner] (f) [left=of e] {f};
      \node[terminal] (g) [above right=0.75 and 1.3 of c] {g};
      % Edges
      \begin{scope}[every edge/.style={draw=black, thick}]
        \draw[<->] (a) edge node[below]{4} (b);
        \draw[<->] (b) edge node[near start]{5} (d);
        \draw[<->] (b) edge node[below]{8} (c);
        \draw[<->] (c) edge node{3} (d);
        \draw[<->] (c) edge node{2} (g);
        \draw[<->] (c) edge node[near start]{5} (e);
        \draw[<->] (d) edge node{6} (e);
        \draw[<->] (d) edge node{10} (g);
        \draw[<->] (e) edge node{1} (f);
      \end{scope}
    \end{tikzpicture}
    \caption{Instance (\ref{fig:stp:01}) as a SAP instance. Root node is coloured blue.}
  \end{subfigure}
  \caption{Reduction from STP to SAP.}
  \label{fig:stptosap}
\end{figure}

\subsubsection{Prize-Collecting Steiner Tree Problem}

\subsection{ILP Formulations}
\section{Prize-Collecting Steiner Tree Problem}
Given an undirected graph
$$G = (V, E, c, p)$$
where $c: E \to \RR^+$ defines edge weights,
and $p: V \to \RR_0^+$ defines vertex \textit{prizes}, then the solution to the \textit{Prize-Collecting
  Steiner Tree Problem} (PCSTP) is a tree
$$T = (V_T, E_T, c, p) \subseteq G$$
which minimizes
$$c(T) = \min_T GW(T) = \sum_{(i,j) \in E_T} c_{ij} + \sum_{v\in (V \setminus V_T)} p_v$$
which is also known as the {\textit{Goemans-Williamson Minimization Problem}}. This involves minimising
 the total cost of edges in $T$ while trying to minimise the amount of uncollected prize, or perhaps paid penalties.

In the context of the PCSTP, We denote vertices with nonzero prize as \textit{terminals}, giving the terminal set
$$N = \{v \mid p_v > 0\} \subseteq V\mathnormal{.}$$

\begin{figure}[h]\centering
\begin{tikzpicture}[auto, node distance=1.5 cm]
  %Nodes
  \node[terminal, label={12}] (a) {a};
  \node[steiner] (b) [right=of a] {b};
  \node[steiner] (c) [right=of b] {c};
  \node[terminal, label={10}] (d) [above =of c] {d};
  \node[steiner] (e) [left=of d] {e};
  \node[steiner] (f) [left=of e] {f};
  \node[terminal, label={3}] (g) [above right=0.75 and 1.3 of c] {g};
  % Edges
  \begin{scope}[every edge/.style={draw=black, thick}]
    \draw (a) edge node[below]{4} (b);
    \draw (b) edge node[near start]{5} (d);
    \draw (b) edge node[below]{8} (c);
    \draw (c) edge node{3} (d);
    \draw (c) edge node{2} (g);
    \draw (c) edge node[near start]{5} (e);
    \draw (d) edge node{6} (e);
    \draw (d) edge node{10} (g);
    \draw (e) edge node{1} (f);
  \end{scope}
\end{tikzpicture}
\caption{Instance of the Prize-Collecting Steiner Tree Problem. Terminals are coloured black and non-terminals coloured white.}
\label{fig:pcstp:01}
\end{figure}

Figure (\ref{fig:pcstp:01}) shows an instance, $G$, of the PCSTP created by assigning prizes to the terminals of the STP instance, $G_{STP}$, in Figure (\ref{fig:stp:01}).
Figure (\ref{fig:pcstp:01})
shows an optimal solution, $T = (V_T, E_T)$, to $G$. An interesting observation is that
$T$ is a sub-graph of the solution in Figure (\ref{fig:stp:01:min}) to $G_{STP}$.
In fact, if we were to modify $G_{STP}$ by setting its terminal set to the vertices spanned
by $T$, that is we define $G'_{STP} =  G_{STP}$ where  $N_{G'_{STP}} = V_T$, then $T$ is a Steiner
tree in $G'_{STP}$. In other words, we can describe the search space of the PCSTP
 as the set of all Steiner trees in a graph $G$.

\begin{figure}[h]\centering
\begin{tikzpicture}[auto, node distance=1.5 cm]
  %Nodes
  \node[terminal, label={12}] (a) {a};
  \node[steiner] (b) [right=of a] {b};
  \node[steiner] (c) [right=of b] {c};
  \node[terminal, label={10}] (d) [above =of c] {d};
  \node[steiner] (e) [left=of d] {e};
  \node[steiner] (f) [left=of e] {f};
  \node[terminal, label={3}] (g) [above right=0.75 and 1.3 of c] {g};
  % Edges
  \begin{scope}[every edge/.style={draw=black, thick}]
    \draw (a) edge[selected] node[below]{4} (b);
    \draw (b) edge[selected] node[near start]{5} (d);
    \draw (b) edge node[below]{8} (c);
    \draw (c) edge node{3} (d);
    \draw (c) edge node{2} (g);
    \draw (c) edge node[near start]{5} (e);
    \draw (d) edge node{6} (e);
    \draw (d) edge node{10} (g);
    \draw (e) edge node{1} (f);
  \end{scope}
\end{tikzpicture}
\caption{Optimal solution to (\ref{fig:pcstp:01})\textnormal{.}}
\label{fig:pcstp:01:opt}
\end{figure}

Unsurprisingly, the PCSTP is an NP-Hard problem. It is a generalisation of the STP. Any instance of
the STP with graph $G = (V, E, c)$ and terminal set $N$ can be reduced to an instance of the PCSTP
with graph $G' = (V, E, c, p)$ where
$$p_v =
\begin{cases}
  \infty & v \in N\mathnormal{,} \\
  0 & \mathnormal{otherwise.}
\end{cases}$$
Any optimal solution to $G'$ is then a Steiner tree in $G$ and vice versa.

\subsection{Variants}

 The PCSTP can stated differently as the {\textit{Net Worth Maximization Problem}} \citep{Johnson:2000:PCS:338219.338637}.
  This involves maximising the function,
$$NW(T) = \sum_{v \in V_T} p_v - \sum_{(i,j) \in E_T} c_{ij}$$
or in other words, maximising the profit of $T$.

While this formulation differs philosophically from the GW optimisation problem
-- i.e. the difference between maximising gain or minimising loss --
the GW minimisation problem and NW maximisation problem are equivalent in the sense that they share optimal solutions trees, $T$.


We can easily show this by seeing that the tree which solves the maximisation problem
$$\max_T \: NW(T) = \max_T \left\{ \sum_{v \in V_T} p_v - \sum_{(i,j) \in E_T} c_{ij} \right\}$$
must be the tree which finds the minimum of its negative,
$$\min_T \left\{ \sum_{(i,j) \in E_T} c_{ij} - \sum_{v \in V_T} p_v\right\}$$
and finally since the total prize in $G$ is independent of $T$, it must also minimise,
\begin{align*}
  &\phantom{=}\min_T \left\{ \sum_{(i,j) \in E_T} c_{ij} - \sum_{v \in V_T} p_v + \sum_{v \in V} p_v \right\} \\
  &=\min_T \left\{ \sum_{(i,j) \in E_T} c_{ij} + \sum_{v \in (V \setminus V_T)} p_v \right\} \\
  &= \min_T \: GW(T)\mathnormal{.}
\end{align*}

\paragraph{Budget and Quota Problems}

Other close related problems can be generated by removing either edge costs or prize collection from the
objective function and instead representing them as constraints. This gives two new optimisation problems
which we call the \textit{Quota} and \textit{Budget} problems \citep{Johnson:2000:PCS:338219.338637}.

In the case of the former, we are looking to minimize the total edge cost of the solution tree, $T$, while
collecting \textit{at least} some quota, $P \in \RR$, of prize, that is we wish to minimise,
$$\sum_{e \in E_T} c_e \quad \text{s.t.} \: \sum_{v \in V_T} p_v \geq P\mathnormal{.}$$
As with the PCSTP, the Quota problem is NP-hard by reduction from the STP. Given any instance of the STP,
we can create an equivalent Quota problem by setting prizes
$$p_v =
\begin{cases}
  1 & v \in N \\
  0 & v \in V \setminus N
\end{cases}
$$
and then setting the quota to be the number of terminals, i.e. $P = |N|$. Then all feasible solutions
must span all terminals, and an optimal solution to the STP is an optimal solution to the quota problem
 and vice versa.

Similarly, the Budget problem involves collecting as much prize as possible without allowing the
 total edge cost of $T$ to increase beyond some budget, $B \in \RR$, this is we wish to maximise,
 $$\sum_{v \in V_T} p_v \quad \text{s.t.} \: \sum_{e \in E_T} c_e \leq B\mathnormal{.}$$
 The budget problem is similar in spirit to the well-known Knapsack problem. In fact we can
 reduce instances of the Knapsack problem to instances of the budget problem, which shows that
  the latter is also NP-hard.

 Given a set of items $I = 1,...,n$, their weights $w_1,...,w_n$, their values $v_1,...,v_n$, and
 a weight budget $B$, we
 can construct an instance of the budget problem by creating a vertex for each item with weight
 $v_i \leq B$ with prize $p_i = v_i$ and connecting them all to an auxiliary vertex, $0$, with
 edges which have cost $c_{0i} = w_i$. Figure \ref{fig:pcstp:budget:knapsack} shows this for
  a small instance of the Knapsack problem with four items.

 \begin{figure}[h!]\centering
\begin{tikzpicture}[auto]
  \node[steiner, label={[inner sep=0.5]75:{$0$}}] (mid) {$0$};
  \node[terminal, label={$v_1$}] (v1) [above=of mid] {$1$};
  \node[terminal, label={$v_2$}] (v2) [right=of mid]{$2$};
  \node[terminal, label=left:{$v_3$}] (v3) [below=of mid]{$3$};
  \node[terminal, label={$v_4$}] (v4) [left=of mid]{$4$};
  \begin{scope}[every node/.style={circle, inner sep=.7}]
  \draw (mid) edge node {$w_1$} (v1);
  \draw (mid) edge node {$w_2$} (v2);
  \draw (mid) edge node {$w_3$} (v3);
  \draw (mid) edge node {$w_4$} (v4);
\end{scope}
\end{tikzpicture}
\caption{A small instance of the Knapsack Problem as an instance of the Budget Problem.}
\label{fig:pcstp:budget:knapsack}
\end{figure}

\paragraph{Fractional PCSTP}


\subsection{ILP Formulations}
%%% Local Variables:
%%% TeX-master: "report"
%%% reftex-default-bibliography: ("lit.bib")
%%% End:
