
\chapter{The Median Tree Problem}
\label{chap:mediantree}

\section{Problem Definition}

Let $G = (V, E, c, d)$ be an undirected graph. Denote $c : E \to \RR^+$ as an \textit{edge cost} function
and $d : V \times V  \to \RR^+$ be an \textit{assignment cost} function where we have
$$d_{ii} = 0 \mathnormal{.}$$
Then the \textit{Median Tree Problem}
is defined as finding a \textit{connected subgraph} $T = (V_T, E_T)$ of $G$
where $V_T \subseteq V$ and
$E_T \subseteq E$ which minimises the cost function,
$$c(T) = \sum_{ij \in E_T} c_{ij} + \sum_{i \in V} \min_{j \in V_T} d_{ij}\mathnormal{.}$$
We say that such a subgraph is a \textit{Median Tree} of $G$.

\begin{figure}[h]\centering
  \begin{subfigure}[b]{0.47\linewidth}\centering
    \begin{tikzpicture}[auto, node distance=1.5 cm]
      % Nodes
      \node[terminal] (a) {a};
      \node[terminal] (b) [right=of a] {b};
      \node[terminal] (c) [right=of b] {c};
      \node[terminal] (d) [above =of c] {d};
      \node[terminal] (e) [left=of d] {e};
      \node[terminal] (f) [left=of e] {f};
      \node[terminal] (g) [above right=0.75 and 1.3 of c] {g};
      % Edges
      \begin{scope}[every edge/.style={draw=black, thick}]
        \draw (a) edge node[below]{4} (b);
        \draw (b) edge node[near start]{5} (d);
        \draw (b) edge node[below]{8} (c);
        \draw (c) edge node{3} (d);
        \draw (c) edge node{2} (g);
        \draw (c) edge node[near start]{5} (e);
        \draw (d) edge node{6} (e);
        \draw (d) edge node{10} (g);
        \draw (e) edge node{1} (f);
      \end{scope}
    \end{tikzpicture}
    \caption{The graph $G$ with edge costs.}
  \end{subfigure}
  \begin{subfigure}[b]{0.47\linewidth}\centering
    \begin{tabular}{r|c|c|c|c|c|c|c}
 $d_{ij}$ & $a$ & $b$ & $c$ & $d$ & $e$ & $f$ & $g$ \\ \hline
      $a$ &  0  &  5  &  8  &     &     &  6  &     \\ \hline
      $b$ &  2  &  0  &  2  &  2  &     &     &     \\ \hline
      $c$ &     &     &  0  & 10  &  4  &     &  1  \\ \hline
      $d$ &     &  3  &     &  0  &  6  &     &  2  \\ \hline
      $e$ &     &     &     &     &  0  &     &     \\ \hline
      $f$ &  2  &  8  &     &     &  5  &  0  &     \\ \hline
      $g$ &     &     &  5  &  5  &     &     &  0  
    \end{tabular}
    \caption{The distance function $d$.}
    \end{subfigure}
  \caption{Instance of the MTP problem.}
  \label{fig:mtp:01}
\end{figure}

\begin{figure}[h!]
  \centering
      \begin{tikzpicture}[auto, node distance=1.5 cm]
      % Nodes
      \node[terminal] (a) {a};
      \node[terminal] (b) [right=of a] {b};
      \node[terminal] (c) [right=of b] {c};
      \node[terminal] (d) [above =of c] {d};
      \node[terminal] (e) [left=of d] {e};
      \node[terminal] (f) [left=of e] {f};
      \node[terminal] (g) [above right=0.75 and 1.3 of c] {g};
      % Edges
      \begin{scope}[every edge/.style={draw=black, thick}]
        \draw (a) edge node[below]{4} (b);
        \draw (b) edge node[near start]{5} (d);
        \draw (b) edge node[below]{8} (c);
        \draw (c) edge node{3} (d);
        \draw (c) edge[selected] node{2} (g);
        \draw (c) edge[selected] node[near start]{5} (e);
        \draw (d) edge node{6} (e);
        \draw (d) edge node{10} (g);
        \draw (e) edge[selected] node{1} (f);
      \end{scope}
      \begin{scope}[every edge/.style={draw=red, thin, opacity=0.6}]
        \draw[->] (a) edge node{6} (f);
        \draw[->] (b) edge[bend right=60] node[below]{2} (c);
        \draw[->] (d) edge[bend left=60] node{2} (g);
      \end{scope}
    \end{tikzpicture}

  \caption{Sol}
  \label{fig:mtp:01:opt}
\end{figure}
\todo[inline]{Add a figure... figure out how to show assignment costs}
\todo[inline]{Maybe define terms Nonterminals and Supply points for vertices which
  respectively have either no assignment costs associated with them, are free to assign to any facility node,
  and are}

\paragraph{Reduction from the Prize-Collecting Steiner Tree Problem}

\section{Applications}
\textit{* Any kind of supply situation: military / water (piping) / aid / maybe rail / postal}
\section{Integer Programming Formulation}

Given an instance of the Median Tree Problem as defined above,
Formulation (\ref{form:mtp:cut}) is an ILP formulation of the Median Tree Problem.
It is loosely inspired by the one defined by \citet{lucena2004strong}
(Formulation \ref{form:lower:gsec}, Section \ref{sec:lower:gsec}) for the Prize-Collecting
Steiner Tree Problem.

The variables $\bd x$ and $\bd y$ are boolean
decision vectors which are interpreted as follows.
As with the PCSTP formulation, when $x_{ij} = 1$,
the edge between vertices $v_i$ and $v_j$ is part of the solution -- note that only
one of $x_{ij}$ and $x_{ji}$ are variables in model and we will use both indices interchangably
to refer to the same variable for the sake of readability.

Similarly, $\bd y$ describes the assignment relation of a solution. When
$y_{ij} = 1$, vertex $v_i$ is \textit{assigned} to vertex $v_j$ and the corresponding
assignment cost must be paid. 
These relations are reflected in the objective function.
When $y_{kk} = 1$, we consider that vertex assigned to
itself, which implies that vertex $v_k$ is part of the facility.

 \begin{formulation}[h!]
   \begin{subequations}
     \begin{alignat}{3} 
       &\underset{\bd x, \bd y}{\text{minimize}}
       & & \sum_{ij \in E} c_{ij} x_{ij} +  \sum_{i, j \in V} d_{ij}y_{ij}  & \\
       & \text{subject to}\quad
       & & \sum_{ij \in E} x_{ij} = \sum_{i \in V} y_{ii} - 1 &&  \label{form:mtp:tree}\\
       &&& x(E(S)) \leq \sum_{i \in S \setminus \{s\}} y_{ii}
       && \forall S \subseteq V, s \in S \label{form:mtp:gsec} \\
       &&& \sum_{j \in V} y_{kj} = 1 && \forall k \in V \label{form:mtp:assignment}\\
       &&& y_{ik} \leq  y_{kk}
       && \forall i, k \in V \label{form:mtp:facility}\\
       &&& y_{kk} \leq \sum_{i \in \delta(k)} x_{ik}
       && \forall k \in V \label{form:mtp:legal} \\
       &&& \bd x \in \BB^{|E|} && \\
       &&& \bd y \in \BB^{|V \times V|}
     \end{alignat}\label{form:mtp:cut}
   \end{subequations}
   \caption{TBD}
 \end{formulation}

 Since edges have nonnegative weights, we know that the shape of the facility must be a tree.
 Constraint (\ref{form:mtp:tree}) ensures that any facility has exactly one less edge than
 vertices, and constraints (\ref{form:mtp:gsec}) -- commonly known as Generalised Subtour
 Elimination Contraints -- ensures that no subset of vertices in the facility can form
 a cycle. Together, these ensure that any feasible
 solution must be a single, connected subgraph of $G$
 which is shaped as a tree.

 The rest of the constraints are concerned with making sure that every vertex in $G$ is
 assigned to a vertex in the facility in any feasible solution. Constraint
 (\ref{form:mtp:assignment}) ensures that every vertex in $G$ is assigned to exactly one
 vertex, and constraint (\ref{form:mtp:facility}) ensures that a vertex can only be assigned
 to if it is part of the facility. This could alternatively be modelled as
 $$\sum_{i \in V} y_{ik} \leq  M y_{kk}  \qquad \forall i \in V$$
 for some $M \gg |V|$.

 Finally, the constraints (\ref{form:mtp:legal}) ensure that no vertex can feasibly be part
 of the facility unless an edge in its adjacency is selected -- that is, unless it is spanned
 by the solution tree.

 \paragraph{Valid Inequalities}
 Whenever a vertex is connected to the facility, it trivially must be assigned to itself. This
 follows directly from the fact that $d_{ii} = 0$ for all $i \in V$ and thus have $d_{ij} \geq d_{ii}$
 for all $j \in V$. The fact that this is implied by the probe (i.e. it is always correct to assign a
 vertex in the facility to itself) allows us to not formalise it in the problem. However, we can
 still add the constraints
 \begin{equation}
 y_{ii} \geq x_{ji} \qquad \forall i \in V,  \forall j \in \delta(i)
\end{equation}
 to as valid constraints to formalise this relation.

 Similar to how \citeauthor{ljubic2005solving} constrain the degree of Nonterminals in Constraint
 (\ref{form:exact:strength}), the same idea holds for the MTP but in a slightly different manner.
 Consider the set of vertices which cost nothing to assign and cannot be assigned to as \textit{Nonterminals},
 $$N = \left\{ i \in V \middle\vert \sum_{j \in v} d_{ij} = 0 \wedge  \forall j. d_{ji} = \infty\right\}\mathnormal{.}$$
 Any nonterminal cannot be a leaf in an optimal solution, and thus must have degree of at least
 two should they be part of an optimal facility. Hence it is valid to add the constraint
 \begin{equation}\label{mtp:valid:deg}
   \sum_{j \in \delta(i)}x_{ij} \geq 2 x_{ik} \qquad \forall i \in N, \: \forall k \in \delta(i)
 \end{equation}
 to the model. The intuitive interpretation of (\ref{mtp:valid:deg}) is that if any edge adjacent to
 a nonterminal is selected, \textit{at least} two edges adjacent to the nonterminal must be selected.
\section{Algorithm and Implementation}

\paragraph{Branch and Cut}

\paragraph{Separation of GSECs} \textit{Max flow method. From Lucena/Resende.}

\paragraph{Primal Heuristics}


\textit{Select vertex subset to be part of the facility from LP
  relaxation after separation (perhaps randomness)
  -- apply MST. Similar to Ljubic.}


\section{Computational Experience}

\subsection{Datasets}

\subsection{Results}

\section{Conclusion}

%%% Local Variables:
%%% TeX-master: "report"
%%% reftex-default-bibliography: ("lit.bib")
%%% End:
