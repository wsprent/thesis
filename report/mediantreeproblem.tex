
\chapter{The Median Tree Problem}
\label{chap:mediantree}


\section{Problem Definition}

Let $G = (V, E, c, d)$ be an undirected graph. Denote $c : E \to \RR^+$ as an \textit{edge cost} function
and $d : V \times V  \to \RR^+$ be an \textit{assignment cost} function where we have
$$d_{ii} = 0 \mathnormal{.}$$
Then the \textit{Median Tree Problem}
is defined as finding a \textit{connected subgraph} $T = (V_T, E_T)$ of $G$
where $V_T \subseteq V$ and
$E_T \subseteq E$ which minimises the cost function,
$$c(T) = \sum_{ij \in E_T} c_{ij} + \sum_{i \in V} \min_{j \in V_T} d_{ij}\mathnormal{.}$$
We say that such a subgraph is a \textit{Median Tree} of $G$.


\section{Applications}

\section{Integer Programming Formulation}

Given an instance of the Median Tree Problem as defined above,
Formulation (\ref{form:mtp:cut}) is an ILP formulation of the Median Tree Problem.
It is loosely inspired by the one defined by \citet{lucena2004strong}
(Formulation \ref{form:lower:gsec}, Section \ref{sec:lower:gsec}) for the Prize-Collecting
Steiner Tree Problem.

The variables $\bd x$ and $\bd y$ are boolean
decision vectors which are interpreted as follows.
As with the PCSTP formulation, when $x_{ij} = 1$,
the edge between vertices $v_i$ and $v_j$ is part of the solution.
Similarly, $\bd y$ describes the assignment relation of a solution. When
$y_{ij} = 1$, vertex $v_i$ is \textit{assigned} to vertex $v_j$ and the corresponding
assignment cost must be paid. 
These relations are reflected in the objective function.
When $y_{kk} = 1$, we consider that vertex assigned to
itself, which implies that vertex $v_k$ is part of the facility.

 \begin{formulation}[h!]
   \begin{subequations}
     \begin{alignat}{3} 
       &\underset{\bd x, \bd y}{\text{minimize}}
       & & \sum_{ij \in E} c_{ij} x_{ij} +  \sum_{i, j \in V} d_{ij}y_{ij}  & \\
       & \text{subject to}\quad
       & & \sum_{ij \in E} x_{ij} - 1 = \sum_{i \in V} y_{ii} &&  \label{form:mtp:tree}\\
       &&& x(E(S)) \leq \sum_{i \in S \setminus \{s\}} y_{ii}
       && \forall S \subseteq V, s \in S \label{form:mtp:gsec} \\
       &&& \sum_{j \in V} y_{kj} = 1 && \forall k \in V \label{form:mtp:assignment}\\
       &&& y_{ik} \leq  y_{kk}
       && \forall i, k \in V \label{form:mtp:facility}\\
       &&& y_{kk} \leq \sum_{i \in \delta(k)} x_{ik}
       && \forall k \in V \label{form:mtp:legal} \\
       &&& \bd x \in \BB^{|E|} && \\
       &&& \bd y \in \BB^{|V \times V|}
     \end{alignat}\label{form:mtp:cut}
   \end{subequations}
   \caption{TBD}
 \end{formulation}

 Since edges have nonnegative weights, we know that the shape of the facility must be a tree.
 Constraint (\ref{form:mtp:tree}) ensures that any facility has exactly one less edge than
 vertices, and constraints (\ref{form:mtp:gsec}) -- commonly known as Generalised Subtour
 Elimination Contraints -- ensures that no subset of vertices in the facility can form
 a cycle. Together, these ensure that any feasible
 solution must be a single, connected subgraph of $G$
 which is shaped as a tree.

 The rest of the constraints are concerned with making sure that every vertex in $G$ is
 assigned to a vertex in the facility in any feasible solution. Constraint
 (\ref{form:mtp:assignment}) ensures that every vertex in $G$ is assigned to exactly one
 vertex, and constraint (\ref{form:mtp:facility}) ensures that a vertex can only be assigned
 to if it is part of the facility. This could alternatively be modelled as
 $$\sum_{i \in V} y_{ik} \leq  M y_{kk}  \qquad \forall i \in V$$
 for some $M \gg |V|$.

 Finally, the constraints (\ref{form:mtp:legal}) ensure that no vertex can feasibly be part
 of the facility unless an edge in its adjacency is selected -- that is, unless it is spanned
 by the solution tree.

 \todo[inline]{Perhaps strengthen the formulation.}
\section{Algorithm and Implementation}

\section{Computational Experience}

\subsection{Instances}

\subsection{Results}

\section{Conclusion}

%%% Local Variables:
%%% TeX-master: "report"
%%% reftex-default-bibliography: ("lit.bib")
%%% End:
