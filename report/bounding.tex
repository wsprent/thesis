\section{Lower Bounding Procedures}
\label{sec:solving:lower}

\subsection{Generalised Subtour Elimination Constraints}
\label{sec:lower:gsec}
\citet{lucena2004strong} introduced a method for producing lower bounds for the PCSTP based on solving
the LP relaxations of the PCSTP using a constraint separation procedure.

Their main contributions
are the stating of a \textit{generalised subtour elimination constraint} ILP (see Formulation \ref{form:lower:gsec}) formulation
for the PCSTP, the separation procedure for solving the LP-relaxation, and strengthening of the formulation by considering
single vertex solution seperately.

Define the function
$$E(S) = \{(i,j) \in E \mid i \in S \wedge j \in S\}$$
then Formulation (\ref{form:lower:gsec}) is a ILP formulation of the PCSTP. Constraints (\ref{form:lower:gsec:gsec}) are
referred to as generalised subtour elimination constraints, and together with constraint (\ref{form:lower:gsec:sum})
ensure that the solution defined by $(x, y)$ is a tree which spans the vertices $y_v = 1$.

 \begin{formulation}[h!]
   \begin{subequations}
     \begin{alignat}{3} 
       &\underset{x,y}{\text{minimize}}
       & & \sum_{e \in E} c_e x_e +  \sum_{v \in V} p_v (1 - y_v)  & \\
       & \text{subject to}\quad
       & & x(E) = y(V) - 1 &&  \label{form:lower:gsec:sum}\\
       &&& x(E(S)) \leq y(S \setminus \{s\}) && \forall s \in S, S \subseteq V \label{form:lower:gsec:gsec}\\
       \intertext{ILP:}
       &&& x_e \in \BB  && \forall e \in E \\
       &&& y_v \in \BB  && \forall v \in V \\
       \intertext{LP:}
       &&& 0 \leq x_e \leq 1  && \forall e \in E \\
       &&& 0 \leq y_v \leq 1  && \forall v \in V \\
       \intertext{To exclude single vertex solutions:}
       &&& \sum_{i \in \delta(v)} x_{vi} \geq
       \begin{cases}
         y_v & p_v > 0 \\
         2y_v & p_v = 0
       \end{cases} \qquad && \forall v \in V \label{form:lower:gsec:single}
     \end{alignat}\label{form:lower:gsec}
   \end{subequations}
   \caption{(GSEC-IP) GSEC formulation of the PCSTP.}
 \end{formulation}

 Expanding the formulation with constraint (\ref{form:lower:gsec:single}), excludes single vertex solutions.
 \citet{lucena2004strong} found that adding these constraints and
 considering single vertex solutions in a separate routine results faster runtimes.

 \paragraph{Separation of GSECs} Each subset $S \in V$ produces $|S|$ GSECs. To reduce the size of the LP relaxation,
 \citet{lucena2004strong} suggest relaxing these constraints and readding them in a separation procedure.
 This separation procedure can be stated a max-flow problem as follows:

 Let $(\bar{x}, \bar{y})$ be a solution to the relaxed LP. Consider the support graph induced by vertices and edges with nonzero decision veriables,
  and vertex $l$ with decision variable $\bar y_l > 0$
  then iff
  $$\max_{S_l \subseteq V} \left\{ \bar x (E(S_l)) - (\bar y (S_l) - \bar y_l) \right\} > 0$$
  there is a violated GSEC for a subset containing $l$, and constraint corresponding to $S_l^{max}$ is the most violated constraint.

  We can model finding this subset for a given $l$ as follows. Let $\bar{N} = (\bar{V}, \bar{A})$ be a flow-network
  for the support graph, with vertices
  $$\bar V = \{s, t\} \cup \{v \in V \mid \bar y_v > 0 \}$$
  and arcs
  $$\bar A = \{(i, j) \in E \mid \bar x_{ij} > 0 \} \cup \{(j, i) \in E \mid \bar x_{ij} > 0\} \cup \{(s, v) \mid v \in \bar V\} \cup \{(v, t) \mid v \in \bar V\}\mathnormal{.}$$
  Additionally we have capacities original edges,
  $$\xi_{ij} = \xi_{ji} = \frac{\bar x_{ij}}{2} \qquad \forall (i,j) \in \bar A$$
  and if we let $\delta (v)$ be all vertices adjacent to $v$ in the support graph then node capacities are
  $$\xi_v = \sum_{u \in \delta(v)} \xi_{vu}\mathnormal{,}$$
  and finally
  for arcs from and to the source and the sink, we have
  $$\xi_{sv} =
  \begin{cases}
    \infty & v = l \\
    \max(\xi_v - \bar y_v, 0) & \text{otherwise}
  \end{cases}$$
  and
  $$\xi_{vt} = \max(\bar y - \xi_v, 0)\mathnormal{.}$$
  Then, similarly to \citet{padberg1983trees}, we can see that
  the minimum $(S,T)$ cut on the flow network will minimise
  \begin{align*}
    c(S,T) = &\sum_{v \in S} \max \left\{\bar y_v - \xi_v, 0\right\} +
               \sum_{i \in S, j \in T} \xi_{ij} +
               \sum_{v \in T} \max\left\{\xi_v - \bar y_v, 0\right\} \\
    =&\left(\sum_{v \in S} \max \left\{\bar y_v - \xi_v, 0\right\} - \sum_{v \in S} \max\left\{\xi_v - \bar y_v, 0\right\} \right) +
v       \sum_{i \in S, j \in T}  \xi_{ij} +
       \sum_{v \in V} \max\left\{\xi_v - \bar y_v, 0\right\} \\
    =&\sum_{v \in S} \bar y_v - \xi_v +
       \sum_{i \in S, j \in T} \xi_{ij} +
       \sum_{v \in V} \max\left\{\xi_v - \bar y_v, 0\right\} \\
    =&\sum_{v \in S} \bar y_v -
       \sum_{v \in S} \sum_{u \in \delta(v)} \frac{\bar x_{vu}}{2} +
       \sum_{i \in S, j \in T} \frac{\bar x_{ij}}{2} +
       \sum_{v \in V} \max\left\{\xi_v - \bar y_v, 0\right\} \\
    =&\sum_{v \in S} \bar y_v -
       \left(\sum_{i \in S} \sum_{j \in S \setminus \{i\}} \frac{\bar x_{ij}}{2} +
       \sum_{i \in S, j \in T} \frac{\bar x_{ij}}{2}\right) +
       \sum_{i \in S, j \in T} \frac{\bar x_{ij}}{2} +
       \sum_{v \in V} \max\left\{\xi_v - \bar y_v, 0\right\} \\
    =&\sum_{v \in S} \bar y_v -
       \bar x (E(S)) +
       \sum_{v \in V} \max\left\{\xi_v - \bar y_v, 0\right\}\mathnormal{.}
  \end{align*}
  Since $\sum_{v \in V} \max\left\{\xi_v - \bar y_v, 0\right\}$ is independent of
  any cuts, and
  $l$ \textit{must} be in $S$ due to $\xi_{sy} = \infty$, any minimum cut, $(S,T)$, will maximise
  $$\bar x (E(S_l)) - (\bar y (S_l) - \bar y_l)\mathnormal{.}$$
  Hence any set of vertices $S' = S \setminus \{s\}$ where $S$ induces a minimal $S,T$ cut in the flow network will correspond to
  a most violated GSEC for the solution $(\bar x, \bar y)$ to the relaxed LP.

  Finding violated GSECs then reduces to solving $|V|$ maximum flow problems, which can be done in polynomial time. However, there is guarantee
  that the $|V|$ problems will result in different violated constraints.
  To avoid adding finding the same constraints multiple times, \citet{lucena2004strong} suggest that whenever a max flow problem has been
  solved for some $l = v \in V$, then we can set $\xi_{lt} = \infty$ to avoid $v$ being on the source-side of any subsequent min-cuts.
  \todo[inline]{Wolsey (I think) suggests deleting edges/vertices from the support graph, maybe mention that}

%%% Local Variables:
%%% TeX-master: "report"
%%% reftex-default-bibliography: ("lit.bib")
%%% End:
