\makeatletter
\def\toclevel@chapter{-1}
\makeatother
\chapter{Conclusions}
This chapter serves as a summary of the work we have presented in this thesis.
Section~\ref{sec:con:fut} looks at some of the work which we think would be
interesting to explore, but have not either due to the scoping of our thesis or
time constraints.
Section~\ref{sec:con:con} then concludes this thesis with a look back at what we
have achieved (and not achieved) and presents what we think are our main results.
\section{Future Work}\label{sec:con:fut}
\textit{DiGraph transformation of \gls{mtp}.}
\textit{Produce real world datasets / datasets which represent different flavours of mtp}
\textit{Develop a strong heuristic for the MTP}
\textit{Reimplement the solver with system programming callbacks}

\section{Conclusion}\label{sec:con:con}
In this thesis, we have surveyed the state of research on
an NP-hard combinatorial graph optimisation problem in
the \acrlong{pcstp}. We have given a detailed overview of state-of-the-art
methods used to solve this problem.
With this knowledge in hand, we have then implemented a solver for the related
\acrlong{mtp}, a generalisation of the \gls{pcstp}.

We have learned that the gls{pcstp}
is a well-research problem, and stands out as the most mature field of
research among related ``prize-collecting'' problems.
We have seen that research on the \gls{pcstp} is heavily based on corresponding
research on the \gls{stp}.

Furthermore, we have seen that some of the methods
---particularly within preprocessing---
are very sepecific to the \gls{pcstp} as they lean heavily on
invariants inherent in the problem itself. Thus, while preprocessing routines
are key results ---and widely used--- within the realm of the \gls{pcstp}
they are not easily transferable to other problems, if transferable at all.

Other areas with research on the \gls{pcstp} lend themselves much more to such
translations. We have shown this, by applying methods
with a varying degree of success for the
\gls{pcstp} in our solver for the \gls{mtp}.

While we have not seen great results for the methods we have translated
---oftentimes showing that default methods to our \gls{mip} being superior---
we have no reason to believe that this is due to the methods not being
translatable. Particularly, we have seen that our primal heuristic implemented
in the relatively slow Python3 interface gives ambivalent results in our experimental
section. This suggests that a better implementation may eclipse the default heuristic
altoghether.

% Main contributions
In conclusion, we have made four main contributions,
\begin{itemize}
\item A survey on the \gls{pcstp}. This survey serves as a comprehensive introduction
  and overview of research on the \gls{pcstp} and gives insight into the details of
  methods used to solve the \gls{pcstp}. Futhermore, for methods we do not cover
  in a detailed manner, it serves as an index for further reading.
\item An introduction to the \acrlong{mtp}, a NP-hard optimisation problem along with
  a problem defintion in the vein of the family of ``prize-collecting'' problems.
  Additionally, we have formulated the problem as an \gls{ilp} problem, using \glspl{gsec}.
\item A new small dataset for the \gls{mtp} which can be used to compare results, as well
  as tools to convert STP format data sets for the \gls{pcstp}. These tools can be used
  \textit{as-is} or modified to generate different kinds of instances.
\item A solver for the \gls{mtp}. This solver is able to solve the truncated version of
  our dataset to optimality, but not the full sized version within reasonable time.
\end{itemize}

Through this, we have explored and learned about the subject of
NP-hard graph optimisation problems from
the perspective of the \acrlong{pcstp}.



%%% Local Variables:
%%% TeX-master: "report"
%%% reftex-default-bibliography: ("lit.bib")
%%% End:
