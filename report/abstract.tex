The \gls{pcstp} in graphs is an NP-hard optimisation problem which involves
finding a connected subgraph which minimises the total sum of edge costs of
the subgraph and sum of vertex prizes for vertices not spanned by the subgraph.
It is a very well researched problem --- not least due to it being one of the
subjects of the \textit{11th DIMACS Implementation Challenge}.

In this master's thesis, we present a comprehensive survey of the research landscape
on the \gls{pcstp}. This involves a thorough look at some of the most interesting
and important algorithms for the \gls{pcstp}. Among the surveyed subjects are
powerful preprocessing routines, the widely cited Goemans-Williamson approximation
algorithm, and two state-of-the-art solvers.

We make use of this survey to consider the relation of the \gls{pcstp} other close
 problems, and take a new look at the younger and much less researched problem, the
\gls{mtp}. This problem is a generalisation of the \gls{pcstp} where instead of paying
static vertex prizes for unspanned vertices, a cost is now paid to \textit{assign}
unspanned vertices to the result subgraph.

For the \gls{mtp} we present a new \acrlong{ilp} formulation as well as a solver which
is able to generate exact solutions to the problem. Furthermore, we present a new
dataset based on previously released datasets for the \gls{pcstp}.

%%% Local Variables:
%%% TeX-master: "report"
%%% reftex-default-bibliography: ("lit.bib")
%%% End:
