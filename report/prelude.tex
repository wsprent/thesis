\usepackage[english]{babel}
\usepackage[utf8]{inputenc}
\usepackage[T1]{fontenc}
\usepackage[british]{isodate}

\usepackage[usenames]{color}

\usepackage{graphicx}
\usepackage{syntax}

\usepackage{listings}
\lstset{
  basicstyle=\ttfamily, % Global Code Style
  captionpos=b, % Position of the Caption (t for top, b for bottom)
  extendedchars=true, % Allows 256 instead of 128 ASCII characters
  tabsize=4, % number of spaces indented when discovering a tab
  columns=fixed, % make all characters equal width
  keepspaces=true, % does not ignore spaces to fit width, convert tabs to spaces
  showstringspaces=false, % lets spaces in strings appear as real spaces
  breaklines=true, % wrap lines if they don't fit
  frame=trbl, % draw a frame at the top, right, left and bottom of the listing
  frameround=tttt, % make the frame round at all four corners
  framesep=4pt, % quarter circle size of the round corners
  numbers=left, % show line numbers at the left
  numberstyle=\small\ttfamily, % style of the line numbers
  commentstyle=\slshape\bfseries\color{green}, % style of comments
  keywordstyle=\bfseries\color{purple}, % style of keywords
  stringstyle=\color{blue}, % style of strings
  emph=[1] {
    END,
    SECTION,
    E,
    D,
    Nodes,
    Edges
  },
  emphstyle=[1]{\color{blue}},
  moredelim=**[is][\color{red}]{@@}{@@}
}

% TOC
\renewcommand{\thesubsubsection}{\thesubsection.\alph{subsubsection}}


\setcounter{tocdepth}{3}
\setcounter{secnumdepth}{3}
% \usepackage{parskip}

% Margins
\usepackage[hmargin={55pt, 55pt}, vmargin={2.8cm, 2.8cm}]{geometry}

%%% MATH
\usepackage{amsmath}
\usepackage{amssymb}
\usepackage{amsbsy}
\usepackage{amsthm}                % better theorem environments
\usepackage{mathtools}
\usepackage{algorithm}
\usepackage{algpseudocode}
\algnewcommand\algorithmicinput{\textbf{Input:}}
\algnewcommand\Input{\item[\algorithmicinput]}
\algnewcommand\algorithmicoutput{\textbf{Output:}}
\algnewcommand\Output{\item[\algorithmicoutput]}

% various theorems, numbered by section

\newtheorem{thm}{Theorem}[section]
\newtheorem{lem}[thm]{Lemma}
\newtheorem{prop}[thm]{Proposition}
\newtheorem{cor}[thm]{Corollary}
\newtheorem{conj}[thm]{Conjecture}

\newcommand{\nn}{\mathcal{N}}
\newcommand{\uu}{\mathcal{U}}
\newcommand{\bb}{\mathcal{B}}
\newcommand{\ee}{\mathrm{e}}
\newcommand{\rd}[1]{\mathrm{#1}}
\newcommand{\bd}[1]{\mathbf{#1}}  % for bolding symbols
\newcommand{\RR}{\mathbb{R}}      % for Real numbers
\newcommand{\ZZ}{\mathbb{Z}}      % for Integers
\newcommand{\BB}{\mathbb{B}}      % for Booleans
\newcommand{\PP}{\mathbb{P}}      % for Prob
\newcommand{\EE}{\mathbb{E}}      % for Expectation
\newcommand{\II}{\mathbbm{1}}      % for Indicator fun
\newcommand{\NN}{\mathbb{N}}      % for Prob
\newcommand{\col}[1]{\left[\begin{matrix} #1 \end{matrix} \right]}
\newcommand{\comb}[2]{\binom{#1^2 + #2^2}{#1+#2}}

\newcommand{\sint}[1]{\shortintertext{#1}}

% Captions
\usepackage[margin=1cm, labelfont=bf]{caption}
\usepackage{subcaption}
\usepackage{hyperref}

\usepackage[inline]{enumitem}

% Citations
\usepackage[numbers]{natbib}

% TikZ
\usepackage{tikz}
\usetikzlibrary{arrows,fit, positioning, shapes,
  calc, fadings, decorations.pathmorphing, hobby}

\tikzset{font=\tiny}
\tikzset{terminal/.style={circle, draw=black, fill=black, text=white}}
\tikzset{steiner/.style={circle, draw=black, fill=white}}
\tikzset{selected/.style={draw=red, ultra thick}}
\tikzset{assignment/.style={draw=red, thin, opacity=0.6}}
\tikzset{root/.style={draw=cyan, ultra thick}}
\tikzset{subgraph/.style={font=\large, text=gray}}
\tikzset{snake it/.style={-stealth,
    decoration={snake, 
      amplitude = .2mm,
    segment length = 2mm,
    post length=0.9mm},decorate}}
\tikzset{snake node/.style={above=1mm,
    midway,
    text width=3cm,
    sloped,
    align=center}}
% Caption types

\DeclareCaptionType{formulation}[Formulation][List of formulations]

% TODO
\usepackage{todonotes}
\presetkeys{todonotes}{size=\tiny}{}

% Theorems
\newtheorem{theorem}{Theorem}[section]
\newtheorem{lemma}{Lemma}[theorem]
\newtheorem{Lemma}{Lemma}[section]
\newtheorem{corollary}{Corollary}[theorem]
\theoremstyle{definition}
\newtheorem{example}{Example}[section]

% Table
\usepackage{multirow}
\usepackage{multicol}
% Ops

\DeclareMathOperator{\id}{id}
\DeclareMathOperator*{\argmin}{argmin}
\DeclareMathOperator*{\argmax}{argmax}
\DeclareMathOperator{\radius}{radius}
\DeclareMathOperator{\pcradius}{pcradius}

% Glossary

\newcommand*{\newdualentry}[5][]{%  
  \newglossaryentry{main-#2}{name={#4},%  
  text={#3\glsadd{#2}},%  
  description={#5},%  
  #1  
  }%  
  \newacronym{#2}{#3\glsadd{main-#2}}{#4}%  
}

\usepackage{glossaries}
\makeglossaries

\newacronym{ilp}{ILP}{integer linear programming}
\newacronym{gsec}{GSEC}{Generalised Subtour Elimination Constraint}
\newacronym{mip}{MIP}{mixed-integer programming}
\newdualentry{pcstp}{PCSTP}
{Prize-Collecting Steiner Tree Problem}
{An NP-Hard graph optimisation problem (\textit{See Section~\ref{sec:pcstp}})}
\newdualentry{pctsp}{PCTSP}
{Prize-Collecting Travelling Salesman Problem}
{An NP-Hard graph optimisation problem (\textit{See Section~\ref{sec:rel:tsp}})}
\newdualentry{ptp}{PTP}{Profitable Tour Problem}
{A problem achieved by removing the minimum prize collection constraint from the
  \acrlong{pctsp} (\textit{See Section~\ref{sec:rel:tsp}})}
\newacronym{stp}{STP}{Steiner Tree Problem}
\newacronym{tsp}{TSP}{Travelling Salesman Problem}
\newacronym{sap}{SAP}{Steiner Arborescence Problem}
\newacronym{pcsap}{PCSAP}{Prize-Collecting Steiner Arborescence Problem}
\newacronym{mtp}{MTP}{Median Tree Problem}

\newglossaryentry{delta}
{
  name={\ensuremath{\delta}},
  description={Function: $V \to \mathcal P (V)$, returns the adjacency of
    the input vertex \textit{(See Section~\ref{sec:intro:notation})}},
  sort=delta
}

\newglossaryentry{realplus}
{
  name={\ensuremath{\RR^+}},
  description={Subset of the Real numbers. Includes the interval $[0, \infty[$ and $\infty$},
  sort=real
}

%%% 
%%% Local Variables:
%%% TeX-master: "report"
%%% reftex-default-bibliography: ("lit.bib")
%%% End:
