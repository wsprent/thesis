
\chapter{Introduction}
\section{Background and Motivation}
\label{sec:intro:background}
There exists a family of combinatorial optimisation problems in graphs which involve
finding a subgraph which optimises the trade-off of including a vertex for its prize
(or to avoid paying a penalty) and paying the cost of the edge(s) needed to connect
to it.

These problems generally come with the moniker of \textit{Prize-Collecting}
and were first introduced by \citet*{balas1989prize}
with the introduction of the \textit{\gls{pctsp}}. This problem is a generalisation of
the famous \gls{tsp}, relaxing its Hamiltonian tour constraint by instead applying penalities
to missed vertices and setting a minimum on collected ``prize''.

Perhaps unsurprisingly, this concept of collecting prize spread to another famour optimisation
problem, that is the \gls{stp}. This was done by \citet{Bienstock1993} in conjuction with
defining an approximation algorithm for the \gls{pctsp}. This variant of the \gls{stp},
aptly named the \gls{pcstp}, gained traction --- particularly with the
\textit{11th DIMACS Implementation Challenge}\citep{DIMACS}.

At the onset of writing this thesis, we started off with the goal of surveying research done
on the \gls{pcstp}, and attempting to fill out any holes with methods defined for the older
\gls{pctsp}. This was considered with the assumption that the older problem had
been more covered.
However, it turned out that the \gls{pcstp} had been \textit{far} more covered
by research with multiple exact algorithms
\citep{ljubic2005solving, leitner2016dual, gamrath2017scip},
approximation algorithms \citep{Bienstock1993,goemans1995general,Johnson:2000:PCS:338219.338637},
and primal heuristics \citep{canuto2001local,fu2014knowledge,akhmedov2016divide}
having been already proposed while the \gls{pctsp} had received little such attention
\citep{archetti2014chapter}.

Still wanting to have the \gls{pcstp} as our starting point, we decided to still survery the
problem with the hope that any knowledge gained could be applied somehow. We found an application
for this when running into the \gls{mtp}. The \gls{mtp} is a generalisation of the \gls{pcstp}
which, to the best of our knowledge, has not been explored in the general case. Thus we set out
to apply knowledge from the \gls{pcstp} to the \gls{mtp} to hopefully learn something about
both problems.
\section{Aim and Scope}
The aim of this thesis is two-fold and split in two parts. In the first part, we perform a
survey of the \acrlong{pctsp} where we report, detail, and bring intution to the most interesting
and/or effective methods for solving the \gls{pcstp}.

In the second part, we widen our lenses and look at related problems from the perspective of
the \gls{pcstp}. First superficially, and then with a deep dive in the \acrlong{mtp} for which
we proprose an \gls{ilp} formulation, a new dataset, and a solver.

\section{Related Work}
Besides the work we relay in this thesis, there has been some recent work which is closely related
 to what we present. Particularly, there has been some in surveying the
\acrlong{pcstp} from different angles.

\citet*{rehfeldt2016reduction} present a comprehensive study of all preprocessing routines
for the \gls{pcstp} and the Node-Weighted Steiner Tree Problem. This is similar in scope to
Section~\ref{sec:solving:pre} in this thesis. The reader is encouraged to visit their paper
if they wish to see a more theoretical and comprehensive look at preprocessing routines for
the \gls{pcstp}. We aim to complement this with a slightly more ituitive description of the
routines.

\citet*{sun2018classical} presents a survey of the practical applications of the \gls{pcstp}.
In their thesis, they first perform a short and sweet summary on the historical perspective of
solving the \gls{pcstp}. Their main constribution, however, is a detailed look at the applications
of the problem in real world scenarios.





\section{Notation}
\label{sec:intro:notation}


%%% Local Variables:
%%% TeX-master: "report"
%%% reftex-default-bibliography: ("lit.bib")
%%% End: