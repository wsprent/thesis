\clearpage
\section{Approximation Algorithms}\label{sec:solving:approx}

\subsection{The Goemans-Williamson Algorithm}\label{sec:solving:approx}
\cite*{goemans1995general} presented a primal-dual 2-approximation algorithm for the rooted variant of the
PCSTP, based on an algorithm for solving the general \textit{constrained forest problem}.

\todo[inline]{Quick summary of usage. It is popular.}

\paragraph{Definitions}
\citeauthor{goemans1995general} stated the ILP (GW-ILP) in Formulation \ref{form:approx:gw} for the rooted PCSTP.

 \begin{formulation}[h!]
   \begin{subequations}
     \begin{alignat}{3} 
       &\underset{x}{\text{minimize}}
       & & \sum_{e \in E} c_e x_e + \sum_{X \subset V; r \not\in X} z_X \left( \sum_{v \in X} p_v \right) & \\
       & \text{subject to}\quad
       & & x(\delta(S)) + \sum_{X \supseteq S}z_X \geq 1 \qquad&& \forall S \subset V; r \not\in S \\
       &&& \sum_{S \subset V; r \not\in S} z_X \leq 1 &&\nonumber\\
       &&& x_e \in \BB  && \forall e \in E \\
       &&& z_X \in \BB  && \forall S \subset V; r \not\in S
     \end{alignat}\label{form:approx:gw}
   \end{subequations}
   \caption{(GW-IP) formulation of the PCSTP.}
 \end{formulation}

 GW-IP has two decision vectors. The decision vector $x$ denotes which edges are contained in the solution, $z_X$ is
 $1$ when all $v \in X$ are not part of a solution. Clearly, any optimal solution will have $z_X = 1$ for just a single
 $X \subset V$.

 \begin{formulation}[h!]
   \begin{subequations}
     \begin{alignat}{3} 
       &\underset{x}{\text{maximize}}
       & & \sum_{S \subset V; r \not\in S} y_S & \\
       & \text{subject to}\quad
       & & \sum_{S: e \in \delta(S)} y_S \leq c_e \qquad&& \forall e \in E \label{form:gw:pe}\\
       &&& \sum_{S \subseteq X} y_S \leq \sum_{v \in X} p_v  && \forall X \subset V; r \not\in X \label{form:gw:pv} \\
       &&& y_S \geq 0  && \forall S \subset V; r \not\in S
     \end{alignat}\label{form:approx:gw:dual}
   \end{subequations}
   \caption{(GW-D): Dual of the LP relaxation of (GW-ILP) from Formulation \ref{form:approx:gw}.}
 \end{formulation}
 

 The dual to the LP relaxation of GW-IP is the linear program in \ref{form:approx:gw:dual}. We make note of the
 \textit{packing} constraints \ref{form:gw:pe} and \ref{form:gw:pv}, and note that by \textit{complementary slackness}
 of the primal LP, we have for optimal primal solutions to the LP, $(x^*, z^*)$,
 $$x^*_e > 0 \Rightarrow \sum_{S: e \in \delta(S)} y_S = c_e$$
 and
 $$z^*_X > 0 \Rightarrow \sum_{S \subseteq X} y_S = \sum_{v \in X} p_v\mathnormal{.}$$
 

 
 \paragraph{The Algorithm} The GW-Algorithm consists of a growing phase and a pruning phase. The growing phase
 builds a feasible solution $T$ by maintaining a set of active and deactive components \textit{components}
 (initially singleton sets which are active if they do not contain the root vertex), and simultaneously increasing
  the dual variable value for all active components until one of the following happen:
 \begin{enumerate}[label=\alph*)]
 \item If a packing constraint (\ref{form:gw:pe}) becomes binding for some edge $e = (i,j) \in E$, then $e$ is added
   to $T$, and the components bridged by $e$ are merged into a component with combined cost.
   The new component is considered active iff. the root vertex is not contained in it.
 \item If a packing constraint (\ref{form:gw:pv}) becomes binding for some subset $X  \subset V$ (i.e. component), then $X$ is deactivated
    and all vertices it contains are marked with $X$.
 \end{enumerate}
 This continues until there are no active components left, at which point $T$ is a feasible solution to the IP (GW-IP).

 
 \begin{algorithm}[h!]
   \begin{algorithmic}[1]
     \Input{Graph $G = (V, E, c, p)$ and root vertex $r \in V$.}
     \Output{Tree $T' \subseteq E$, and the set of vertices not spanned by $T'$, $X$.}
     \Procedure{GW-Algorithm}{}
     \State $T \gets \emptyset$ \label{gw:init-start}
     \State $\mathcal{C} \gets \{\{v\} : v \in V\}$
     \For {$v \in V$}
       \State $d(v) \gets 0$
       \State $w(\{v\}) \gets 0$
       \State Unmark $v$.
       \If {$v = r$}
         \State $\lambda(\{r\}) \gets 0$
       \Else
         \State $\lambda(\{v\}) \gets 1$
       \EndIf
     \EndFor \label{gw:init-end}
     \While{$\{ C \in \mathcal{C} \mid \lambda(C) = 1\} \neq \emptyset$}
     \State $\epsilon_1 \gets \min_{i,j} \frac{c_{ij} - d(i) - d(j)}{\lambda (C_i) + \lambda(C_j)},
     \qquad i \in C_i, j \in C_j, C_i \neq C_j$ \label{gw:select-start}
     \State $\epsilon_2 \gets \min_{w} \sum_{v \in C_w} p_v - w(C_w), \qquad C_w \in \mathcal{C}$
     \State $\epsilon \gets \min(\epsilon_1, \epsilon_2)$ 
     \State $w(C) \gets w(C) + \lambda(C) \epsilon$ for all $C \in \mathcal{C}$ \label{gw:add-component}
     \State $d(v) \gets d(v) + \lambda(C) \epsilon$ for all $v \in C \in \mathcal{C}$ \label{gw:add-vertex}
     \If{$\epsilon = \epsilon_1$}
     \State Add $(i,j)$ to $T$: $T \gets T \cup \{(i,j)\}$
     \State Merge $C_i$ and $C_j$ to $C_{ij}$,
     $\mathcal{C} \gets \mathcal{C} \cup \{C_{ij}\} \setminus \{C_i, C_j\}$
     \State If $r \in C_{ij}$ then $\lambda(C_{ij}) \gets 0$ else $\lambda(C_{ij}) \gets 1$
     \State $w(C_{ij}) \gets w(C_i) + w(C_j)$
     \Else
     \State Deactivate $C_w$, $\lambda(C_w) \gets 0$
     \State Mark all $v \in C_w$ with the label $C_w$
     \EndIf
     \EndWhile
     \State $T' \gets \Call{Prune}{T, Labels}$ 
   \EndProcedure
 \end{algorithmic}
 \caption{The GW-Algorithm}\label{approx:gw:alg}
 \end{algorithm}

 The growing phase of the GW-Algorithm is shown in Algorithm \ref{approx:gw:alg}. Lines \ref{gw:init-start}-\ref{gw:init-end}
 initialise the set of components as described above. Only the component containing the root vertex is set as inactive.
 The lines \ref{gw:select-start}-\ref{gw:add-vertex} then determine the amount, $\epsilon$, which the value of the dual decision vector
 can be increased for active components
 until a packing constraint becomes binding. Since this behaviour is not directly clear for the definition of the algorithm,
  we attempt to give some intuition in the following.

 At the beginning of each iteration we have,
 $$d(v) = \sum_{S: v \in S} y_S, \forall v \in V \mathnormal{.}$$
 This follows directly from line \ref{gw:add-vertex}. Suppose that the corresponding
 packing constraint for the edge $e = (i,j)$ is
  the tightest ``active'' constraint. Notice that we can reformulate the left side of the constraint as follows,
 $$\sum_{S: (i,j) \in \delta(S)} y_S = \sum_{S : i \in S} y_S + \sum_{S : j \in S} y_S - \sum_{S : i,j \in S} y_S\mathnormal{.}$$
 Since that we are only considering edges which bridge components, we clearly must have $\sum_{S : i,j \in S} y_S = 0$
 and thus,
 $$\sum_{S: (i,j) \in \delta(S)} y_S = d(i) + d(j)\mathnormal{.}$$
 Suppose, w.l.o.g., that both $i$ and $j$ belong to active components.
 When the iteration is over, we will have increased $d(i)$ and $d(j)$ by $\epsilon = \frac{c_{ij} - d(i) - d(j)}{2}$.
 The packing constraint for $e = (i,j)$ is then binding,
 $$\sum_{S: (i,j) \in \delta(S)} y_S = d(i) + \left( \frac{c_{ij} - d(i) - d(j)}{2} \right) + d(j) +
 \left( \frac{c_{ij} - d(i) - d(j)}{2} \right)   = c_{ij}\mathnormal{.}$$
 The case where a packing constraint (\ref{form:gw:pv}) is the tightest is similar, and follows directly from the loop
 invariant
 $$w(C) = \sum_{S \subset C} y_S, \forall C \in \mathcal{C}\mathnormal{.}$$

 \cite{goemans1995general} describe the pruning procedure loosely as a procedure which removes as many
 edges from $T$ while upholding that:
 \begin{enumerate}
 \item any unlabeled vertex is connected to the root vertex, $r$, and
 \item if a vertex with label $C$ is connected to $r$ then any vertex with
   label $C' \supset C$ must also be connected $r$.
 \end{enumerate}
 The pruning procedure was improved upon by the ``strong pruning'' procedure introduced by \cite{Johnson:2000:PCS:338219.338637},
  which we will detail in Section \ref{sec:approx:strongpruning}.
 \paragraph{Analysis}
 The GW-Algorithm is proven to produce a result which is less than twice as bad as an optimal solution to (GW-IP), $Z^*_{\text{GW-IP}}$, and thus
 is a 2-approximation algorithm. This follows from the following inequality,
 $$\sum_{e \in T'}c_e + \sum_{v \in X} p_v \overset{\text{(I)}}{=}   \sum_{e \in T'} \sum_{S: e \in \delta(S)} y_S  + \sum_{j} \sum_{S \subseteq C_j} y_S
 \overset{\text{(II)}}{\leq} (2 - \frac{1}{n-1}) \sum_{S \subset V} y_S \overset{\text{(III)}}{\leq} (2 - \frac{1}{n-1}) Z^*_{\text{GW-IP}}\mathnormal{.}$$
 The reader is encouraged to read \cite{goemans1997primal} for the full proof, but it roughly goes as follows:
 \begin{itemize}
 \item (I) holds from primal complimentary slackness. An edge is only added to $T$ if its corresponding packing constraint is binding,
   and for any vertex to not be spanned by $T$, it must have been deactivated (and thus the corresponding packing constraint must have
    been binding).
  \item (II) holds from induction an induction proof on main loop of the procedure, and
  \item (III) holds from the feasibility of the $y$ vector implicitly procduced by the procedure and weak duality.
 \end{itemize}
 Furthermore, the GW-Algorithm is proven to run in $O(n^2 \log n)$.
 \subsection{Modifications to the GW-Algorithm}\label{sec:approx:strongpruning}
 \cite{Johnson:2000:PCS:338219.338637} investigated three main modifications for the GW-Algorithm:
 \begin{enumerate}
 \item an improved pruning procedure called strong pruning,
 \item an unrooted version of the algorithm, and
 \item prize pertubation.
 \end{enumerate}
 
 \paragraph{Strong Pruning} The Strong Pruning procedure
 (as shown in Algorithm \ref{alg:approx:gwsp})
 is a simple recursive procedure which walks the tree in post-order calculating the net-worths
 of all subtrees. If --along the way-- a subtree is encountered which is connected with an
  edge with higher cost than the net-worth of the subtree, then the subtree is discarded.
 \begin{algorithm}[h!]
   \begin{algorithmic}[1]
     \Input{Feasible solution to (GW-IP), $T$ rooted in vertex $r$ to $G = (V, E, c, p)$} 
     \Output{A tree $T' \subseteq E$.}
     \State $T' \gets T$
     \Procedure{StrongPrune}{r}
     \State $nw(r) \gets p_r$
     \For{$v \in \Call{Children}{r}$}
     \State \Call{StrongPrune}{v}
     \If{$nw(v) \leq c_{rv}$}
     \State $T' \gets T' \setminus \{(r,v)\}$
     \Else
     \State $nw(r) \gets nw(r) + nw(v) - c_{rv}$
     \EndIf
     \EndFor
   \EndProcedure
 \end{algorithmic}
 \caption{Strong pruning for the GW Algorithm.}\label{alg:approx:gwsp}
\end{algorithm}

The Strong Pruning procedure clearly runs in $O(n)$, which is an improvement on the $O(n^2)$
guarantee for the pruning procedure presented by \cite{goemans1995general}. Additionally,
Strong Pruning does not require any additional information about the candidate solution, $T$,
which is in contrast with the labels required by the GW Pruning procedure.

However, computational experiments performed by \cite{Johnson:2000:PCS:338219.338637}
do not report any significant speedup gained by replacing Strong Pruning in the GW-Algorithm
as the algorithm is dominated by its $O(n^2 \log n)$ growing phase.

However, they do report
improvements in objective value. On 12 instances based on street maps,
the vanilla GW Algorithm producing objective values which are in the range of $2-9\%$
than those produced by the GW Algorithm modified with Strong Pruning. On randomly
 generated instances, this different is reported to be up to $20\%$.
\paragraph{GW for the unrooted PCSTP}
The GW Algorithm is stated and its approximation ration proved
for the rooted variant of the PCSTP. While these apply directly to unrooted variants
when run for all possible roots, this comes with
a worse $O(n^3 \log n)$ runtime.

\cite{Johnson:2000:PCS:338219.338637} verify that if the growth phase of the GW Algorithm
is replaced with a procedure which does not treat the root vertex as special, both the
approximation ratio and asymptotic runtime are still valid.

Computationally, however, they report $1.5-2.0$ slowdown for a unrooted GW Algorithm. They also
report that running the rooted GW Algorithm repeatedly with randomly selected roots give slightly
better objective value (less than $1\%$ difference) than the unrooted variant, although at a starkly
higher cost.
\paragraph{Prize Pertubation} The third thing investigated by \cite{Johnson:2000:PCS:338219.338637} was
the effect of petubing vertex prizes by some constant $\alpha$, that is setting
$$p_v \gets \alpha p_v$$
for all $v \in V$. This ``tricks'' the GW Algorithm into over/undervaluing the importance of gathering prizes,
 leading to altered behaviour.

 Experiments performed by \citeauthor{Johnson:2000:PCS:338219.338637} resulted in $\alpha \approx 0.85$ giving a $\sim 2\%$
 lower objective values compared to $\alpha = 1$ for the GW Algorithm with Strong Pruning on random instances.
  This kind of prize pertubation is explored further by \cite{canuto2001local} -- see Section x \todo{Cross-Reference missing}.
 



%%% Local Variables:
%%% TeX-master: "report"
%%% reftex-default-bibliography: ("lit.bib")
%%% End:
