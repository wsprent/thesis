\clearpage
\section{Approximation Algorithms}\label{sec:solving:approx}
An approximation algorithm was the first contribution to research on the \gls{pcstp}. As a by-product of
defining an approximation algorithm for the \gls{ptp}, \citet{Bienstock1993} produced in \citeyear{Bienstock1993} a 3-approximation
algorithm for the \gls{pcstp} based on solving a number of LP relaxations.

However, the biggest result within approximation algorithms for the \gls{pcstp},
must be the Goemans-Williamson Algorithm. Stated in \citeyear{goemans1995general}
by \citet*{goemans1995general}, and later modified by \cite{Johnson:2000:PCS:338219.338637}.

In \citeyear{archer2011improved}, \citet{archer2011improved} presented a $(2- \epsilon)$-approximation algorithm
based on the GW Algorithm, and in \citeyear{hegde2014fast}, \citet{hegde2014fast}
presented a variant of the GW Algorithm with
improved runtime.

In this section, we present the original GW Algorithm as it gives the clearest view of the principles behind
this ``family'' of approximation algorithms.
\subsection{The Goemans-Williamson Algorithm}\label{sec:solving:approx:gw}
\citet*{goemans1995general} presented a primal-dual 2-approximation algorithm for the rooted variant of the
\gls{pcstp}, based on an algorithm for solving the general \textit{constrained forest problem}.

\paragraph{Definitions}
\citeauthor{goemans1995general} stated the \gls{ilp} model (GW-ILP) in Formulation~\ref{form:approx:gw} for the rooted \gls{pcstp}.

 \begin{formulation}[h!]
   \begin{subequations}
     \begin{alignat}{3} 
       &\underset{\bd x, \bd z}{\text{minimize}}
       & & \sum_{e \in E} c_e x_e + \sum_{X \subset V; r \not\in X} z_X \left( \sum_{v \in X} p_v \right) & \\
       & \text{subject to}\quad
       & & x(\delta(S)) + \sum_{X \supseteq S}z_X \geq 1 \qquad&& \forall S \subset V; r \not\in S \label{form:approx:gw:cut}\\
       &&& \sum_{S \subset V; r \not\in S} z_X \leq 1 && \label{form:approx:gw:cz}\\
       &&& x_e \in \BB  && \forall e \in E \\
       &&& z_X \in \BB  && \forall S \subset V; r \not\in S
     \end{alignat}\label{form:approx:gw}
   \end{subequations}
   \caption{(GW-IP) formulation of the \gls{pcstp}.}
 \end{formulation}

 GW-IP has two decision vectors. The decision vector $\bd x \in \BB^{|E|}$ denotes which edges are contained in the solution,
 and the decision vector $z \in \BB^{|\mathcal P (V \setminus \{r\}) |}$, we have that $z_X$ is
 $1$ when iff. all $v \in X$ are not part of a feasible solution. Clearly, any optimal solution will have $z_X = 1$ for just a single
 $X \subset V$. However, it is still enforced by constraint (\ref{form:approx:gw:cz}).

 Constraint (\ref{form:approx:gw:cut})
 ensures that the solution is a connected subgraph of $G$ by requiring that for each subset $S$, the cut defined by $S$ must either be
 bridged by an edge in a feasible solution or $S$ must be counted as not included.

 \begin{formulation}[h!]
   \begin{subequations}
     \begin{alignat}{3} 
       &\underset{x, z}{\text{maximize}}
       & & \sum_{S \subset V; r \not\in S} y_S & \\
       & \text{subject to}\quad
       & & \sum_{S: e \in \delta(S)} y_S \leq c_e \qquad&& \forall e \in E \label{form:gw:pe}\\
       &&& \sum_{S \subseteq X} y_S \leq \sum_{v \in X} p_v  && \forall X \subset V; r \not\in X \label{form:gw:pv} \\
       &&& y_S \geq 0  && \forall S \subset V; r \not\in S
     \end{alignat}\label{form:approx:gw:dual}
   \end{subequations}
   \caption{(GW-D): Dual of the LP relaxation of (GW-ILP) from Formulation~\ref{form:approx:gw}.}
 \end{formulation}
 
 The dual to the LP relaxation of GW-IP is the linear program in \ref{form:approx:gw:dual}. We make note of the
 \textit{packing} constraints (\ref{form:gw:pe}) and (\ref{form:gw:pv}), and note that by \textit{complementary slackness}
 \citep{chvatal1983linear}
 of the primal LP, we have for optimal primal solutions to the LP, $(x^*, z^*)$,
 \[x^*_e > 0 \Rightarrow \sum_{S: e \in \delta(S)} y_S = c_e\]
 and
 \[z^*_X > 0 \Rightarrow \sum_{S \subseteq X} y_S = \sum_{v \in X} p_v\mathnormal{.}\]
 
 
 \paragraph{The Algorithm} The GW Algorithm consists of a growing phase and a pruning phase. The growing phase
 builds a feasible solution $T$ by maintaining a set of active and inactive \textit{components} which
 initially are singleton sets which are active if they do not contain the root vertex. The dual objective value is then
  increased by simultaneously increasing
  the dual variable values for all active components until one of the following happen:
 \begin{enumerate}[label=\alph*)]
 \item A packing constraint (\ref{form:gw:pe}) becomes binding for some edge $e = (i,j) \in E$. Then $e$ is added
   to $T$, and the components connected by $e$ are merged into a new component with combined cost.
   The new component is considered active iff the root vertex is not contained in it.
 \item A packing constraint (\ref{form:gw:pv}) becomes binding for some subset $X  \subset V$ (i.e. component). Then $X$ is deactivated
    and all vertices it contains are marked with the label $X$.
 \end{enumerate}
 This continues until there are no active components left; at which point $T$ is a feasible solution to the IP (GW-IP).

 
 \begin{algorithm}[h!]
   \begin{algorithmic}[1]
     \Input{Graph $G = (V, E, c, p)$ and root vertex $r \in V$.}
     \Output{Tree $T' \subseteq E$, and the set of vertices not spanned by $T'$, $X$.}
     \Procedure{GW Algorithm}{}
     \State $T \gets \emptyset$ \label{gw:init-start}
     \State $\mathcal{C} \gets \{\{v\} : v \in V\}$
     \For {$v \in V$}
       \State $d(v) \gets 0$
       \State $w(\{v\}) \gets 0$
       \State Unmark $v$.
       \If {$v = r$}
         \State $\lambda(\{r\}) \gets 0$
       \Else
         \State $\lambda(\{v\}) \gets 1$
       \EndIf
     \EndFor \label{gw:init-end}
     \While{$\{ C \in \mathcal{C} \mid \lambda(C) = 1\} \neq \emptyset$}
     \State $\epsilon_1 \gets \min_{i,j} \frac{c_{ij} - d(i) - d(j)}{\lambda (C_i) + \lambda(C_j)},
     \qquad i \in C_i, j \in C_j, C_i \neq C_j$ \label{gw:select-start}
     \State $\epsilon_2 \gets \min_{w} \sum_{v \in C_w} p_v - w(C_w), \qquad C_w \in \mathcal{C}$
     \State $\epsilon \gets \min(\epsilon_1, \epsilon_2)$ 
     \State $w(C) \gets w(C) + \lambda(C) \epsilon$ for all $C \in \mathcal{C}$ \label{gw:add-component}
     \State $d(v) \gets d(v) + \lambda(C) \epsilon$ for all $v \in C \in \mathcal{C}$ \label{gw:add-vertex}
     \If{$\epsilon = \epsilon_1$}
     \State Add $(i,j)$ to $T$: $T \gets T \cup \{(i,j)\}$
     \State Merge $C_i$ and $C_j$ to $C_{ij}$,
     $\mathcal{C} \gets \mathcal{C} \cup \{C_{ij}\} \setminus \{C_i, C_j\}$
     \State If $r \in C_{ij}$ then $\lambda(C_{ij}) \gets 0$ else $\lambda(C_{ij}) \gets 1$
     \State $w(C_{ij}) \gets w(C_i) + w(C_j)$
     \Else
     \State Deactivate $C_w$, $\lambda(C_w) \gets 0$
     \State Mark all $v \in C_w$ with the label $C_w$
     \EndIf
     \EndWhile
     \State $T' \gets \Call{Prune}{T, Labels}$ 
   \EndProcedure
 \end{algorithmic}
 \caption{The GW Algorithm}\label{approx:gw:alg}
 \end{algorithm}

 We show the growing phase of the GW Algorithm in Algorithm \ref{approx:gw:alg}. Lines \ref{gw:init-start}-\ref{gw:init-end}
 initialise the set of components as described above -- only the component containing the root vertex is set as inactive.
 The lines \ref{gw:select-start}-\ref{gw:add-vertex} then determine the amount, $\epsilon$, which is
 the value that dual decision variables
 can be increased by for all active components
 until a packing constraint becomes binding. Since this behaviour is not directly clear for the definition of the algorithm itself,
  we attempt to give some intuition in the following.

 At the beginning of each iteration we have,
 \[d(v) = \sum_{S: v \in S} y_S, \forall v \in V \mathnormal{.}\]
 This follows directly from line \ref{gw:add-vertex}. Suppose that the corresponding
 packing constraint for the edge $e = (i,j)$ is
  the tightest ``active'' constraint. Notice that we can reformulate the left side of the constraint (\ref{form:gw:pe}) as follows,
 \[\sum_{S: (i,j) \in \delta(S)} y_S = \sum_{S : i \in S} y_S + \sum_{S : j \in S} y_S - \sum_{S : i,j \in S} y_S\mathnormal{.}\]
 Since we are only considering edges which connect components, we clearly must have $\sum_{S : i,j \in S} y_S = 0$
 since any subset containing \textit{both} $i$ and $j$ is yet to be activated.
 Thus the above becomes,
 \[\sum_{S: (i,j) \in \delta(S)} y_S = d(i) + d(j)\mathnormal{.}\]
 Suppose, w.l.o.g., that both $i$ and $j$ belong to active components.
 When the iteration is over, we will have increased $d(i)$ and $d(j)$ by $\epsilon = \frac{c_{ij} - d(i) - d(j)}{2}$.
 Then clearly the packing constraint for $e = (i,j)$ is then binding as demostrated below,
 \begin{align*}
\sum_{S: (i,j) \in \delta(S)} y_S &= d(i) + \left( \frac{c_{ij} - d(i) - d(j)}{2} \right) + d(j) +
                                    \left( \frac{c_{ij} - d(i) - d(j)}{2} \right) \\
                              &= d(i) - d(i) + d(j) - d(j) + c_{ij}
                                    &= c_{ij}\mathnormal{.}
 \end{align*}

 The case where a packing constraint (\ref{form:gw:pv}) is the tightest is similar, and follows directly from the loop
 invariant
 \[w(C) = \sum_{S \subset C} y_S, \forall C \in \mathcal{C}\mathnormal{.}\]

 The pruning phase of the algorithm is described by \citet{goemans1995general}
 as a procedure which removes as many
 edges from $T$ while upholding that:
 \begin{enumerate}
 \item any unlabeled vertex is connected to the root vertex, $r$, and
 \item if a vertex with label $C$ is connected to $r$ then any vertex with
   label $C' \supset C$ must also be connected to $r$.
 \end{enumerate}
 This pruning procedure was improved upon by the ``strong pruning'' procedure introduced by \citet{Johnson:2000:PCS:338219.338637},
  which we will detail in Section~\ref{sec:approx:strongpruning}.
 \paragraph{Analysis}
 The GW Algorithm is proven to produce a result which is less than twice as bad as an optimal solution to (GW-IP), $Z^*_{\text{GW-IP}}$, and thus
 is a 2-approximation algorithm. This follows from the following inequality,
 $$\sum_{e \in T'}c_e + \sum_{v \in X} p_v \overset{\text{(I)}}{=}   \sum_{e \in T'} \sum_{S: e \in \delta(S)} y_S  + \sum_{j} \sum_{S \subseteq C_j} y_S
 \overset{\text{(II)}}{\leq} (2 - \frac{1}{n-1}) \sum_{S \subset V} y_S \overset{\text{(III)}}{\leq} (2 - \frac{1}{n-1}) Z^*_{\text{GW-IP}}\mathnormal{.}$$
 The reader is encouraged to read \citet{goemans1997primal} for the full proof, but it roughly goes as follows:
 \begin{itemize}
 \item (I) holds from primal complimentary slackness. An edge is only added to $T$ if its corresponding packing constraint is binding.
   Additionally, for a vertex to not be spanned by $T$, it must have been deactivated at some point during the
    algorithm (and thus the corresponding packing constraint must have
    been binding).
  \item (II) holds from an induction proof on main loop of the procedure, and
  \item (III) holds from the feasibility of the $y$ vector which is implicitly procduced by the procedure and then weak duality.
 \end{itemize}
 Furthermore, the GW Algorithm is proven to run in $O(n^2 \log n)$.
 \subsection{Modifications to the GW Algorithm}\label{sec:approx:strongpruning}
 \citet{Johnson:2000:PCS:338219.338637} investigated three main modifications to the GW Algorithm:
 \begin{enumerate}
 \item an improved pruning procedure called Strong Pruning,
 \item how the algorithm performs on the unrooted \gls{pcstp}, and
 \item how pertubation of the prize vector affects the algorithm.
 \end{enumerate}
 Furthermore, as a part of these investigations, they produced an implementation of the GW Algorithm.
 In this section, we will detail the Strong Pruning procedure, and touch upon the results from the two other
  investigations.
 \paragraph{Strong Pruning} The Strong Pruning procedure
 (as shown in Algorithm \ref{alg:approx:gwsp})
 is a simple recursive procedure which walks the tree in post-order while calculating the net-worths
 of all subtrees. If --- along the way --- a subtree is encountered which is connected by an
  edge which has higher cost than the net-worth of the subtree, then the subtree is discarded.
 \begin{algorithm}[h!]
   \begin{algorithmic}[1]
     \Input{Feasible solution to (GW-IP), $T$ rooted in vertex $r$ to $G = (V, E, c, p)$} 
     \Output{A tree $T' \subseteq E$.}
     \State $T' \gets T$
     \Procedure{StrongPrune}{r}
     \State $nw(r) \gets p_r$
     \For{$v \in \Call{Children}{r}$}
     \State \Call{StrongPrune}{v}
     \If{$nw(v) \leq c_{rv}$}
     \State $T' \gets T' \setminus \{(r,v)\}$
     \Else
     \State $nw(r) \gets nw(r) + nw(v) - c_{rv}$
     \EndIf
     \EndFor
   \EndProcedure
 \end{algorithmic}
 \caption{Strong pruning for the GW Algorithm.}\label{alg:approx:gwsp}
\end{algorithm}

The Strong Pruning procedure clearly runs in $O(n)$ time, which is an improvement on the $O(n^2)$
guarantee for the pruning procedure presented by \citet{goemans1995general}. Additionally,
Strong Pruning does not require any additional information about the candidate solution, $T$,
which is in contrast with the labels required by the GW Pruning procedure.

However, computational experiments performed by \citet{Johnson:2000:PCS:338219.338637}
do not report any significant speedup gained by replacing GW Pruning with
Strong Pruning in the GW Algorithm
as the algorithm is dominated by its $O(n^2 \log n)$ growing phase.

However, they do report
improvements in objective value. On 12 instances based on street maps,
the vanilla GW Algorithm produced objective values which are in the range of $2-9\%$
worse than those produced by the GW Algorithm modified with Strong Pruning. On randomly
generated instances, this is reported to be up to $20\%$.

Thus, modifying the GW Algorithm to perform Strong Pruning results in better solutions at
 no extra cost.
\paragraph{GW  Algorithm for the unrooted \gls{pcstp}}
The GW Algorithm is stated and its approximation ratio proved
for the rooted variant of the \gls{pcstp}. While these apply directly to unrooted variants
when run for all possible roots, this comes with
a worsened $O(n^3 \log n)$ runtime.

\citet{Johnson:2000:PCS:338219.338637} verify that if the growth phase of the GW Algorithm
is replaced with a procedure which does not treat the root vertex as special, both the
approximation ratio and asymptotic runtime proven for the rooted variant are still valid.

Computationally, however, they report a $1.5-2.0$ slowdown for a unrooted GW Algorithm. They also
report that running the rooted GW Algorithm repeatedly with randomly selected roots give slightly
better objective value (less than $1\%$ difference) than the unrooted variant, although at a starkly
higher cost.
\paragraph{Prize Pertubation} The third thing investigated by \citet{Johnson:2000:PCS:338219.338637} was
the effect of pertubing vertex prizes by some constant $\alpha$, that is setting
\[p_v \gets \alpha p_v\]
for all $v \in V$. This ``tricks'' the GW Algorithm into over/undervaluing the importance of gathering prizes,
 leading to altered behaviour.

 Experiments performed by \citeauthor{Johnson:2000:PCS:338219.338637} resulted in $\alpha \approx 0.85$ giving a $\sim2\%$
 lower objective values compared to $\alpha = 1$ for the GW Algorithm with Strong Pruning on random instances.
  This kind of prize pertubation is explored further by \citet{canuto2001local} -- see Section \ref{sec:canuto-search}.


%%% Local Variables:
%%% TeX-master: "report"
%%% reftex-default-bibliography: ("lit.bib")
%%% End:
