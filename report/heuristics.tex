\section{Heuristics}\label{sec:solving:heuristics}

\subsection{Local Search with Pertubation and Relinking}\label{sec:canuto-search}

\cite{canuto2001local} presented a multi-phase heuristics which makes use of the GW Algorithm.
 Algorithm \ref{alg:heuristics:canuto} sketches the full procedure. 

 The heuristics can be seen as a multi start heuristics which uses the GW Algorithm
 to generate good intitial
 solutions (line \ref{alg:canuto:line:gw})
 which it progressively refines using using a hill climbing local search
 (line \ref{alg:canuto:line:hc}),
 and then a path relinking scheme against a previous good solution
 (line \ref{alg:canuto:line:relink}).
 Pertubation of the prize vector, $p$, is used to push the GW Algorithm to generate
 different solutions on multiple starts, and finally, a
 variable neighbourhood search is applied as a
 post processing step to refine the best candidate solution found.

 \begin{algorithm}[h!]
   \begin{algorithmic}[1]
     \Input{Graph $G = (V, E, c, p)$.}
     \Output{Tree $\tilde{T} \subseteq G$.}
     \Procedure{Canuto-LocalSearch}{}
     \State $E \gets \emptyset$, $\hat{p} \gets p$, $\hat{z}^* \gets \infty$
     \For{$i \in 1...\mathit{max\_iterations}$}
     \State $S \gets \Call{GW-Algorithm}{V,E, c, \hat{p}}$ \label{alg:canuto:line:gw}
     \If{$S$ hasn't been seen in previous iterations}
     \State $S' \gets \Call{Hill-Climbing}{S', G = (V, E, c, p)}$ \label{alg:canuto:line:hc}
     \State Add $S'$ to $E$ if it satisfies the necessary criteria \label{alg:canuto:line:elite}
     \State Pick random $Y \in E$
     \State $S' \gets \Call{Relink}{S', Y}$ \label{alg:canuto:line:relink}
     \If{$c(S') < \hat{z}^*$}
     \State $\hat{z}^* \gets c(S')$
     \State $S^* \gets S'$
     \EndIf
     \EndIf
     \State Pertubate $\hat{p}$
     \EndFor
     \State $S^* \gets \Call{VNS}{S^*, G = (V, E, c, p)}$\label{alg:canuto:line:vns}
     \State \Return $\Call{MST}{G[S^*]}$
     \EndProcedure
 \end{algorithmic}
 \caption{The heuristics defined by \cite{canuto2001local}.}\label{alg:heuristics:canuto}
 \end{algorithm}

 In this section we will
 describe further the characteristics of the local search, the
 path relinking, and the postprocessing used in the algorithm.
\paragraph{Local Search with Pertubations}
\cite{canuto2001local} define their search space as the space of minimum spanning
trees on all graphs induced by a subset of vertices
 $S \subset V$. If $G[S]$ is the subgraph induced by the vertex subset $S \subseteq V$, then
 this search space can formally be defined as
$$\mathcal{S_G} = \{MST(G[S]) : S \subseteq V \}\mathnormal{.}$$
A neighbour to a given MST is then any MST of a subgraph induced by a subset which differs from $S$
 by exactly one vertex, that is,
$$\mathcal{N}(S) = \Big\{MST(G[S']) : S' \subseteq V \Bigm| |S' \triangle S| = 1 \Big\}\mathnormal{.}$$ 
Clearly this neighbourhood structure means that all feasible solutions are reachable
given any initial subset as starting point. Additionally, they define the
$k$'th order neighbourhood as,
$$\mathcal{N}^k(S) = \Big\{MST(G[S']) : S' \subseteq V \Bigm| |S' \triangle S| = k \Big\}\mathnormal{.}$$ 

To reduce clutter we will denote the cost
of a given subset, $S \subseteq V$, of vertices as the cost
of the MST of the graph induced by $S$, that is
$$c(S) = c(MST(G[S]))\mathnormal{.}$$

The search algorithm itself is just a plain hill-climbing algorithm which repeatedly
 moves to the first
 found neighbouring solution with lower cost until it arrives at a local minima.

 To overcome the problem of running into the same local minima repeatedly,
 \cite{canuto2001local} employ two pertubation schemes which are alternately applied
 on even and odds iterations:
 \begin{enumerate}
 \item A fixed percentage of vertices which have appear both in the initial solution generated
   by the GW Algorithm \textit{and}  in the solution subsequently generated by
    the hill climbing procedure have their prizes
    set to zero.
  \item A pertubation factor, $\alpha$, is randomly picked within a fixed interval $[1 -a, 1+a]$
    and prizes are set to $p_v \gets \alpha p_v$ for all $v \in V$.
 \end{enumerate}

\paragraph{Path Relinking}
For every iteration of the main loop, the relinking procedure in Algorithm \ref{heuristics:canuto:relink}
is applied to a solution generated by the hill climbing procedure and 
a random \textit{elite} solution which is picked from the
persistant pool of elite solutions, $E$.

The relinking procedure basically transforms $S$ into $Y$ one vertex at a time in a greedy fashion.
The solution along the way which has the best objective value is then returned.
 \begin{algorithm}[h!]
   \begin{algorithmic}[1]
     \Procedure{Apply}{S, v}
     \If{$v \in S$}
     \State \Return $S \setminus \{v\}$
     \Else
     \State \Return $S \cup \{v\}$
     \EndIf
     \EndProcedure
     \Procedure{Relink}{S, Y}
     \State $D \gets S \triangle Y$
     \State $S' \gets S$
     \State $\tilde{S} \gets S$
     \While{$S' \neq Y$}
     \State $v^* \gets \argmin_{v \in D} c(\Call{Apply}{S,v})$
     \State $S' \gets \Call{Apply}{S', v^*}$
     \State if $c(S') < c(\tilde{S})$ then set $\tilde{S} \gets S'$
     \State $D \gets D \setminus \{v\}$
     \EndWhile
     \State \Return $\tilde{S}$
     \EndProcedure
 \end{algorithmic}
 \caption{The relinking scheme used by \cite{canuto2001local}.}\label{heuristics:canuto:relink}
 \end{algorithm}

 Entry into $E$ for a subset $S$ is dermined by two criteria:
 either we must
 $$c(\tilde{S}) < \min_{S \in E} c(S)\mathnormal{,}$$
 or we must have both
 $$c(\tilde{S}) < \max_{S \in E} c(S)$$
  and that the Hamming distances between the characteristic vectors of $\tilde{S}$
   and \textit{all} solutions in $E$ are greater than $\rho |V|$ for some $\rho \leq 1$. 

\paragraph{Postprocessing}
Variable neighbourhood search (VNS) (See \cite{hansen2010variable}) is applied
to best candidate solution as a post-processing step. In principle, the VNS
procedure (shown in Algorithm \ref{alg:heuristics:canuto:vns}) is a hill climbing
search within a union of the first $k_{max}$ order neighbourhoods with an early stopping
 criteria.
\begin{algorithm}[h!]
   \begin{algorithmic}[1]
     \Procedure{VNS}{S}
     \For{$i \in 1...\mathit{max\_iterations}$}
     \State $k \gets 1$
     \While{$k  \leq k_{max}$}
     \State Pick random $S' \in \mathcal{N}^k(S)$
     \State $\tilde{S} \gets \Call{Hill-Climbing}{S'}$
     \If{$c(\tilde{S}) < c(S)$}
     \State $k \gets 1$
     \State $S \gets \tilde{S}$
     \Else
     \State $k \gets k + 1$
     \EndIf
     \EndWhile
     \EndFor
     \State \Return $S$
     \EndProcedure
 \end{algorithmic}
 \caption{The Variable Neighbourhood Search
   used by \cite{canuto2001local}.}\label{alg:heuristics:canuto:vns}
 \end{algorithm}

 As a postprocessing step, this procedure is only applied to the single best solution generated
  by all the the previous steps.
\paragraph{Evaluation}
Based on the implementation of the GW Algorithm by \cite{Johnson:2000:PCS:338219.338637},
\cite{canuto2001local} performed experiments, detailing the objective function value
and running time of
\begin{itemize}
\item The GW Algorithm,
\item The multi-start local search procedure with pertubations
\item The above combined with path relinking, and
\item The full heuristics.
\end{itemize}

All three ``parts'' of the heuristics are shown to help improve upon the
objective value generated by the previous parts.
But in particular, \citeauthor{canuto2001local} report that adding the
path relinking scheme results in a stark increase
 in number of instances solved to optimality.


%%% Local Variables:
%%% TeX-master: "report"
%%% reftex-default-bibliography: ("lit.bib")
%%% End:
